We give semantics to global types using \emph{partially-ordered
  multi-set} (pomsets for short).
%
Following~\cite{gaifman1987partial}, a \emph{pomset} is an isomorphism
class of labelled partially-ordered sets (lposet) where, fixed a set
of labels $\lset$,
\begin{itemize}
\item an lposet is a triple $(\eset, \leq, \alf)$, with $\eset$ a set
  of events, $\leq$ is a partial order on $\eset$, and
  $\alf: \eset \rightarrow \lset$ a labelling function mapping events
  in $\eset$ to labels in $\lset$;
\item two lposets $(\eset, \leq, \alf)$ and $(\eset', \leq', \alf')$
  are \emph{isomorphic} if there is a bijection
  $\phi: \eset \rightarrow \eset'$ such that
  $\ae \leq \ae' \iff \phi(\ae) \leq' \phi(\ae')$ and
  $\alf = \alf' \circ \phi$.
\end{itemize}
%
Intuitively, the partial order $\leq$ yields a causality relation
among events; for $\ae \neq \ae'$, if $\ae \leq \ae'$ then $\ae'$ is
caused by $\ae$ or, in other words, the occurrence of $\ae'$ must be
preceded by the one of $\ae$ in any execution respecting the order
$\leq$.
%
Note that $\alf$ is not required to be injective: for
$\ae \neq \ae' \in \eset$, $\alf(\ae) = \alf(\ae')$ means that $\ae$
and $\ae'$ model different occurrences of the same action.
%
In the following, $[\eset, \leq, \alf]$ denotes the isomorphism class
of $(\eset, \leq, \alf)$, symbols $\apom,\apom', \dots$ (resp.
$\aR, \aR', \dots$) range over (resp. sets of) pomsets, and we assume
that pomsets $\apom$ contain at least one lposet which will possibly
be referred to as $(\esetof \apom$, $\leqof \apom, \alfof \apom)$.
%
The empty pomset is denoted as $\emptypom$.

An event $\ae$ is an \emph{immediate predecessor} of an event $\ae'$
(or equivalently $\ae'$ is an \emph{immediate successor} of $\ae$) in
a pomset $\apom$ if $\ae \neq \ae'$, $\ae \leqof \apom \ae'$, and for
all $\ae'' \in \esetof \apom$ such that
$\ae \leqof \apom \ae'' \leqof \apom \ae'$ either $\ae = \ae''$ or
$\ae' = \ae''$.
%
We will represent pomsets as (a variant\footnote{Edges of Hasse
  diagrams are usually not oriented; here we use arrows so to draw
  order relations between events also horizontally.} of) Hasse
diagrams of the immediate predecessor relation; for instance, the
pomset
\[
  \left[\{\ae_1,\ae_2,\ae_3,\ae_4\}, \{(\ae_1,\ae_2),(\ae_1,\ae_3),
    (\ae_1,\ae_4), (\ae_4,\ae_5)\},
    \alf% :
    % \begin{cases}
    %   \ae_1 \mapsto \apref[\ptp p][]
    %   \\
    %   \ae_2,\ae_3 \mapsto \apref[][{\amulti q}]
    %   \\
    %   \ae_4 \mapsto \apref[\amulti R][][@][@][.]
    %   \\
    %   \ae_5 \mapsto \apref[][\amulti R][@][@][.]
    % \end{cases}
  \right]
\]
is more conveniently written as
\[
  \pomsetrep{
    \node (out) {$\ae_1$};
    \node[below left = of out] (in1) {$\ae_2$};
    \node[below right = of out] (in2) {$\ae_3$};
    \node[above right = of in2] (out2) {$\ae_4$};
    \node[below = of out2] (in3) {$\ae_5$};
    \draw[->] (out) -- (in1);
    \draw[->] (out) -- (in2);
    \draw[->] (out) -- (out2);
    \draw[->] (out2) -- (in3);
  }[\alf][][node distance = .5cm and .3cm]
  \qquad\text{or}\qquad
  \pomsetrep{
    \node (out) {$\alf(\ae_1)$}; % {$\apref[{\ptp p}][]$};
    \node[below left = of out] (in1) {$\alf(\ae_2)$}; % {$\apref[][{\amulti q}]$};
    \node[below right = of out] (in2) {$\alf(\ae_3)$}; % {$\apref[][{\amulti q}]$};
    \node[above right = of in2] (out2) {$\alf(\ae_4)$}; % {$\apref[\amulti R][][@][@][.]$};
    \node[below = of out2] (in3) {$\alf(\ae_5)$}; % {$\apref[][\amulti R][@][@][.]$};
    \draw[->] (out) -- (in1);
    \draw[->] (out) -- (in2);
    \draw[->] (out) -- (out2);
    \draw[->] (out2) -- (in3);
  }[][][node distance = .5cm and .1cm]
\]
  
% Given an autonomous prefix $\apref$, we let $\pomsetsingle$ to be
% the pomset with a single event labelled by $\apref$.
%
We will use the auxiliary operations on pomsets described below.
%
Define the (disjoint) union of two pomsets $\apom$ and $\apom'$ as
\[
  \apom \pomsetcup \apom' =
  [\esetof{\apom} \uplus \esetof{\apom'},
  \leqof{\apom} \uplus \leqof{\apom'},
  \alfof{\apom} \uplus \alfof{\apom'}]
\]
The sequential composition of $\apom$ and $\apom'$ composes an
instance of $\apom'$ with every maximal event of $\apom$.
%
Formally, for each $\ae \in \max \apom$, let
$\apom'_{\ae} = [\{\ae\} \times \eset_{\apom'}, \leq, \alf]$ where
$(\ae,\ae_1) \leq (\ae,\ae_2) \iff \ae_1 \leqof{\apom'} \ae_2$ and
$\alfof{\apom'_{\ae}} = (\ae,\ae') \mapsto \alfof{\apom'}(\ae')$ for
all $\ae' \in \esetof{\apom'}$.
%
Observe that $\apom'_{\ae}$ is isomorphic to $\apom'$, and define
\[
  \rseq[\apom][\apom'] = 
  [\esetof{\apom} \uplus \bigcup_{\ae \in \max \apom}\esetof{\apom'_{\ae}},
  \leq,
  \alfof{\apom} \uplus \bigcup_{\ae \in \max \apom}\alfof{\apom'_{\ae}}]
\]
where $\leq$ is the reflexo-transitive closure of
$\leqof{\apom} \cup \bigcup_{\ae \in \max \apom}\leqof{\apom'_{\ae}}
\cup \leqof{\apom'} \cup \{(\ae,(\ae,\ae')) \sst \ae \in
\esetof{\apom} \land (\ae,\ae') \in \esetof{\apom'_{\ae}} \land
\roles{\alfof{\apom}(\ae)} = \roles{\alfof{\apom'}(\ae')}\}$ (recall
that the labels of events are autonomous prefixes for which
$\roles{}$ is a singleton).
%
%\eMnote{quando $\leq = \leqof{\apom} \uplus \leqof{\apom'}$ abbiamo il parallelo}
%
Finally, for $h \geq 1$ and a pomset $\apom$ we define
\[
  \apom^h =
  \begin{cases}
    r & \text{if } h = 1
    \\
    r \pomsetcup r^{h-1} & \text{if } h > 1
  \end{cases}
\]
The operation $\_^h$ extends element-wise to sets of pomsets.

The labels of the events in our semantics are (decorations of)
autonomous prefixes $\apref$: labels are either just autonomous
prefixes $\apref$ or of the form $\alabel$ with $\apref$ an autonomous
prefix.
%
Intuitively, a label $\alabel[@][][\arole][@][@][.]$
(resp. $\alabel[@][\arole][][@][@][.]$) represents the fact that the
$i^\mathit{th}$ instance of $\arole$ consumed (resp. produced) a tuple
of type $\atuple$.
%
Labels $\apref$ not prefixed with $[\_]$ simply specify that the event
can be performed by any instance of the role in $\apref$.
% 
The semantics of a global type builds upon \emph{basic pomsets} $\bp$
of prefixes $\apref$; in fact, the semantics of a global type is build
by taking sets of disjoint unions of basic pomsets of prefixes
sequentially composed together.
%
Let $h \geq 1$ and define
$\bp = \left\{ \pomsetrep{\node {$\alabel$}}[] \right\}$ where, for
$\arole,\arole' \in \roleset$,
\begin{align*}
  \bp[@][{\apref[\arole][\arole'][@][@][.]}] =
  &
    \begin{cases}
      \left\{
        \pomsetrep{
        \node (out) {$\alabel[@][\arole][][@][@][.]$};
        \node[below = of out] (in) {$\alabel[@][][\arole'][@][@][.]$};
        \draw[->] (out) -- (in);
      }[]\right\}
      &
      \text{if } \arole' \in \participants \text{ or $\atuple$ generates} 
      \\
      \bigcup_{h \geq 1}
      \left\{
        \pomsetrep{
          \node (out) {$\alabel[@][\arole][][@][@][.]$};
          \node[below left = of out] (rd1) {$\ae_1$};
          \node[below = of out] (dots) {};
          \node[below right = of out] (rd2) {$\ae_h$};
          \draw[->] (out) -- (rd1);
          \draw[->] (out) -- (rd2);
          \draw[dotted] (rd1) -- (rd2);
        }[{\big(\scriptsize
        \begin{array}[c]{l}
          \ae_j \mapsto \\ \apref[][\arole'][@][@][.]
        \end{array}
        \big)_{1 \leq j \leq h}}][][node distance = 1cm and -.25cm]\right\}
      \cup
      \left\{
        \pomsetrep{\node {$\alabel[@][\arole][][@][@][.]$}}[]
      \right\}
    &
    \text{otherwise}
    \end{cases}
\end{align*}
%
and
%
\begin{align*}
  \bp[@][{\apref[\arole][\arole']}] =
  &
    \left\{
    \pomsetrep{
    \node (out) {$\alabel[@][\arole][]$};
    \node[below left = of out] (rd1) {$\ae_1$};
    \node[below = of out] (dots) {};
    \node[below right = of out] (rd2) {$\ae_h$};
    \node[below = of dots] (in) {$\alabel[@][][\arole']$};
    \draw[->] (out) -- (rd1);
    \draw[->] (out) -- (rd2);
    \draw[->] (rd1) -- (in);
    \draw[->] (rd2) -- (in);
    \draw[dotted] (rd1) -- (rd2);
    }[(\ae_j \mapsto {\apref[][\arole'][@][@][.]})_{1 \leq j \leq h}][][node distance = 1cm and -.25cm]\right\}
    \cup
    \left\{
    \pomsetrep{
    \node (out) {$\alabel[@][\arole][]$};
    \node[below = of out] (in) {$\alabel[@][][\arole']$};
    \draw[->] (out) -- (in);
    }[]
    \right\}
\end{align*}

We first give semantics of prefixes.
\begin{align*}
  \ksem{\apref} =
  &
    \begin{cases}
      \bp[1][\apref]
      & \text{if $\apref$ autonomous } \land \roles{\apref} \subseteq \participants
      \\
      \bigcup_{i \geq 1}\bp[@][\apref]
      % \bigcup_{h \geq 1}\big\{\pomsetrep{
      % \node {$\alabel[1]\cdots\alabel[h]$};
      % }[]\big\},
      & \text{if $\apref$ autonomous } \land \roles{\apref} \not\subseteq \participants
    \end{cases}
%  \qquad \text{if $\apref$ autonomous}
  \\
  \ksem{\apref[\arole][\arole'][@][@][.]} =
  &
    \begin{cases}
      \bp[1][{\apref[\arole][\arole'][@][@][.]}]
      & \text{if } \arole \in \participants
      \\
      \bigcup_{i \geq 1}\bp[@][{\apref[\arole][\arole'][@][@][.]}]
      & \text{otherwise}
    \end{cases}
  \\
  \ksem{\apref[\arole][\arole']} =
  &
    \begin{cases}
      \bp[1][{\apref[\arole][\arole']}]
      & \text{if } \arole \in \participants
      \\
      \bigcup_{i \geq 1}\big(\bp[@][{\apref[\arole][\arole']}]\big)
      & \text{otherwise}
    \end{cases}
\end{align*}

We can now give the semantics of (closed) global types.
%
As customary in other choreographic approaches, we define the
semantics of \emph{well-formed} global views of choreographies.
%
The well-formedness of our global views presents some peculiarities
respect to standard notions.
%
In particular, \emph{well-sequencedness} and \emph{well-branchedness}
of our global types need some care.

We tackle well-sequencedness first.
%
Consider the following choreography
\begin{align}\label{eq:seq}
  \apref[\arole_1][\arole_2][{\asort[str] \cdot \wildcard}][\aloc] \seqop
  \apref[\arole_2][\arole_3][{\asort[str] \cdot \asort[int]}][\aloc]
\end{align}
%
where an instance of $\arole_2$ transforms a pair generated by
$\arole_1$ into a pair for $\arole_3$.
%
The choreography specified in \eqref{eq:seq} may be violated when
$\arole_1$ generates a tuple of type $\asort[str] \cdot \asort[int]$.
%
In fact, such a tuple could match the type consumed by $\arole_3$ and
therefore $\arole_3$ could \quo{steal} the tuple from $\arole_2$.
%
The problem is due to the fact that ($i$) the input could match the
tuple sent by $\arole_1$ and ($ii$) $\arole_3$ can consume the tuple
sent by $\arole_1$ because it is on the same locality $\aloc$.
%
More generally, write $(\atuple, \aloc) \in \aK$ when there is a
prefix in $\aK$ whose tuple is $\atuple$ and whose location is
$\aloc$; we say that $(\atuple, \aloc)$ is \emph{local} to $\aK$ if
either of the following holds:
\begin{itemize}
\item $\aK = \asum[@][@][][]$ there is $i \in I$ such that either
  $(\atuple, \aloc)$ is local to $\aK_i$ or $\apref_i$ outputs
  $\atuple$ at $\aloc$ and there is $\atuple'$ in an input from
  $\aloc$ in $\aK$ and $\atuple \matches \atuple'$
\item $\aK = \aK_1 \seqop \aK_2$ either $(\atuple, \aloc)$ is local to
  $\aK_1$ or   $(\atuple, \aloc)$ is local to $\aK_2$
\item $\aK = \arec[@][\aK']$ $(\atuple, \aloc)$ is local to $\aK'$
\end{itemize}
%
A choreography $\aK_1 \seqop \aK_2$ is \emph{conflict-free} if for
$i \neq j \in \{1,2\}$
%
\begin{itemize}
\item for all $(\atuple,\aloc)$ local to $\aK_i$ and for all
  $(\atuple', \aloc) \in \aK_j$, $\atuple \matches \atuple'$ implies
  $(\atuple', \aloc)$ is in a read-only prefix in $\aK_j$
\item for all $(\atuple,\aloc)$ in a non-local consuming-input prefix
  of $\aK_i$ and for all $(\atuple', \aloc)$ in an output prefix of
  $\aK_j$, $\atuple \matches \atuple'$ implies $(\atuple',\aloc)$ is
  in a consuming output prefix in $\aK_j$.
\end{itemize}
% such that $\atuple_i$ occurs in a prefix of $\aK_i$ both at a locality $\aloc$, \eMnote{Problema: $\aloc$ vs $\alocvar$}

Finally, the equations giving the semantics of global types are:
\begin{align*}
  % 
  \ksem{\asum[@][@][][]} =
  &
    \begin{cases}
      \{ \emptypom \} & \text{if } I = \emptyset
      \\
      \bigcup_{\apom \in \ksem{\apref_i},\apom' \in \ksem{\aK_i}} \rseq[\apom][\apom']
      & \wb[{\big\{\bigcup_{i \in I}\apref_i\mkop{.}\aK_i\big\}}]
      \\
      \mathit{undefined} & \text{otherwise}
    \end{cases}
  \\
  \ksem{\aK_1 \seqop \aK_2} =
  &
    \begin{cases}
      \rseq[\ksem{\aK_1}][\ksem{\aK_2}] & \text{if } \ws[\aK_1][\aK_2]
      \\
      \mathit{undefined} & \text{otherwise}
    \end{cases}
  \\
  %
  \ksem{\arec} =
  &
    \begin{cases}
      \bigcup_{h \geq 0} \ksem{\unfold h \arec} & \text{if } \ws[\aK\sust X \nil][\aK\sust X \nil]
      \\
      \mathit{undefined} & \text{otherwise}
    \end{cases}
  \\
  & \text{where } \unfold h \arec =
    \begin{cases}
      \nil & \text{if } h = 0
      \\
      \aK \seqop \unfold{h-1}{\arec} & \text{otherwise}
    \end{cases}
\end{align*}
where predicates $\wb[]$ and $\ws[][]$ respectively check the
\emph{well-branchedness} and \emph{well-sequencedness} conditions.
%
The former is defined below whereas $\ws[\aK_1][\aK_2]$ holds when
$\aK_1 \seqop \aK_2$ is conflict-free and any participant $\ptp a$
performing actions in $\aK_2$ acts in $\aK_1$ as well and all the
first actions of $\ptp a$ in $\aK_2$ causally depend on all last
actions of $\ptp a$ in $\aK_1$ in the pomset
$\rseq[\ksem{\aK_1}][{\ksem{\aK_2}}]$.

% To avoid this problem we introduce \emph{run-time} tuples
We now consider well-branchedness, the other condition of
well-formedness.
%
Well-branchedness requires two conditions: single selector and knowledge of 
choices.
%
This can be formalised by requiring that one process in the choice is
\emph{active}, namely it selects the branch to take, while the others
are \emph{passive}, namely they are informed of the chosed branch by
inputting some information that unambiguously identifies each branch
of the choice.
%
%
We syntactically enforce uniqueness of selectors; a choice with
several branches, takes the form
\begin{align}
  \asum\label{eq:ch}
\end{align}
namely the instance of $\arole$ act a \emph{unique} selectors.
%
This is just for simplicity as we could adopt definitions similar to
the ones in \cite{gt16,gt17} at the cost of higher technical
complexity.

Intuitively, a passive instance (for example one enacting role
$\arole_i$) in \eqref{eq:ch} has to be able to ascertain which branch
the selector decided when the choice was taken.
%
A simple way to ensure this is to require that the first input actions
of their roles in the branches are \quo{disjoint} (\ie\ non matching
tuples or different locations).
%
This is just for simplicity as we could adopt more general definitions
similar to the one based on divergence points in \cite{gt16,gt17}.

The conditions on active and passive processes alone are not enough:
in our framework, the notion of well-branchedness is slightly
complicated by the presence of multiroles.
%
For instance, even assuming unique selectors, many instances of a
selector role could exercise choices concurrently.
%
This may create confusion if different branches generate matching
tuples on a locality as illustrated by next example.
\begin{example}\label{ex:nonwb}
  Let
  \begin{align*}
    \aK_\mathrm{bad} = & \apref[\amulti A][\amulti B][{\asort[int]}][\aloc] \mkop{.} K_1 \chop \apref[\amulti A][\amulti B][{\asort[str]}][\aloc] \mkop{.} K_2
    \\
    \aK_1 = & \apref[\amulti B][\amulti C][{\asort[str]}][\aloc] \mkop{.} \apref[\amulti C][\amulti B][{\asort[bool]}][\aloc]
    \\
    \aK_2 = & \apref[\amulti B][\amulti C][{\asort[bool]}][\aloc]
  \end{align*}
  In $\aK_\mathrm{bad}$ confusion may arise that may alter the
  intended data flow when two groups of instances of $\amulti A$,
  $\amulti B$, and $\amulti C$ execute the choice so that each group
  takes a different branch.
  %
  In fact, the instance of $\amulti C$ executing $K_2$ in the second
  branch may receive the boolean that the instance of $\amulti C$ in
  $K_1$ executing the first branch generates for $\amulti B$.
  %
  \eMnote{This type of confusion does not seem to introduce deadlocks}
  %
  \finex
\end{example}
%
Therefore we require that tuples types in different branches of a
choice do not match when they are at the same locality and that if a
branch of a choice involves a participant then none of the branches of
the choice involves multiroles.
%
This condition, dubbed \emph{confusion-free branching} ensures that
different \quo{groups} of instances involved in concurrent resolutions
of a choice do not \quo{interfere} with each other and such groups can
execute the choice many times only if none of the processes implements
a participant.
%
We remark that the above condition is not a limitation; in fact, we
can pre-process branches of choices by adding an extra field in all
tuples of the branch so to unequivocally identify on which branch the
tuple type is used.

To sum up, a choice as in~\eqref{eq:ch} is \emph{well-branched},
written $\wb[{\big\{\bigcup_{i \in I}\apref_i\mkop{.}\aK_i\big\}}]$,
when it is confusion free, there is a unique active role, all other
roles are passive.


  %
% The notion of well formedness seems to guarantee a much weaker
% notion of correctness than the usual ones.
% %
% Firstly, note that well-formedness here does not imply deadlock
% freedom (but this is fine since we are not interested in properties
% of the control of processes).
% %
%
% A example could be the following:
%
%
\eMnote{We can tackle this issue in two different way (at least): one way is
to statically ensure that tuples generated on one branch do not match
any other tuple on another branch; another way is to modify the
semantics of the choice by implicitly inserting an extra field in each
tuple with a unique identifier of each branch.
\\
Under the current interpretation of our semantics, probably the notion
of correctness we can guarantee is that any set of instances taking a
choice will fully execute a branch.
}




%%% Local Variables:
%%% mode: latex
%%% TeX-master: "main"
%%% End:

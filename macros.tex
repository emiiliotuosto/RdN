% !TEX root =  main.tex
%%% Macros for Klaimographies ;-)

\newcommandx{\asort}[1][1 = s, usedefault=@]{\mathbf{#1}}
\newcommand{\wildcard}{\star}
\newcommand{\atuple}{\texttt{t}}
\newcommand{\aloc}[1][l]{\texttt{\color{blue}l}}
\newcommand{\locset}{\mathcal{L}}
\newcommand{\mkop}[1]{\textcolor{ForestGreen}{#1}}
\newcommand{\at}{\mkop{\tiny @}}
\newcommand{\outop}{\mkop{!}}
\newcommand{\parop}{\mkop{\mid}}
\newcommand{\chop}{\mkop{+}}
\newcommand{\inop}{\mkop{?}}
\newcommand{\toop}{\mkop{\to}}
\newcommand{\rec}{\mkop{rec}\xspace}
\newcommand{\nil}{\mkop{\textbf{0}}}
\newcommandx{\atupleat}[3][1={}, 2=\atuple, 3=\aloc, usedefault=@]{
  \ifempty{#1}{({#2}) \at {#3}}{{#2} \at {#3}}
}
\newcommand{\clashes}{\sharp}
\newcommand{\matches}{\bowtie}
\newcommand{\arole}{\rho}
\newcommand{\ptp}[1]{{\mathsf{\MakeLowercase{#1}}}}
\newcommand{\amulti}[1]{{\mathsf{\MakeUppercase{#1}}}}
\newcommand{\roleset}{\mathcal{R}}
\newcommand{\participants}{\mathcal{P}}
\newcommand{\multiroles}{\mathcal{M}}
%\newcommand{\unknownstar}{\mkop{\ast}}
%\newcommand{\unknown}{\mkop{\textbf{?}}}
\newcommandx{\unknownstar}[1][1=P]{{\amulti {#1}}^{\mkop{\ast}}}
\newcommandx{\unknown}[1][1= P]{{\amulti {#1}}^{\mkop{\textbf{?}}}}
\newcommand{\aK}[1][K]{\texttt{\textcolor{blue}#1}}
\newcommand{\aL}[1][L]{\texttt{\textcolor{orange}#1}}
\newcommandx{\proj}[2][1=\aK,2=\arole,usedefault=@]{
  \ifempty{#1}{\_}{#1} \downharpoonleft_{\ifempty{#2}{\_}{#2}}
}
\newcommandx{\aint}[3][1=\arole, 2=\arole', 3=\aK, usedefault=@]{
  {#1} \toop {#2} \mkop{:} {#3}
}
\newcommandx{\apref}[5][1={},2={},3=\atuple,4=\aloc,5={},usedefault=@]{
  \ifempty{#1}{
    \ifempty{#2}{\pi}{
      \ifempty{#5}{\ain[{#2}][{#3}][{#4}]}{\ard[{#2}][{#3}][{#4}]}
    }
  }{
    \ifempty{#2}{
      \aout[{#1}][{#3}][{#4}]
    }{
      {#1} \toop {#2} \mkop{:} \ifempty{#5}{({#3})\at{#4}}{{#3}\at{#4}}
    }
  }
}
\newcommandx{\asum}[4][1=i,2=I,3=\apref,4=\aK,usedefault=@]{
  \displaystyle{\sum_{#1 \ifempty{#2}{}{\in #2}}}{#3_{#1} \ifempty{#4}{}{. #4_{#1}}}
}
\newcommandx{\aout}[3][1=\arole,2=\atuple,3=\aloc,usedefault=@]{
  \ifempty{#1}{#2 \outop #3}{{#1} \outop {#2} \at {#3}}
}
\newcommandx{\ain}[3][1=\arole,2=\atuple,3=\aloc,,usedefault=@]{
  \ifempty{#1}{(#2) \inop #3}{{#1} \inop {(#2)} \at {#3}}
}
\newcommandx{\ard}[3][1=\arole,2=\atuple,3=\aloc,,usedefault=@]{
  \ifempty{#1}{#2 \inop #3}{{#1} \inop {#2} \at {#3}}
}
\newcommandx{\arec}[2][1=X,2=\aK,usedefault=@]{
  \rec\ #1 \mkop{.} #2
}
\newcommandx{\aLsum}[4][1=i,2=I,3=\arole,4=\aL,usedefault=@]{
    \displaystyle{\sum_{#1 \ifempty{#2}{}{\in #2}}}{#3_{#1} : {#4_{#1}}}
}

\newcommand{\selectors}[1]{\it{sel}(#1)}
\newcommand{\roles}[1]{\it{roles}(#1)}
\newcommand{\eqR}{\sim}

%%% Local Variables:
%%% mode: latex
%%% TeX-master: "main"
%%% End:

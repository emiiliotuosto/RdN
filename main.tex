\documentclass[runningheads,a4paper]{llncs}


%\usepackage{mathpartir}

\usepackage{amsmath}
\usepackage{amssymb}
\usepackage{mathtools}
\usepackage{stmaryrd}
\usepackage{graphicx}
\usepackage{subfigure}
\newcommand{\erlcode}[1]{\textcolor{blue}{\mintinline{erlang}|#1|}}

\usepackage{multicol}
%\usepackage{hyperref}
\usepackage[usenames,dvipsnames,svgnames,table,monochrome]{xcolor} % must be loaded before tikz %% use monochrome option to remove colors
\usepackage{tikz}
\usepackage{xstring}
\usepackage{graphics,framed}
\usepackage[capitalise]{cleveref}
\crefformat{enumi}{condition~#2#1#3}
\crefname{fact}{Fact}{Facts}
\Crefname{fact}{Fact}{Facts}
\crefformat{fact}{Fact~#2#1#3}
\usepackage{xargs}
\usepackage{listings}
\usepackage{etoolbox}
\usepackage{bussproofs}
\usepackage{etoolbox}
\usepackage{minted}
\AtBeginEnvironment{minted}{\fontsize{7.1}{7.1}\selectfont}
% \setminted{linenos}
%\usepackage{showlabels}
\usepackage{pslatex}
\usepackage{wrapfig}

%\usepackage{xypic}
\usepackage{xspace}

\usepackage[inline]{enumitem}

% !TEX root =  main.tex
%%% Macros usuful almost everywhere


% such that

%%% My flags
\newif\ifemi
%%% My flags end

\newcommand{\nota}[2][TODO]{::\textcolor{red}{#1}:: \color{cyan} #2}



\DeclareGraphicsExtensions{%
    .png,.PNG,%
    .pdf,.PDF,%
    .jpg,.mps,.jpeg,.jbig2,.jb2,.JPG,.JPEG,.JBIG2,.JB2}


\usepackage[draft]{fixme}
\fxusetheme{color}
% \FXRegisterAuthor{eM}{aeM}{\color{orange} {\underline{eM}}}

\usepackage{bm}
%\usepackage[disable]{todonotes}

\usepackage[normalem]{ulem} % underline command breaks over line ends

\usepackage{xifthen}        % for conditional commands
\newcommand{\ifempty}[3]{%
  \ifthenelse{\isempty{#1}}{#2}{#3}%
}


\newcommand{\mkfun}[4][\colorFun]{
  \newcommand{#2}[1][#4]{
    {#1\textsf{#3}}
    \ifempty{##1}{}{
      \big({##1}\big)}
  }
}

\newcommand{\mkuop}[4][\colorFun]{
  \newcommand{#2}[1][#4]{
    {#1\textsf{#3}}
    \ifempty{##1}{}{
      \, {##1}}
  }
}

%%% Meta comments
\newcommand{\hidden}[1]{}

\newcommand{\cf}[2]{
  \fontsize{#1}{#1}{\selectfont{#2}}
}
\ifemi
\usepackage{showlabels}
\renewcommand{\showlabelsetlabel}[1]{{\textcolor{ForestGreen}{\showlabelfont{\tiny #1}}}}
\newcommand{\emi}[2]{
  \marginpar{\fcolorbox{red}{shadecolor}{\cf{#1}{{#2}}}}
}
\newcommand{\emic}[2]{\par
  \fcolorbox{red}{shadecolor}{\parbox{\linewidth}{ 
      \color{gray}
      \begin{description}
      \item[{\color{blue} #2}]{\sf #1}
      \end{description}}}
}
\else
\newcommand{\emi}[2]{}
\newcommand{\emic}[2]{}{}
\fi

\newcommand{\marginnote}[1]{
  {\makebox[0pt]{\color{orange}}}
  \marginpar{\parbox{2cm}{\flushleft \tiny \color{orange}{#1}}}
}
\newcommand{\side}[2][6]{
  \marginpar{\cf{#1}{\hl{#2}}}
}
%%% Meta comments end



%%% Maths & logic
\newcommand{\mkset}[1]{\big\{ {#1} \big\}} %%% formerly called \ASET
\newcommand{\partsof}[1]{2^{#1}} %%% formerly called \ASET
\newcommand{\setof}[1]{\mkset{#1}}
\newcommand{\setdef}[2]{\mkset{#1 \sst #2}}
\newcommand{\card}[1]{\|{#1}\|}
\newcommand{\sst}{\;\big|\;}
\newcommand{\qst}{\;\colon\;} %such that
%\newcommand{\dom}[1]{\operatorname{dom} {#1}}
\newcommand{\dom}[1]{\textit{dom}(#1)}

%\newcommand{\cod}[1]{\operatorname{cod} {#1}}
\newcommand{\cod}[1]{\textit{cod}(#1)}

\newcommand\myenddef{~\hfill \ensuremath{\diamond}}
\newcommand{\conf}[1]{\ensuremath{\langle {#1} \rangle}}
\newcommand{\tuple}[1]{\conf{#1}}
\newcommand{\acsconf}[2]{\conf{{#1} \ ; \ {#2}}}
%\renewcommand{\vec}[1]{\overset{\to}{#1}}
\newcommand{\ith}[2]{\vec{#1}[{#2}]}
\newcommand{\bnfdef}{\ ::=\ }
\newcommand{\bnfmid}{\;\ \big|\ \;}
\newcommand{\sep}{\;\bnfmid\;}
\newcommand{\emptyword}{\varepsilon}
\newcommand{\aword}{\varphi}
\newcommand{\langof}[1]{\mathcal{L}(#1)}
\newcommand{\sem}[2][]{\mbox{\ensuremath{\llbracket{#2}\rrbracket_{#1}}}}
\newcommand{\qmmdef}{\quad \mmdef \quad}
\newcommand{\qqmmdef}{\qquad \mmdef \qquad}
\newcommand{\qqand}{\qquad\text{and}\qquad}
\newcommand{\qand}{\quad\text{and}\quad}
\newcommand{\verum}{\mathtt{tt}}
\newcommand{\falsus}{\mathtt{ff}}
\newcommand{\nat}{\mathbb{N}}
\newcommand{\upd}[3]{{#1}[{#2} \mapsto {#3}]}

%%% Maths & logic end



%%% Typographic style
\newcommand{\defrule}[1]{%
  % \hypertarget{rule:{#1}}
  {%
    \text{\scriptsize[\textbf{#1}]}%
  }%
}
\newcommand{\ie}{\text{i.e.}}
\newcommand{\cfw}{\text{cf.}}
\newcommand{\eg}{\text{e.g.}}
\newcommand{\aka}{\text{a.k.a.}}
\newcommand{\mypar}[1]{\noindent\paragraph{\textbf{#1}}\ }
\newcommand{\secref}[1]{\S~\ref{#1}}
\newcommand{\figref}[1]{Fig.~\ref{#1}}
\newcommand{\exref}[1]{Ex.~\ref{#1}}
\newcommand{\defref}[1]{Definition~\ref{#1}}
\newcommand{\exeref}[1]{Exercise~\ref{#1}}
\newcommand{\rmkref}[1]{Remark~\ref{#1}}
\newtheorem{rmk}{Remark}{\bfseries}{\rmfamily}
\newtheorem{exe}{Exercise}{\bfseries}{\rmfamily}
\newcommand{\sidebyside}[2]{
  \begin{tabular}{ll}
    \begin{minipage}{.5\linewidth} {#1}  \end{minipage}
    &
    \begin{minipage}{.5\linewidth} {#2}  \end{minipage}
  \end{tabular}
}
\newcommand{\eqdef}{\ \triangleq\ }
\newcommand{\mmdef}{\eqdef}
%%% Typographic style end



%%% Arrows
\def\ured{\rightarrow}
\def\tred#1{\overstackrel{\rightarrowfill}{#1}}
\def\iured#1{\rightarrow_{\descr{\scriptstyle #1}}}
\def\itred#1#2{\mathrel{{\overstackrel{\rightarrowfill}{#2}}_{\descr
      {\scriptstyle #1}}}}
\def\sytr#1{\overstackrel{\longmapsto}{#1}}
\def\tr#1{\overstackrel{\rightarrowfill}{#1}}
\def\actr#1#2{\overstackrel{\rightarrowfill}{\act {#1}{#2}}}
\def\iactr#1#2#3{\overstackrel{\rightarrowfill}{\act {#2}{#3}}_{\descr
    {\scriptstyle #1}}}
\def\dlarrow#1#2{\overunderstackrel{\rightarrowfill}{#1}{#2}}
\def\ddlarrow#1#2{\overunderstackrel{\longmapsto}{#1}{#2}}
\def\Dlarrow#1#2{\overunderstackrel{\Longrightarrow}{#1}{#2}}
\def\dmtarrow#1#2{\overunderstackrel{\longmapsto}{#1}{#2}}
\def\mtarrow#1{\overstackrel{\longmapsto}{#1}}
\newcommand{\mylarrow}[1]{\overstackrel{\rightarrowfill}{#1}}

\newcommand{\partarrow}[2]{\overunderstackrel{\rightarrowfill}{\ \ \mbox{\scriptstyle $#1$}\ \ }{\ \mbox{\scriptstyle $#2$}\ }\!\!\!>}
\newcommand{\arw}[1]{\overstackrel{\rightarrowfill}{#1}}
\newcommand{\newpartarrow}[2]{\partarrow{#1}{}}
\newcommand{\spartarrow}[1]{\overstackrel{\rightarrowfill}{\ \ \mbox{\scriptstyle $#1$}\ \ \ }\!\!\!\!\!>}
\newcommand{\trlts}[1]{\overstackrel{\rule[.8mm]{7mm}{.2mm}}{#1}\!\!\blacktriangleright}
\newcommand{\trred}{\overstackrel{\rule[.8mm]{7mm}{.2mm}}\!\!\vartriangleright}
%%% Arrows end

%%% Others
\newcommand{\squo}[1]{\lq {#1}\rq}
\newcommand{\quo}[1]{\lq\lq {#1}\rq\rq}
\def\finex{{\unskip\nobreak\hfil
\penalty50\hskip1em\null\nobreak\hfil$\diamond$
\parfillskip=0pt\finalhyphendemerits=0\endgraf}}
\newenvironment{myex}{\begin{example}\it}{\finex\end{example}}
\newcommand{\HSLtag}{\scalebox{1.25}{%
  \begin{tikzpicture}
  \draw (0.1,0.2) -- (0.2,0.115) -- (0.3,0.2) ;
  %
  \draw (0.1,0.2) -- (0.15,0.1) ;
  \draw (0.3,0.2) -- (0.25,0.1) ;
  %
  \draw (0.12,0.1) -- (0.2,0.05) -- (0.28,0.1) ;
  %
  \draw (0.2,0) circle (0.12) ; 
  \draw (0.16,0.04) circle (0.01) ;
  \draw (0.24,0.04) circle (0.01) ;
  \draw (0.15,-0.07) -- (0.25,-0.07) ;
  \end{tikzpicture}
}}
\newcommand{\HSLbox}{\rule{1ex}{1ex}}
\newcommand{\hsl}[1][]{{\color{red}\vbox{\medskip\noindent\hrulefill \\[5pt]
  \HSLtag \hspace{\stretch{1}}HIC SUNT
  LEONES \; {#1}\hspace{\stretch{1}} \HSLtag \\ \smallskip\noindent\hrulefill \\}}}
%%% Others end



%%% My colors
\definecolor{shadecolor}{rgb}{1,0.99,0.9}
\definecolor{bg}{rgb}{0.95,0.95,0.95}
\newcommand{\newgreen}{green!50!blue!100}
%%% My colors end














% !TEX root =  main.tex
%%% Macros for Klaimographies ;-)

\newcommandx{\asort}[1][1 = s, usedefault=@]{\mathbf{#1}}
\newcommand{\wildcard}{\star}
\newcommandx{\atuple}[1][1 = t]{\texttt{#1}}
\newcommandx{\aloc}[1][1=l]{\texttt{\color{blue}#1}}
\newcommand{\locset}{\mathcal{L}\!\mathit{oc}}
\newcommand{\tupleset}{\mathcal{T}}
\newcommand{\mkop}[1]{\textcolor{ForestGreen}{#1}}
\DeclareMathOperator{\at}{\mkop{\tiny @}}
\DeclareMathOperator{\outop}{\mkop{!}}
\DeclareMathOperator{\parop}{\mkop{\mid}}
\DeclareMathOperator{\chop}{\mkop{+}}
\DeclareMathOperator{\inop}{\mkop{?}}
\DeclareMathOperator{\toop}{\mkop{\to}}
\newcommand{\rec}{\mkop{rec}\xspace}
\newcommand{\nil}{\mkop{\textbf{0}}}
\newcommandx{\atupleat}[3][1={}, 2=\atuple, 3=\aloc, usedefault=@]{
  \ifempty{#1}{{#2} \at {#3}}{{#2} \at {#3}}
}
\newcommand{\clashes}{\sharp}
\newcommand{\matches}{\bowtie}
\newcommand{\arole}{\rho}
\newcommand{\ptp}[1]{{\mathsf{\MakeLowercase{#1}}}}
\newcommand{\amulti}[1]{{\mathsf{\titlecap{#1}}}}
\newcommand{\roleset}{\mathcal{R}}
\newcommand{\participants}{\mathcal{P}}
\newcommand{\multiroles}{\mathcal{M}}
\newcommand{\unknownstarop}{{\mkop{\ast}}}
\newcommand{\unknownop}{{\mkop{\odot}}}
\newcommandx{\unknownstar}[1][1=P]{{\amulti {#1}}^{\unknownstarop}}
\newcommandx{\unknown}[1][1= P]{{\amulti {#1}}^{\unknownop}}
\newcommand{\aK}[1][K]{\texttt{\textcolor{blue}#1}}
\newcommand{\aL}[1][L]{\texttt{\textcolor{orange}#1}}
\newcommandx{\proj}[2][1=\aK,2=\arole,usedefault=@]{
  \ifempty{#1}{\_}{#1} \downharpoonleft_{\ifempty{#2}{\_}{#2}}
}
\newcommandx{\aint}[3][1=\arole, 2=\arole', 3=\aK, usedefault=@]{
  {#1} \toop {#2} \, \mkop{:} \, {#3}
}
\newcommandx{\apref}[5][1={},2={},3=\atuple,4=\aloc,5={},usedefault=@]{
  \ifempty{#1}{
    \ifempty{#2}{\pi}{
      \ifempty{#5}{\ain[{#2}][{#3}][{#4}]}{\ard[{#2}][{#3}][{#4}]}
    }
  }{
    \ifempty{#2}{
      \ifempty{#5}{\aout[{#1}][({#3})][{#4}]}{\aout[{#1}][{#3}][{#4}]}
    }{
      {#1} \toop {#2} \, \mkop{:} \, \ifempty{#5}{({#3})\at{#4}}{{#3}\at{#4}}
    }
  }
}
\newcommandx{\asum}[4][1=i,2=I,3=\apref,4=\aK,usedefault=@]{
  \displaystyle{\sum_{#1 \ifempty{#2}{}{\in #2}}}{#3_{#1} \ifempty{#4}{}{. #4_{#1}}}
}
\newcommandx{\aout}[3][1=\arole,2=\atuple,3=\aloc,usedefault=@]{
  \ifempty{#1}{#2 \outop (#3)}{{#1} \outop {#2} \at {#3}}
}
\newcommandx{\ain}[3][1=\arole,2=\atuple,3=\aloc,usedefault=@]{
  \ifempty{#1}{(#2) \inop #3}{{#1} \inop {(#2)} \at {#3}}
}
\newcommandx{\ard}[3][1=\arole,2=\atuple,3=\aloc,,usedefault=@]{
  \ifempty{#1}{#2 \inop #3}{{#1} \inop {#2} \at {#3}}
}
\newcommandx{\arec}[2][1=X,2=\aK,usedefault=@]{
  \rec\ #1 \mkop{.} #2
}
\newcommandx{\aLsum}[4][1=i,2=I,3=\arole,4=\aL,usedefault=@]{
    \displaystyle{\sum_{#1 \ifempty{#2}{}{\in #2}}}{#3_{#1} : {#4_{#1}}}
}

\newcommand{\selectors}[1]{\it{sel}(#1)}
\newcommand{\roles}[1]{\it{roles}(#1)}
\newcommand{\eqR}{\sim}

\newcommand{\irule}[2]{\frac{\textstyle\rule[-1.3ex]{0cm}{3ex}#1}{\textstyle\rule[-.5ex]{0cm}{3ex}#2}}


\def \mathaxiom #1#2{
  \begin{array}{l}%
    \ifempty{#2}{}{\hspace{0em}\mbox{\footnotesize$\mathsf{[#2]}$}\\}
    {#1}
  \end{array}
}

\def \mathrule #1#2#3{
  \begin{array}{l}%
    \ifempty{#3}{}{\hspace{0em}\mbox{\footnotesize$\mathsf{[#3]}$}\\}
    \irule{#1}{#2}
  \end{array}
}

\newcommand{\envmv}{\textcolor{cyan}{\Delta}}
\newcommand{\envtuple}{\textcolor{cyan}{\Gamma}}

\newcommandx{\red}[1][1={}]{\xlongrightarrow{#1}}

\newcommand{\roleof}[1][\apref]{\mathsf{role}\ifempty{#1}{}{(#1)}}

%%%%%%%%%%%%%%%%%%%%%%%%%%%%%%%%%%%%%%%%%%%%%%%%%%%%%%%%%%%%%%%%%%%%%%%%%%%%%
%%%                        START POMSETS MACROS                           %%%
%%%%%%%%%%%%%%%%%%%%%%%%%%%%%%%%%%%%%%%%%%%%%%%%%%%%%%%%%%%%%%%%%%%%%%%%%%%%%
\newcommand{\apom}{r}
\newcommand{\emptypom}{\epsilon}
\newcommand{\alf}{\lambda}
\newcommand{\projpom}[2]{{#1}\!\!\downharpoonright_{#2}}

%%% chosem macros to add to ggmacros
\newcommand{\eset}{\mathcal{E}}
\newcommand{\aR}[1][R]{{\colorR{#1}}}
\newcommand{\efst}[1]{\pi_1\ifempty{#1}{}{({#1})}}
\newcommand{\aConf}{s}
\newcommand{\alfof}[1]{\alf_{#1}}
\newcommand{\minev}[1]{\mathtt{min}_{[#1]}}
\newcommand{\esetof}[1]{\eset_{#1}}
\newcommand{\leqof}[1]{\leq_{#1}}
\newcommandx{\detM}[1][1=\aCM,usedefault=@]{\Delta({#1})}

\tikzset{
  pomset/.style={
    scale = .7,
    transform shape,
    smooth
  }
}
%
\newcommandx{\pomsetrep}[4][2=\alf, 3={}, 4={}, usedefault=@]{
  \left[
    \begin{array}[c]{c}
      \begin{tikzpicture}[every node/.style={pomset},#4]
        {#1;}
      \end{tikzpicture}
    \end{array}\right]_{#2}^{#3}
}
\DeclareMathOperator{\pomsetcup}{\sqcup}
\newcommand{\pomsetsingle}[1][\apref]{\pomsetrep{\node {$#1$}}[]}

\def\colorMsg{\color{BrickRed}}    
\def\colorR{\color{OliveGreen}}
\newcommand{\msg}[1][m]{\mathsf{\colorMsg{#1}}}
\newcommand{\lset}{\mathcal{L}}
\def\colorE{\color{orange}}
\newcommand{\al}[1][l]{{\colorE{#1}}}
\def\colorE{\color{orange}}
\renewcommand{\ae}[1][e]{{\colorE{#1}}}
\newcommandx{\rseq}[2][1=\aG,2={\aG'},usedefault=@]{\gfun{seq}({#1},{#2})}
\def\colorFun{\color{black}}
\newcommand{\gfun}[1]{\ensuremath{\mathsf{\colorFun #1}}}
\newcommandx{\bp}[2][1=h,2=\apref,usedefault=@]{\gfun{bp}(\ifempty{#1}{\_}{#1}, \ifempty{#2}{\_}{#2})}
\newcommandx{\unique}[1][1=\apref,usedefault=@]{\gfun{unique}(\ifempty{#1}{\_}{#1})}
\newcommand{\ksem}[1]{\llbracket \ifempty{#1}{\_}{#1} \rrbracket}

%%%%%%%%%%%%%%%%%%%%%%%%%%%%%%%%%%%%%%%%%%%%%%%%%%%%%%%%%%%%%%%%%%%%%%%%%%%%% 
%%%                        END POMSETS MACROS                             %%%
%%%%%%%%%%%%%%%%%%%%%%%%%%%%%%%%%%%%%%%%%%%%%%%%%%%%%%%%%%%%%%%%%%%%%%%%%%%%%


%%% Local Variables:
%%% mode: latex
%%% TeX-master: "main"
%%% End:

\renewcommand{\al}{\ae}
\renewcommand{\lset}{\events}
\newcounter{myfact}[section]
\newenvironment{fact}[1][]{\refstepcounter{myfact}\par\medskip
  \noindent \textbf{Fact~\themyfact. #1} \it}{\smallskip}
% \pagestyle{plain}

\newcommand{\keywords}[1]{\par\addvspace\baselineskip
  \noindent\keywordname\enspace\ignorespaces#1
}

\newcommand{\proj}[2]{(#1)\downarrow_{#2}}
\newcommand{\mproj}[2]{(#1)\lightning_{#2}}

% Adrian's shortand macros

%% Generic shorthand
\newcommand{\naive}{na\"{\i}ve\xspace}
\newcommand{\Naive}{Na\"{\i}ve\xspace}
% \newcommand{\ie}{\textsl{i.e.,}\xspace}
% \newcommand{\eg}{\textsl{e.g.,}\xspace}
\newcommand{\Eg}{\textsl{E.g.,}\xspace}
% \newcommand{\cf}{\textsl{cf.}\xspace}
\newcommand{\etc}{\textsl{etc.}\xspace}
\newcommand{\wrt}{with respect to\xspace}
% {\textsl{wrt.}\xspace}
\newcommand{\etal}{\textsl{et~al.}\xspace}
\newcommand{\envs}{\textsl{Env.}\xspace}
%\newcommand{\resp}{respectively\xspace}{\textsl{resp.,}\xspace}
\newcommand{\pg}{\textsl{p.g.}\xspace}
\newcommand{\sth}{\textsl{s.t.}\xspace}



% Comments macros

\newcommand{\af}[1]{\textbf{\textcolor{cyan}{{\scriptsize A: #1} }}}   % Adrian's comments
\newcommand{\cam}[1]{\textbf{\textcolor{blue}{{\scriptsize C: #1} }}}   % Claudio's comments
\newcommand{\et}[1]{\textbf{\textcolor{red}{{\scriptsize E: #1} }}}   % Emilio's comments





\begin{document}





\mainmatter  % start of an individual contribution

% first the title is needed
\title{
  Data-driven choreographies in Klaim
}

% a short form should be given in case it is too long for the running head

% the name(s) of the author(s) follow(s) next
%
% NB: Chinese authors should write their first names(s) in front of
% their surnames. This ensures that the names appear correctly in
% the running heads and the author index.
%

\author{
  Roberto Bruni \inst{1} \and
  Fabio Gadducci \inst{1} \and
  Hernan Melgratti \inst{2} \and
  Andrea Corradini \inst{1} \and
  Emilio Tuosto\inst{3}
}
\institute{
  Universit\`a di Pisa, Italy
  \and
  Departamento de Computaci\'on, FCEyN, Universidad de Buenos Aires - Conicet, Argentina
  \and
  University of Leicester, UK
}

% \setcounter{tocdepth}{2}
% \listoffixmes
% \newpage

\maketitle

\begin{abstract}
  Klaim is one of the many contributions of Rocco.
  %
  It mixes together many flavours and borders many research areas.
  %
  One can appreciate Klaim as a basic calculus alternative to other
  name passing calculi such as the pi, join, or ambient.
  %
  Or one can see Klaim as a coordination model bringing the generative
  mechanisms of Linda in the distributed setting.
  %
  Or else one can envisage Klaim as a programming paradigm.

  We propose Klaim as a suitable base for a novel choreographic
  framework.
  %
  More precisely we advocate Klaim as a suitable language onto which
  to project \emph{data-driven} global specifications based on
  distributed tuple spaces.
  %
  These specifications, akin behavioural types, describe the coordination
  from a global point of view.
  %
  Differently from behavioural types though, our specifications
  express the data flow across distributed tuple spaces rather than
  detailing the communication pattern of processes.
  %
  We devise a typeing system to validate Klaim programs against projections
  of our global specifications.
  %
  An interesting feature of our typeing approach is that well-typed
  systems have arbitrary number of participants.
  %
  In standard approaches based on behavioural types, this is often
  achieved at the cost of tremendous technical complications.
\end{abstract}

\section{Introduction}
\label{sec:intro}
% !TEX root =  main.tex

Communication-centered programming is playing a tremendous role in the production of nowadays software. Programming peers that need to exchange information is an error-prone activity and the behaviour of even small systems is subject to a combinatorial blow-up as the number of peers increases.
Therefore well-structured principles and rigorous foundations are needed to develop well-engineered, trustworthy software. 
One possibility is to exploit some sort of behavioural types~\cite{DBLP:journals/csur/HuttelLVCCDMPRT16,dd09} to manage abstract descriptions of peers and formally study their properties such as communication safety, absence of deadlocks, progress or session fidelity: given the types of the peers the emerging behaviour of their composition is analysed.
In the seminal paper~\cite{DBLP:conf/popl/HondaYC08}, recently nominated the \emph{most influential POPL paper (Award 2018)}, the authors push forward an abstract notion of global type of interaction that represents a sort of contract between the communicating peers. This is paired with the notion of local type that gives an abstract description of the behaviour of each peer, as taken in isolation.
Interestingly, local types can be obtained for free by projection from global types, while the properties of interest can be studied and guaranteed just at the level of global types, without the need of studying the composition of local types. The conformance of peers implementation w.r.t. the global type can be studied instead at the level of local types, allowing a more efficient form of type checking. Roughly this means that properties are stated globally but checked locally. Global types have been inspired by choreography languages in service oriented computing, where complex interactions are modelled from the point of view of the global sequence of events that must take place in order to successfully complete the computation.

In the literature, global/local types have been studied mostly in the
context of point-to-point channel-based interactions. This means that
the main action in a choreography is the sending of a message from one
peer to another on a specific channel (of a given type). In this paper we
explore a different setting, where interaction over tuple-spaces
replaces message passing, in the style of Linda-like
languages~\cite{DBLP:journals/toplas/Gelernter85}.  Instead of
primitives for sending and receiving messages, here there are
primitives for inserting a tuple on a tuple space, for reading
(without consuming) a tuple from a tuple space or for retrieving a
tuple from a tuple space. We call these interactions data-driven, as
decisions will be taken on the basis of the type of the tuples that
are manipulated. We coined the term Klaimographies in honour of the
process language Klaim~\cite{DBLP:journals/tse/NicolaFP98,klaim}, one
of the main contributions of Rocco De Nicola in the fields of process
algebras and distributed programming. Inspired by Klaim,
Klaimographies exploit the notion of distributed tuple-space
localities to separate the access to data on the basis of the
interactions that are carried out. Localities can be communicated in
tuples, allowing name mobility.

There are several major advances w.r.t. to the literature on global types.
First, Klaimographies can have an arbitrary number of participants, while in global types the number of participants is usually fixed a priori.
Second, interactions are not point-to-point because each tuple can be read many times, while messages have exactly one producer and one consumer, even when asynchronous communication is considered.
Third, all interactions involve a tuple space locality instead of a channel name.
Fourth, Klaimographies are data-driven.

The main contribution of this paper is to set up the formal setting of Klaimographies and prepare the ground for several interesting research directions: we fix the syntax of global and local type description and define the projection from global type to local types as typical of choreographic frameworks.
Global types are equipped with a partial order semantics of events and local types with an ordinary operational semantics in the SOS style. Then the conditions under which the behaviour of projected local types are faithful to the semantics of global types are spelled out. 

Shifting the focus from control to data in choreographic framework has
several implications.
%
Firstly, the emphasis is no longer on properties related to
computational actors.
%
For instance, Klaimographies admit computations where some processes
may not terminate and is left waiting for some data.
%
In standard choreographic framework those would be undesired behaviour
to rule out with suitable typing disciplines.
%
Nonetheless, we claim that in some application domain computations with
deadlocked processes have to be considered non-erroneous.
%
For instance, in reactive systems based on event-notification
frameworks some \quo{listener} components must be kept waiting for
events to occur.
%
Our work paves the way to the formal study of properties of data, like consumption, persistence and availability, in a choreographic setting.

Another main innovation of Klaimographies is that they allow us to easily
represent protocols where an arbitrary number of components can
play a role in the choreography.
%
We give an example of such protocol in \cref{sec:examples}.
%
Remarkably, those protocols can be specified in some existing
choreographic frameworks~\cite{ydbh10,chjny19}, but in a less abstract way
that requires the explicit quantification on components.

\paragraph{Structure of the paper}
\todo{eM: revise}
After some preliminaries in Section~\ref{sec:tuples}, we define Klaimographies as global types in Section~\ref{sec:gt} and give some examples in Section~\ref{sec:examples}.
In Section~\ref{sec:globsem} we define the semantics of global types and give the adequacy conditions for projecting global types to local types.
In Section~\ref{sec:locsem} we define the operational semantics of local types and state its correspondence with the global semantics.
Some concluding remarks together with the discussion of related and future work are in Section~\ref{sec:conc}.

%%% Local Variables:
%%% mode: latex
%%% TeX-master: "main"
%%% End:




\section{Background}
\label{sec:background}
% !TEX root =  main.tex


%%% Local Variables:
%%% mode: latex
%%% TeX-master: "main"
%%% End:


\section{Conclusions}
\label{sec:disc}\label{sec:conc}
% !TEX root =  main.tex


This paper, a modest attempt to thank Rocco for his work and
friendship, addresses the following question:
%
\begin{quote}
  What notion of behavioural types corresponds to Linda-based
  coordination mechanisms?
\end{quote}
%
To answer such question we advocate Klaim-based global and local
types, dubbed klaimographies.
%
Klaim has been designed to program distributed systems consisting of
processes interacting via multiple distributed tuple spaces.

For simplicity, we have neglected code mobility, a distinctive feature
of Klaim.
%
Accommodating the mobility mechanism of Klaim would require to control
the multiplicity of running instances and to generalise the
well-formedness conditions to dynamically spawned processes.
%
A further challenge, would be to include mobility of
processes-as-values featured by Klaim, which shares many
similarities with session delegation.
%
However, this can be associated to control-driven problems.
%
These challenges are scope for future work.

We have also not considered parallel types.
%
A simple way to compose klaimographies in parallel would be to follow standard
approaches restricting roles on single threads and disjoint tuple
spaces.
%
We consider this not very interesting, and plan to explore more expressive
settings for parallel types such as the one in~\cite{gt16,gt17}.
%
In particular, we conjecture that to add parallel composition
$\aK \mid \aK'$ of klaimographies it is enough to require that
$\neg(\atuple \matches \atuple')$ for all
$(\atuple, \aloc) \in \aK, (\atuple', \aloc) \in \aK'$.
%
This condition is the counterpart of the \emph{well-forkedness}
condition of~\cite{gt16,gt17}, that requires that different threads of
a choregraphy have disjoing input actions.

Klaim has been extended with several features designed on theoretical
foundations and implemented in a suite of
prototypes~\cite{klaim}.
%
On the one hand, klaimographies share similarities with standard
behavioural types centred on point-to-point channel based
communications, on the other hand they also have some peculiarities,
some of which we highlighted here.

The closest work to our is~\cite{chjny19}, which develops the initial
proposal on parameterised choreographies in~\cite{ydbh10,dybh12}.
%
Notably,~\cite{chjny19} is the first work to support indexed roles and
to statically infer the participants inhabiting them.
%
The main difference with the approach in~\cite{chjny19} is that
klaimographies do not focus on processes, but rather on data.
%
We envisage behavioural types as specifications of how to guarantee
general properties of tuple spaces.
%
For instance, take the marketplace example (cf. \cref{ex:market}),
one would like to check properties such as
\begin{quote}
  for each tuple type
  $\atuple = {{i : \asort[str]} \cdot {p : \asort[int]} \cdot {\nu l :
      \asort[loc]}}$ consumed from locality $\aloc[m]$ either a tuple type
  $\asort[sold]$ is eventually generated at locality $l$ or $\atuple$
  is eventually generated at $\aloc[m]$.
\end{quote}
%
Such property does not concern typical properties
controlled by behavioural types (e.g., progress of processes, message
orphanage, or unspecified reception).

As scope for future work, we aim to characterise the (classes of)
properties of interest that klaimographies enforce.
%
We conjecture that the well-formedness conditions defined here
are strong enough to guarantee the property above.
%
Another interesting line of research is to identify typing principles
for Klaim processes.
%
We believe that klaimographies can enable the possibility that a same
process enacts different roles.
%
For instance, considering again the marketplace example, a process
can act both as seller and as buyer.

We have adopted a few simplifying assumptions.
%
Other variants seem rather interesting.
%
For instance, guards of sums could be autonomous inputs and not just
consuming interactions, or even read-only access prefixes.
%
Relaxing the constraint that read-only tuples cannot generate, would
lead to a sort of multi-cast mechanism of fresh localities.
%
We plan to study those variants in future work.


%%% Local Variables:
%%% mode: latex
%%% TeX-master: "main"
%%% End:


\bibliographystyle{abbrv}
\bibliography{biblio}

\end{document}

\documentclass[runningheads,a4paper]{llncs}

% !TEX root =  main.tex
\usepackage{hyperref}
\usepackage{amsmath}
\usepackage{amssymb}
\usepackage{mathtools}
\usepackage{stmaryrd}
\usepackage{graphicx}
\usepackage{subfigure}
\usepackage{multicol}
\usepackage[usenames,dvipsnames,svgnames,table]{xcolor} % must be loaded before tikz %% use monochrome option to remove colors
\usepackage{tikz}
\usepackage{xstring}
\usepackage{graphics,framed}
\usepackage[capitalise]{cleveref}
\crefformat{enumi}{condition~#2#1#3}
\crefname{fact}{Fact}{Facts}
\Crefname{fact}{Fact}{Facts}
\crefformat{fact}{Fact~#2#1#3}
\usepackage{xargs}
\usepackage{titlecaps}
\usepackage{listings}
\usepackage{etoolbox}
\usepackage{bussproofs}
\usepackage{etoolbox}
\usepackage{extarrows}
%\usepackage{minted}
%\AtBeginEnvironment{minted}{\fontsize{7.1}{7.1}\selectfont}
\usepackage{wrapfig}
\usepackage{lineno}

%\usepackage{xypic}
\usepackage{xspace}

\usepackage[inline]{enumitem}

% managing draft and final version (uncomment finaltrue to get final)
\newif\iffinal
% \finaltrue

\usepackage[textsize=tiny%
\iffinal
,disable%
\fi
]{todonotes}

\iffinal
\else
% for todo notes
\setlength{\marginparwidth}{2.2cm}
\fi



%%% Local Variables:
%%% mode: latex
%%% TeX-master: "main"
%%% End:

% !TEX root =  main.tex
%%% Macros usuful almost everywhere


% such that

%%% My flags
\newif\ifemi
%%% My flags end

\newcommand{\nota}[2][TODO]{::\textcolor{red}{#1}:: \color{cyan} #2}



\DeclareGraphicsExtensions{%
    .png,.PNG,%
    .pdf,.PDF,%
    .jpg,.mps,.jpeg,.jbig2,.jb2,.JPG,.JPEG,.JBIG2,.JB2}


\usepackage[draft]{fixme}
\fxusetheme{color}
% \FXRegisterAuthor{eM}{aeM}{\color{orange} {\underline{eM}}}

\usepackage{bm}
%\usepackage[disable]{todonotes}

\usepackage[normalem]{ulem} % underline command breaks over line ends

\usepackage{xifthen}        % for conditional commands
\newcommand{\ifempty}[3]{%
  \ifthenelse{\isempty{#1}}{#2}{#3}%
}


\newcommand{\mkfun}[4][\colorFun]{
  \newcommand{#2}[1][#4]{
    {#1\textsf{#3}}
    \ifempty{##1}{}{
      \big({##1}\big)}
  }
}

\newcommand{\mkuop}[4][\colorFun]{
  \newcommand{#2}[1][#4]{
    {#1\textsf{#3}}
    \ifempty{##1}{}{
      \, {##1}}
  }
}

%%% Meta comments
\newcommand{\hidden}[1]{}

\newcommand{\cf}[2]{
  \fontsize{#1}{#1}{\selectfont{#2}}
}
\ifemi
\usepackage{showlabels}
\renewcommand{\showlabelsetlabel}[1]{{\textcolor{ForestGreen}{\showlabelfont{\tiny #1}}}}
\newcommand{\emi}[2]{
  \marginpar{\fcolorbox{red}{shadecolor}{\cf{#1}{{#2}}}}
}
\newcommand{\emic}[2]{\par
  \fcolorbox{red}{shadecolor}{\parbox{\linewidth}{ 
      \color{gray}
      \begin{description}
      \item[{\color{blue} #2}]{\sf #1}
      \end{description}}}
}
\else
\newcommand{\emi}[2]{}
\newcommand{\emic}[2]{}{}
\fi

\newcommand{\marginnote}[1]{
  {\makebox[0pt]{\color{orange}}}
  \marginpar{\parbox{2cm}{\flushleft \tiny \color{orange}{#1}}}
}
\newcommand{\side}[2][6]{
  \marginpar{\cf{#1}{\hl{#2}}}
}
%%% Meta comments end



%%% Maths & logic
\newcommand{\mkset}[1]{\big\{ {#1} \big\}} %%% formerly called \ASET
\newcommand{\partsof}[1]{2^{#1}} %%% formerly called \ASET
\newcommand{\setof}[1]{\mkset{#1}}
\newcommand{\setdef}[2]{\mkset{#1 \sst #2}}
\newcommand{\card}[1]{\|{#1}\|}
\newcommand{\sst}{\;\big|\;}
\newcommand{\qst}{\;\colon\;} %such that
%\newcommand{\dom}[1]{\operatorname{dom} {#1}}
\newcommand{\dom}[1]{\textit{dom}(#1)}

%\newcommand{\cod}[1]{\operatorname{cod} {#1}}
\newcommand{\cod}[1]{\textit{cod}(#1)}

\newcommand\myenddef{~\hfill \ensuremath{\diamond}}
\newcommand{\conf}[1]{\ensuremath{\langle {#1} \rangle}}
\newcommand{\tuple}[1]{\conf{#1}}
\newcommand{\acsconf}[2]{\conf{{#1} \ ; \ {#2}}}
%\renewcommand{\vec}[1]{\overset{\to}{#1}}
\newcommand{\ith}[2]{\vec{#1}[{#2}]}
\newcommand{\bnfdef}{\ ::=\ }
\newcommand{\bnfmid}{\;\ \big|\ \;}
\newcommand{\sep}{\;\bnfmid\;}
\newcommand{\emptyword}{\varepsilon}
\newcommand{\aword}{\varphi}
\newcommand{\langof}[1]{\mathcal{L}(#1)}
\newcommand{\sem}[2][]{\mbox{\ensuremath{\llbracket{#2}\rrbracket_{#1}}}}
\newcommand{\qmmdef}{\quad \mmdef \quad}
\newcommand{\qqmmdef}{\qquad \mmdef \qquad}
\newcommand{\qqand}{\qquad\text{and}\qquad}
\newcommand{\qand}{\quad\text{and}\quad}
\newcommand{\verum}{\mathtt{tt}}
\newcommand{\falsus}{\mathtt{ff}}
\newcommand{\nat}{\mathbb{N}}
\newcommand{\upd}[3]{{#1}[{#2} \mapsto {#3}]}

%%% Maths & logic end



%%% Typographic style
\newcommand{\defrule}[1]{%
  % \hypertarget{rule:{#1}}
  {%
    \text{\scriptsize[\textbf{#1}]}%
  }%
}
\newcommand{\ie}{\text{i.e.}}
\newcommand{\cfw}{\text{cf.}}
\newcommand{\eg}{\text{e.g.}}
\newcommand{\aka}{\text{a.k.a.}}
\newcommand{\mypar}[1]{\noindent\paragraph{\textbf{#1}}\ }
\newcommand{\secref}[1]{\S~\ref{#1}}
\newcommand{\figref}[1]{Fig.~\ref{#1}}
\newcommand{\exref}[1]{Ex.~\ref{#1}}
\newcommand{\defref}[1]{Definition~\ref{#1}}
\newcommand{\exeref}[1]{Exercise~\ref{#1}}
\newcommand{\rmkref}[1]{Remark~\ref{#1}}
\newtheorem{rmk}{Remark}{\bfseries}{\rmfamily}
\newtheorem{exe}{Exercise}{\bfseries}{\rmfamily}
\newcommand{\sidebyside}[2]{
  \begin{tabular}{ll}
    \begin{minipage}{.5\linewidth} {#1}  \end{minipage}
    &
    \begin{minipage}{.5\linewidth} {#2}  \end{minipage}
  \end{tabular}
}
\newcommand{\eqdef}{\ \triangleq\ }
\newcommand{\mmdef}{\eqdef}
%%% Typographic style end



%%% Arrows
\def\ured{\rightarrow}
\def\tred#1{\overstackrel{\rightarrowfill}{#1}}
\def\iured#1{\rightarrow_{\descr{\scriptstyle #1}}}
\def\itred#1#2{\mathrel{{\overstackrel{\rightarrowfill}{#2}}_{\descr
      {\scriptstyle #1}}}}
\def\sytr#1{\overstackrel{\longmapsto}{#1}}
\def\tr#1{\overstackrel{\rightarrowfill}{#1}}
\def\actr#1#2{\overstackrel{\rightarrowfill}{\act {#1}{#2}}}
\def\iactr#1#2#3{\overstackrel{\rightarrowfill}{\act {#2}{#3}}_{\descr
    {\scriptstyle #1}}}
\def\dlarrow#1#2{\overunderstackrel{\rightarrowfill}{#1}{#2}}
\def\ddlarrow#1#2{\overunderstackrel{\longmapsto}{#1}{#2}}
\def\Dlarrow#1#2{\overunderstackrel{\Longrightarrow}{#1}{#2}}
\def\dmtarrow#1#2{\overunderstackrel{\longmapsto}{#1}{#2}}
\def\mtarrow#1{\overstackrel{\longmapsto}{#1}}
\newcommand{\mylarrow}[1]{\overstackrel{\rightarrowfill}{#1}}

\newcommand{\partarrow}[2]{\overunderstackrel{\rightarrowfill}{\ \ \mbox{\scriptstyle $#1$}\ \ }{\ \mbox{\scriptstyle $#2$}\ }\!\!\!>}
\newcommand{\arw}[1]{\overstackrel{\rightarrowfill}{#1}}
\newcommand{\newpartarrow}[2]{\partarrow{#1}{}}
\newcommand{\spartarrow}[1]{\overstackrel{\rightarrowfill}{\ \ \mbox{\scriptstyle $#1$}\ \ \ }\!\!\!\!\!>}
\newcommand{\trlts}[1]{\overstackrel{\rule[.8mm]{7mm}{.2mm}}{#1}\!\!\blacktriangleright}
\newcommand{\trred}{\overstackrel{\rule[.8mm]{7mm}{.2mm}}\!\!\vartriangleright}
%%% Arrows end

%%% Others
\newcommand{\squo}[1]{\lq {#1}\rq}
\newcommand{\quo}[1]{\lq\lq {#1}\rq\rq}
\def\finex{{\unskip\nobreak\hfil
\penalty50\hskip1em\null\nobreak\hfil$\diamond$
\parfillskip=0pt\finalhyphendemerits=0\endgraf}}
\newenvironment{myex}{\begin{example}\it}{\finex\end{example}}
\newcommand{\HSLtag}{\scalebox{1.25}{%
  \begin{tikzpicture}
  \draw (0.1,0.2) -- (0.2,0.115) -- (0.3,0.2) ;
  %
  \draw (0.1,0.2) -- (0.15,0.1) ;
  \draw (0.3,0.2) -- (0.25,0.1) ;
  %
  \draw (0.12,0.1) -- (0.2,0.05) -- (0.28,0.1) ;
  %
  \draw (0.2,0) circle (0.12) ; 
  \draw (0.16,0.04) circle (0.01) ;
  \draw (0.24,0.04) circle (0.01) ;
  \draw (0.15,-0.07) -- (0.25,-0.07) ;
  \end{tikzpicture}
}}
\newcommand{\HSLbox}{\rule{1ex}{1ex}}
\newcommand{\hsl}[1][]{{\color{red}\vbox{\medskip\noindent\hrulefill \\[5pt]
  \HSLtag \hspace{\stretch{1}}HIC SUNT
  LEONES \; {#1}\hspace{\stretch{1}} \HSLtag \\ \smallskip\noindent\hrulefill \\}}}
%%% Others end



%%% My colors
\definecolor{shadecolor}{rgb}{1,0.99,0.9}
\definecolor{bg}{rgb}{0.95,0.95,0.95}
\newcommand{\newgreen}{green!50!blue!100}
%%% My colors end














% !TEX root =  main.tex
%%% Macros for Klaimographies ;-)

\newcommandx{\asort}[1][1 = s, usedefault=@]{\mathbf{#1}}
\newcommand{\wildcard}{\star}
\newcommandx{\atuple}[1][1 = t]{\texttt{#1}}
\newcommandx{\aloc}[1][1=l]{\texttt{\color{blue}#1}}
\newcommand{\locset}{\mathcal{L}\!\mathit{oc}}
\newcommand{\tupleset}{\mathcal{T}}
\newcommand{\mkop}[1]{\textcolor{ForestGreen}{#1}}
\DeclareMathOperator{\at}{\mkop{\tiny @}}
\DeclareMathOperator{\outop}{\mkop{!}}
\DeclareMathOperator{\parop}{\mkop{\mid}}
\DeclareMathOperator{\chop}{\mkop{+}}
\DeclareMathOperator{\inop}{\mkop{?}}
\DeclareMathOperator{\toop}{\mkop{\to}}
\newcommand{\rec}{\mkop{rec}\xspace}
\newcommand{\nil}{\mkop{\textbf{0}}}
\newcommandx{\atupleat}[3][1={}, 2=\atuple, 3=\aloc, usedefault=@]{
  \ifempty{#1}{{#2} \at {#3}}{{#2} \at {#3}}
}
\newcommand{\clashes}{\sharp}
\newcommand{\matches}{\bowtie}
\newcommand{\arole}{\rho}
\newcommand{\ptp}[1]{{\mathsf{\MakeLowercase{#1}}}}
\newcommand{\amulti}[1]{{\mathsf{\titlecap{#1}}}}
\newcommand{\roleset}{\mathcal{R}}
\newcommand{\participants}{\mathcal{P}}
\newcommand{\multiroles}{\mathcal{M}}
\newcommand{\unknownstarop}{{\mkop{\ast}}}
\newcommand{\unknownop}{{\mkop{\odot}}}
\newcommandx{\unknownstar}[1][1=P]{{\amulti {#1}}^{\unknownstarop}}
\newcommandx{\unknown}[1][1= P]{{\amulti {#1}}^{\unknownop}}
\newcommand{\aK}[1][K]{\texttt{\textcolor{blue}#1}}
\newcommand{\aL}[1][L]{\texttt{\textcolor{orange}#1}}
\newcommandx{\proj}[2][1=\aK,2=\arole,usedefault=@]{
  \ifempty{#1}{\_}{#1} \downharpoonleft_{\ifempty{#2}{\_}{#2}}
}
\newcommandx{\aint}[3][1=\arole, 2=\arole', 3=\aK, usedefault=@]{
  {#1} \toop {#2} \, \mkop{:} \, {#3}
}
\newcommandx{\apref}[5][1={},2={},3=\atuple,4=\aloc,5={},usedefault=@]{
  \ifempty{#1}{
    \ifempty{#2}{\pi}{
      \ifempty{#5}{\ain[{#2}][{#3}][{#4}]}{\ard[{#2}][{#3}][{#4}]}
    }
  }{
    \ifempty{#2}{
      \ifempty{#5}{\aout[{#1}][({#3})][{#4}]}{\aout[{#1}][{#3}][{#4}]}
    }{
      {#1} \toop {#2} \, \mkop{:} \, \ifempty{#5}{({#3})\at{#4}}{{#3}\at{#4}}
    }
  }
}
\newcommandx{\asum}[4][1=i,2=I,3=\apref,4=\aK,usedefault=@]{
  \displaystyle{\sum_{#1 \ifempty{#2}{}{\in #2}}}{#3_{#1} \ifempty{#4}{}{. #4_{#1}}}
}
\newcommandx{\aout}[3][1=\arole,2=\atuple,3=\aloc,usedefault=@]{
  \ifempty{#1}{#2 \outop (#3)}{{#1} \outop {#2} \at {#3}}
}
\newcommandx{\ain}[3][1=\arole,2=\atuple,3=\aloc,usedefault=@]{
  \ifempty{#1}{(#2) \inop #3}{{#1} \inop {(#2)} \at {#3}}
}
\newcommandx{\ard}[3][1=\arole,2=\atuple,3=\aloc,,usedefault=@]{
  \ifempty{#1}{#2 \inop #3}{{#1} \inop {#2} \at {#3}}
}
\newcommandx{\arec}[2][1=X,2=\aK,usedefault=@]{
  \rec\ #1 \mkop{.} #2
}
\newcommandx{\aLsum}[4][1=i,2=I,3=\arole,4=\aL,usedefault=@]{
    \displaystyle{\sum_{#1 \ifempty{#2}{}{\in #2}}}{#3_{#1} : {#4_{#1}}}
}

\newcommand{\selectors}[1]{\it{sel}(#1)}
\newcommand{\roles}[1]{\it{roles}(#1)}
\newcommand{\eqR}{\sim}

\newcommand{\irule}[2]{\frac{\textstyle\rule[-1.3ex]{0cm}{3ex}#1}{\textstyle\rule[-.5ex]{0cm}{3ex}#2}}


\def \mathaxiom #1#2{
  \begin{array}{l}%
    \ifempty{#2}{}{\hspace{0em}\mbox{\footnotesize$\mathsf{[#2]}$}\\}
    {#1}
  \end{array}
}

\def \mathrule #1#2#3{
  \begin{array}{l}%
    \ifempty{#3}{}{\hspace{0em}\mbox{\footnotesize$\mathsf{[#3]}$}\\}
    \irule{#1}{#2}
  \end{array}
}

\newcommand{\envmv}{\textcolor{cyan}{\Delta}}
\newcommand{\envtuple}{\textcolor{cyan}{\Gamma}}

\newcommandx{\red}[1][1={}]{\xlongrightarrow{#1}}

\newcommand{\roleof}[1][\apref]{\mathsf{role}\ifempty{#1}{}{(#1)}}

%%%%%%%%%%%%%%%%%%%%%%%%%%%%%%%%%%%%%%%%%%%%%%%%%%%%%%%%%%%%%%%%%%%%%%%%%%%%%
%%%                        START POMSETS MACROS                           %%%
%%%%%%%%%%%%%%%%%%%%%%%%%%%%%%%%%%%%%%%%%%%%%%%%%%%%%%%%%%%%%%%%%%%%%%%%%%%%%
\newcommand{\apom}{r}
\newcommand{\emptypom}{\epsilon}
\newcommand{\alf}{\lambda}
\newcommand{\projpom}[2]{{#1}\!\!\downharpoonright_{#2}}

%%% chosem macros to add to ggmacros
\newcommand{\eset}{\mathcal{E}}
\newcommand{\aR}[1][R]{{\colorR{#1}}}
\newcommand{\efst}[1]{\pi_1\ifempty{#1}{}{({#1})}}
\newcommand{\aConf}{s}
\newcommand{\alfof}[1]{\alf_{#1}}
\newcommand{\minev}[1]{\mathtt{min}_{[#1]}}
\newcommand{\esetof}[1]{\eset_{#1}}
\newcommand{\leqof}[1]{\leq_{#1}}
\newcommandx{\detM}[1][1=\aCM,usedefault=@]{\Delta({#1})}

\tikzset{
  pomset/.style={
    scale = .7,
    transform shape,
    smooth
  }
}
%
\newcommandx{\pomsetrep}[4][2=\alf, 3={}, 4={}, usedefault=@]{
  \left[
    \begin{array}[c]{c}
      \begin{tikzpicture}[every node/.style={pomset},#4]
        {#1;}
      \end{tikzpicture}
    \end{array}\right]_{#2}^{#3}
}
\DeclareMathOperator{\pomsetcup}{\sqcup}
\newcommand{\pomsetsingle}[1][\apref]{\pomsetrep{\node {$#1$}}[]}

\def\colorMsg{\color{BrickRed}}    
\def\colorR{\color{OliveGreen}}
\newcommand{\msg}[1][m]{\mathsf{\colorMsg{#1}}}
\newcommand{\lset}{\mathcal{L}}
\def\colorE{\color{orange}}
\newcommand{\al}[1][l]{{\colorE{#1}}}
\def\colorE{\color{orange}}
\renewcommand{\ae}[1][e]{{\colorE{#1}}}
\newcommandx{\rseq}[2][1=\aG,2={\aG'},usedefault=@]{\gfun{seq}({#1},{#2})}
\def\colorFun{\color{black}}
\newcommand{\gfun}[1]{\ensuremath{\mathsf{\colorFun #1}}}
\newcommandx{\bp}[2][1=h,2=\apref,usedefault=@]{\gfun{bp}(\ifempty{#1}{\_}{#1}, \ifempty{#2}{\_}{#2})}
\newcommandx{\unique}[1][1=\apref,usedefault=@]{\gfun{unique}(\ifempty{#1}{\_}{#1})}
\newcommand{\ksem}[1]{\llbracket \ifempty{#1}{\_}{#1} \rrbracket}

%%%%%%%%%%%%%%%%%%%%%%%%%%%%%%%%%%%%%%%%%%%%%%%%%%%%%%%%%%%%%%%%%%%%%%%%%%%%% 
%%%                        END POMSETS MACROS                             %%%
%%%%%%%%%%%%%%%%%%%%%%%%%%%%%%%%%%%%%%%%%%%%%%%%%%%%%%%%%%%%%%%%%%%%%%%%%%%%%


%%% Local Variables:
%%% mode: latex
%%% TeX-master: "main"
%%% End:


\FXRegisterAuthor{eM}{aeM}{\color{orange} {\underline{eM}}}
%
% co-authors: please define your own macro for metacomments as the one
% above

\begin{document}

\mainmatter  % start of an individual contribution

% first the title is needed
\title{
  \iffinal
  Data-driven choreographies \`a la Klaim
  \else
  \fcolorbox{black!15}{yellow!20}{Data-driven choreographies \`a la Klaim}
  \fi
}

% a short form should be given in case it is too long for the running head

% the name(s) of the author(s) follow(s) next
%
% NB: Chinese authors should write their first names(s) in front of
% their surnames. This ensures that the names appear correctly in
% the running heads and the author index.
%

\author{
  Roberto Bruni \inst{1} \and
  Andrea Corradini \inst{1} \and
  Fabio Gadducci \inst{1} \and
  Hern\'an Melgratti \inst{2} \and
  Ugo Montanari \inst{1} \and
  Emilio Tuosto\inst{3}
}
\institute{
  Universit\`a di Pisa, Italy
  \and
  Departamento de Computaci\'on, FCEyN, Universidad de Buenos Aires - Conicet, Argentina
  \and
  University of Leicester, UK
}

\iffinal
\else
\authorrunning{\fcolorbox{black!15}{yellow!20}{Check list of fixmes}}
\fi

\maketitle

\begin{abstract}
  % Klaim is one of the many contributions of Rocco.
  % %
  % It mixes together many flavours and borders many research areas.
  % %
  % One can appreciate Klaim as a basic calculus alternative to other
  % name passing calculi such as the pi, join, or ambient.
  % %
  % Or one can see Klaim as a coordination model bringing the generative
  % mechanisms of Linda in the distributed setting.
  % %
  % Or else one can envisage Klaim as a programming paradigm.
  % 
  We propose Klaim as a suitable base for a novel choreographic
  framework.
  %
  More precisely we advocate Klaim as a suitable language onto which
  to project \emph{data-driven} global specifications based on
  distributed tuple spaces.
  %
  These specifications, akin behavioural types, describe the coordination
  from a global point of view.
  %
  Differently from behavioural types though, our specifications
  express the data flow across distributed tuple spaces rather than
  detailing the communication pattern of processes.
  %
  We devise a typing system to validate Klaim programs against projections
  of our global specifications.
  %
  An interesting feature of our typing approach is that well-typed
  systems have arbitrary number of participants.
  %
  In standard approaches based on behavioural types, this is often
  achieved at the cost of tremendous technical complications.
\end{abstract}



\section{Introduction}
\label{sec:intro}
% !TEX root =  main.tex

Communication-centered programming is playing a tremendous role in the production of nowadays software. Programming peers that need to exchange information is an error-prone activity and the behaviour of even small systems is subject to a combinatorial blow-up as the number of peers increases.
Therefore well-structured principles and rigorous foundations are needed to develop well-engineered, trustworthy software. 
One possibility is to exploit some sort of behavioural types~\cite{DBLP:journals/csur/HuttelLVCCDMPRT16,dd09} to manage abstract descriptions of peers and formally study their properties such as communication safety, absence of deadlocks, progress or session fidelity: given the types of the peers the emerging behaviour of their composition is analysed.
In the seminal paper~\cite{DBLP:conf/popl/HondaYC08}, recently nominated the \emph{most influential POPL paper (Award 2018)}, the authors push forward an abstract notion of global type of interaction that represents a sort of contract between the communicating peers. This is paired with the notion of local type that gives an abstract description of the behaviour of each peer, as taken in isolation.
Interestingly, local types can be obtained for free by projection from global types, while the properties of interest can be studied and guaranteed just at the level of global types, without the need of studying the composition of local types. The conformance of peers implementation w.r.t. the global type can be studied instead at the level of local types, allowing a more efficient form of type checking. Roughly this means that properties are stated globally but checked locally. Global types have been inspired by choreography languages in service oriented computing, where complex interactions are modelled from the point of view of the global sequence of events that must take place in order to successfully complete the computation.

In the literature, global/local types have been studied mostly in the
context of point-to-point channel-based interactions. This means that
the main action in a choreography is the sending of a message from one
peer to another on a specific channel (of a given type). In this paper we
explore a different setting, where interaction over tuple-spaces
replaces message passing, in the style of Linda-like
languages~\cite{DBLP:journals/toplas/Gelernter85}.  Instead of
primitives for sending and receiving messages, here there are
primitives for inserting a tuple on a tuple space, for reading
(without consuming) a tuple from a tuple space or for retrieving a
tuple from a tuple space. We call these interactions data-driven, as
decisions will be taken on the basis of the type of the tuples that
are manipulated. We coined the term Klaimographies in honour of the
process language Klaim~\cite{DBLP:journals/tse/NicolaFP98,klaim}, one
of the main contributions of Rocco De Nicola in the fields of process
algebras and distributed programming. Inspired by Klaim,
Klaimographies exploit the notion of distributed tuple-space
localities to separate the access to data on the basis of the
interactions that are carried out. Localities can be communicated in
tuples, allowing name mobility.

There are several major advances w.r.t. to the literature on global types.
First, Klaimographies can have an arbitrary number of participants, while in global types the number of participants is usually fixed a priori.
Second, interactions are not point-to-point because each tuple can be read many times, while messages have exactly one producer and one consumer, even when asynchronous communication is considered.
Third, all interactions involve a tuple space locality instead of a channel name.
Fourth, Klaimographies are data-driven.

The main contribution of this paper is to set up the formal setting of Klaimographies and prepare the ground for several interesting research directions: we fix the syntax of global and local type description and define the projection from global type to local types as typical of choreographic frameworks.
Global types are equipped with a partial order semantics of events and local types with an ordinary operational semantics in the SOS style. Then the conditions under which the behaviour of projected local types are faithful to the semantics of global types are spelled out. 

Shifting the focus from control to data in choreographic framework has
several implications.
%
Firstly, the emphasis is no longer on properties related to
computational actors.
%
For instance, Klaimographies admit computations where some processes
may not terminate and is left waiting for some data.
%
In standard choreographic framework those would be undesired behaviour
to rule out with suitable typing disciplines.
%
Nonetheless, we claim that in some application domain computations with
deadlocked processes have to be considered non-erroneous.
%
For instance, in reactive systems based on event-notification
frameworks some \quo{listener} components must be kept waiting for
events to occur.
%
Our work paves the way to the formal study of properties of data, like consumption, persistence and availability, in a choreographic setting.

Another main innovation of Klaimographies is that they allow us to easily
represent protocols where an arbitrary number of components can
play a role in the choreography.
%
We give an example of such protocol in \cref{sec:examples}.
%
Remarkably, those protocols can be specified in some existing
choreographic frameworks~\cite{ydbh10,chjny19}, but in a less abstract way
that requires the explicit quantification on components.

\paragraph{Structure of the paper}
\todo{eM: revise}
After some preliminaries in Section~\ref{sec:tuples}, we define Klaimographies as global types in Section~\ref{sec:gt} and give some examples in Section~\ref{sec:examples}.
In Section~\ref{sec:globsem} we define the semantics of global types and give the adequacy conditions for projecting global types to local types.
In Section~\ref{sec:locsem} we define the operational semantics of local types and state its correspondence with the global semantics.
Some concluding remarks together with the discussion of related and future work are in Section~\ref{sec:conc}.

%%% Local Variables:
%%% mode: latex
%%% TeX-master: "main"
%%% End:



\section{Klaim-inspired Choreographies}
\label{sec:klaimographies}
\subsection{Tuple types}
\label{sec:tuples}
% !TEX root =  main.tex
%

We consider a set of variables $\varset$ 
\todo{RB: aggiunto}
and a set of localities $\locset$ ranged over by $\aloc$ (and
use $\alocvar$ to range over $\locset\cup\varset$) and we let $\asort$
range over basic sorts which include $\asort[int]$, $\asort[bool]$,
$\asort[str]$ and sort $\asort[loc]$ of \emph{localities}.
%
The set $\tupleset$ of \emph{(tuple) types} consists of the terms
derived from the following grammar:
\begin{eqnarray*}
  \atuple & \bnfdef & \asort \bnfmid
                      x : \asort \bnfmid
                      \nu x : \asort  \bnfmid
%                      \nu \aloc \bnfmid
%                      \aloc \bnfmid
                      \atuple \cdot \atuple \bnfmid
                      \wildcard
\end{eqnarray*}
\todo{RB: spiegare il tipo $\wildcard$ (any type? empty tuple?)}
Fields $\nu x : \asort$ are binders that \emph{define}
$x \in \varset$.
%
Hence, we talk about \emph{free} and \emph{defined} (sorted) names occurring in
tuples.
%
%We assume that in a tuple there is at most one occurrence of $\nu x$,
%for each $x \in \varset$.
%
Formally, the functions $\fn{\_}$ and $\dn{\_}$ returns sets of pairs
$x \mapsto \asort$ assigning sort $\asort$ to $x \in \varset$.
\[
\begin{array}{l@{\hspace{1cm}}l}
\begin{array}{lcl}
  \dn\asort & = & \emptyset
  \\
  \dn{x : \asort} & = & \emptyset
  \\
  \dn{\nu x : \asort} & = & \{x \mapsto \asort \}
  \\
  \dn{\atuple_1 \cdot \atuple_2} & = & \dn{\atuple_1} \cup \dn{\atuple_2}
  \\
  \dn{\wildcard} & = & \emptyset
\end{array}
&
\begin{array}{lcl}
  \fn\asort & = & \emptyset
  \\
  \fn{x : \asort} & = &  \{x \mapsto \asort\}
  \\
  \fn{\nu x : \asort} & = & \emptyset
  \\
  \fn{\atuple_1 \cdot \atuple_2} & = & \fn{\atuple_1} \cup \fn{\atuple_2}
  \\
  \fn{\wildcard} & = & \emptyset
\end{array}
\end{array}
\]
We write $\supp{\_}$ to denote the projection of a set of pairs over its first component. 

We say a tuple $\atuple$ is {\em well-sorted} if the following two
conditions hold:
\begin{itemize}
\item
  $\supp{\fn\atuple} \cap \supp{\dn{\atuple}} = \emptyset$, i.e., free and
  defined names are disjoint; and
\item
  $\atuple = \atuple_1 \cdot \atuple_2$ implies $\atuple_1$ and
  $\atuple_2$ well-sorted and $\supp{\dn{\atuple_1}} \cap \supp{\dn{\atuple_2}} =
  \emptyset$, i.e., defined names are all different.
%  \todo{RB: non \`e conseguenza del fatto che ``We assume that in a tuple there is at most one occurrence of $\nu x$, for each $x \in \varset$''}
\end{itemize}
%
Hereafter, we assume all tuples to be well-sorted.
%
Note that $\fn{\atuple}$ and $\dn{\atuple}$ are partial functions (from names to sorts)
\todo{RB: aggiunto}
for well-sorted tuples.

A capture-avoiding
\todo{RB: toglierei capture-avoiding aggiungendo condizioni esplicite su $x,y,\atuple$.}
substitution of the free occurrences of a variable
$x$ in a (well-sorted) tuple $\atuple$ such that $x\not\in \dn{\atuple}$
\todo{RB: aggiunto}
 by a variable $y\not\in \dn{\atuple}$, written
$\atuple \sust x y$, is defined such that
%
\[
\begin{array}{r@{}l@{\ = \ } ll}
%\asort
%&
%\sust x y  
%&  
%\asort
%\\
%(x : \asort)
%& 
%\sust x y  
%& x : \asort
%\\
%z
%&
%\sust x y 
%&  
%z 
%& 
%{\it if} z\neq x
%\\
(x  : \asort)
&
\sust x y  
&  
y : \asort
\\
(\atuple_1 \cdot \atuple_2)
&
\sust x y  
& 
(\atuple_1\sust x y) \cdot (\atuple_2\sust x y) 
%& 
%{\it if} y\not\in\dn{\atuple_1 \cdot \atuple_2}
%\\
%\wildcard
%&
%\sust x y  
%&  
%\wildcard
\end{array}
\]
%
and it is the identity on the remaining cases. Let
$\sigma = \{y_1/x_1,\ldots,y_n/x_n\}$ such that $x_i\neq x_j$ for all
$i\neq j$ (i.e., a partial endo-function on $\varset$), we write
$\atuple\sigma$ for the simultaneous substitution of each $x_i$ by
$y_i$.
%
We use $\Sigma$ for the set of all substitutions. We write
$\sigma_1\sigma_2$ for the composition of partial functions with
disjoint domain, and $\sigma_1[\sigma_2]$ for the update of $\sigma_1$
with $\sigma_2$.


Tuple types \emph{match} by producing a substitution; formally given
by the partial function
$\matches : \tupleset \times \tupleset \to \Sigma$ defined 
\todo{RB: in $\atuple \matches \atuple'$ si richiede anche che i defined names di $\atuple$ e $\atuple'$ siano diversi?}
such as
%\eMnote{add typing env}
\[
  \atuple \matches \atuple' \generates
    \begin{cases}
     \emptyset
    & 
    \text{if  } \atuple = \wildcard \vee \atuple' = \wildcard  \vee (\atuple = \atuple' \land \atuple\in\{\asort, x : \asort\})
%    \\
%    \sigma_1\sigma_2
%    &
%    \text{if } \atuple = \atuple_1 \cdot \atuple_2
%    \land  \atuple' = \atuple'_1 \cdot \atuple'_2
%    \land \atuple_1 \matches \atuple'_1 \generates \sigma_1
%    \land \atuple_2 \matches \atuple'_2 \generates \sigma_2
    \\
    \sigma
    &
    \text{if } \atuple = \atuple_1 \cdot \atuple_2
    \land  \atuple' = \atuple'_1 \cdot \atuple'_2
    \land \atuple_1 \matches \atuple'_1 \generates \sigma_1
    \land \atuple_2\sigma_1 \matches \atuple'_2\sigma_1 \generates \sigma
    \\
    \sust x y 
    &
    \text{if  } (\atuple = \nu y : \asort \land \atuple' = x: \asort) \vee  (\atuple' = \nu y : \asort \land \atuple = x: \asort) 
    \\
    \textit{undef} & \textit{otherwise}
   \end{cases}
\]
\todo{RB: cambiato caso $\atuple = \atuple_1 \cdot \atuple_2$:
  controllare se la definizione \`e corretta}
%
\todo{eM: non dovrebbe essere $\sust y x$? RB: cambiato come suggerisce eM}
%
We write $\atuple \matches \atuple'$ when
$\atuple \matches \atuple' = \sigma$ for a substitution
$\sigma \in \Sigma$.

%
%
%\[
%  \atuple \matches \atuple' \iff
%  \begin{cases}
%    \atuple = \atuple'  
%    & 
%    \text{if $\atuple\in\{\asort, x : \asort, x, \nu \aloc, \aloc\}$}
%    \\
%    \atuple_1 \matches \atuple'_1
%    \land \atuple_2 \matches \atuple'_2
%    &
%    \text{if } \atuple = \atuple_1 \cdot \atuple_2
%    \text{ and }  \atuple' = \atuple'_1 \cdot \atuple'_2
%    \\
%    \atuple = \wildcard \vee \atuple' = \wildcard & \text{otherwise}
%  \end{cases}
%\]


%
We say that $\atuple$ \emph{generates} when in one of its fields there
is a $\nu x: \asort[loc]$ type. 
%\todo{RB: meglio dire $\nu \aloc: \asort[loc]$}
%

%We define
%\[
%  \atuple \clashes \atuple' \iff \exists \atuple'' \qst \atuple'' \matches \atuple \land \atuple'' \matches \atuple'
%\]
%
%and say that $\atuple$ and $\atuple'$ \emph{are in conflict} when
%$\atuple \clashes \atuple'$.


%%% Local Variables:
%%% mode: latex
%%% TeX-master: "main"
%%% End:

\subsection{Global types}
\label{sec:gt}
% !TEX root =  main.tex
%

We fix two disjoint sets $\participants = \{\ptp p, \ptp q, \ldots\}$
and $\multiroles = \{\amulti P, \amulti Q, \ldots \}$, respectively of
\emph{participants} and \emph{multiple} roles, and define the set of
\emph{roles} $\roleset = \participants \cup \multiroles$.
\todo{AC+RB: forse sarebbe meglio parlare di \emph{unique roles} o \emph{unit roles} invece che di \emph{pariticipants}? la terminologia sarebbe pi\`u uniforme}
% \cup \multiroles^\unknownstarop \cup
% \multiroles^\unknownop$ (ranged over by $\arole, \arole_1,
% \ldots$) where $\multiroles^{\unknownstarop} = \{\unknownstar \sst
% \amulti P \in\multiroles\}$ and $\multiroles^{\unknownop} = \{
% \unknown \sst \amulti p \in \multiroles
% \}$ account for some flexibility when implementing multiroles: an
% action involving a
% $\unknownstar$ role can be optionally executed by an implementation
% of role $\amulti P$, while
% $\unknown$ establishes that exactly one implementer must execute
% that action.
%
We conventionally write multiroles with initial uppercase letter and
unique roles with initial lowercase letter.
%
%We write $\eqR$ for the least equivalence relation on $\roleset$
%satisfying
%\[\amulti P \eqR \unknownstar \hspace{2cm} \amulti P \eqR \unknown\]

Roles are to be thought of as types inhabited by instances of
processes enacting the behaviour specified in a choreography.
%
Participants are unit types while multiroles account for multiple
instances of processes all performing actions according to their role.
%
%\todo{RB: forse \`e importante dire che tutti i multiruoli sono disgiunti, giusto?}
%\todo{eM: direi di no; in the long run vorremmo che un processo possa coprire piu' ruoli}

%
Let us first define the \emph{prefixes} used in global types
with the following grammar:
% \todo{RB: ``consuming output'' mi sembra fuorviante... ma non ho proposte migliori}
%
\[\begin{array}{lcl@{\qquad\qquad}l}
  \apref & \bnfdef
  & \apref[\arole][@][@][\alocvar] & \text{(autonomous) output}
  \\ & \bnfmid
  & \apref[\arole][{}][@][\alocvar][.] & \text{(autonomous) read-only output}
  \\ & \bnfmid
  & \apref[{}][\arole][@][\alocvar] & \text{(autonomous) input}
  \\ & \bnfmid
  & \apref[{}][\arole][@][\alocvar][.] & \text{(autonomous) read}
  \\ & \bnfmid
  & \apref[\arole][\arole'][@][\alocvar] & \text{consuming interaction}
  \\ & \bnfmid
  & \apref[\arole][\arole'][@][\alocvar][.]  & \text{read-only interaction}
  \end{array}
\]
%
The set $\roles \apref \subseteq \roleset$ of roles in $\apref$ is
defined in the obvious way; note that $\roles \apref$ is a singleton
if, and only if, $\apref$ is an autonomous prefix.
%
We syntactically distinguish two kinds of prefixes.
%
The prefixes generated by the first four productions in the grammar of
$\apref$ above are the \emph{autonomous} prefixes, that is those
prefixes that processes can execute directly on a tuple space without
coordinating with other processes.
%
The prefixes generated by the remaining two productions of the grammar
are the \emph{interaction} prefixes, namely those involving a role
generating the tuple and one accessing them.
%
Inspired by Klaim, processes can access tuple types according to two
modalities syntactically distinguished by the round brackets around
the tuple in prefixes.
%
More precisely, when a prefix surrounds a tuple $\atuple$ with round
brackets then $\atuple$ is meant to be consumed otherwise it is meant
to be read-only.
%
% We call \emph{read-only outputs} and \emph{reads} the autonomous
% read-only prefixes while those prescribing the consumption of tuples,
% and consuming (resp. read-only) interactions the interations
% prescribing inputs (resp. reads).
%
We assume that tuple types used in read-only modalities do not
generate.
%
\eMnote{Una variante interessante e' quella dove questo vincolo e'
  rilassato.}

Global types $\aK$ have the following syntax
% \eMnote{non e' chiaro se vogliamo $\aK \parop \aK $}
\begin{eqnarray*}
  \aK & \bnfdef & \asum[@][@][][]
                  \bnfmid
                  % \aK \parop \aK \bnfmid
                  \aK \seqop \aK \bnfmid
                  X \bnfmid
                  \grec
\end{eqnarray*}
% \todo{RB: avendo $\seqop$ direi che il parallelo non \`e necessario}
where $I$ is a finite set of indexes; we write $\nil$ for
$\asum[@][@][][]$ when $I = \emptyset$ (we omit trailing occurrences
of $\nil$) and $\apref_j.\aK_j$ instead of $\asum[@][@][][]$ when
$I = \{j\}$.
%
The set $\roles \aK$ of roles of $\aK$ is the set of roles that are mentioned in $\aK$ and it is defined in the obvious way.

The syntax of global types features prefix guarded choices, sequential
composition, and recursion.
%
To handle recursive behaviour, 
\todo{AC+RB: il comportamento della ricorsione si intuisce solo dopo aver visto la semantica. Forse bisognerebbe tentare di spiegarlo intuitivamente?}
the construct $\grec$ singles out a role $\arole \in \roles \aK$
%
% and specifies an injective function
% $\aphi : \roles \aK \setminus \{\arole\} \to \locset$ such that the
% locations in $\cod(\aphi)$ do not occur in $\aK$; intuitively,
% $\arole$
deciding when the recursion ends.
\todo{RB: pi\`u sotto viene usata la notazione $\alvar$ invece di $X$: bisogna cambiare la grammatica delle coreografie?}
% and, for all $\arole' \neq \arole$ in $\aK$, the location
% $\aphi(\arole')$ is used to communicate the decision of $\arole$ to
% $\arole'$.
%
We omit the decoration $\arole$ when $\roles \aK = \{\arole\}$.

We extend the notions of defined and free names to global types as
follows:
%\todo{RB: ma la $\nu$ pu\`o essere usata anche in una tupla di input/read (senza output)?}
%\todo{eM: si, ma poi non sincronizza}
\[
 \fn{\apref[\arole][@][@][\alocvar]}
 = \fn \atuple \cup \{\alocvar \mapsto \asort[loc]\} 
\qquad
 \dn{\apref[\arole][@][@][\alocvar]} 
 = \dn \atuple 
\]
omitted prefixes are defined analogously.
\[
  \begin{array}{ll}
    \begin{array}{l@{\ =\ } ll}
      \fn{\asum[@][@][][]} & \displaystyle{\bigcup_{i\in I}} \fn{\apref_i} \cup (\fn{\aK_i}\setminus\dn{\apref_i})
      \\
      \fn{\aK_1 \seqop \aK_2} 
                &
                  \fn{\aK_1}\cup\fn{\aK_2}
      \\
      \fn \alvar & (\seq \alocvar \cap \varset)
      \\
      \fn \grec & \fn {\aK}
    \end{array}
    \begin{array}{l@{\ =\ } ll}
      \dn{\asum[@][@][][]} & \displaystyle{\bigcup_{i\in I}} \dn{\apref_i} \cup \dn{\aK_i}
      \\
      \dn{\aK_1 \seqop \aK_2} 
                           &
                             \dn{\aK_1}\cup\dn{\aK_2}
      \\
      \dn \alvar & \emptyset
      \\
      \dn \grec & \dn {\aK}
    \end{array}
  \end{array}
\]
%
We write $\names \_$ for the set of sorted names of a term, i.e.,
$\names \apref = \fn\apref \cup \dn\apref$ and similarly
$\names \aK = \fn\aK \cup \dn\aK$. A set $S$ of sorted names is
consistent, written $\consistent S$, if $x\mapsto \asort \in S$ and
$x\mapsto \asort' \in S$ implies $\asort = \asort'$.
 
The set of well-sorted terms are defined inductively as follows:

\begin{itemize}
\item $\apref$ is well-sorted if $\fn\apref\cap\dn\apref = \emptyset$ and  
$\consistent{\names\apref}$, i.e., there are no clashes/inconsistencies in the sorts of 
the names in the component $\atuple$ of $\pi$ and the locality $\alocvar$ mentioned in $\pi$;
\item $\asum[@][@][][]$ is well-sorted if for all ${i\in I}$ both
  $\apref_i$ and ${\aK_i}$ are well-sorted and
  $\consistent{\names {\apref_i.\aK_i}}$;
\item $\aK_1 \seqop \aK_2$ is well-sorted if $\aK_1$ and $\aK_2$ are
  well-sorted and $\consistent{\names{\aK_1 \seqop \aK_2}}$;
\item $X$ is well-sorted and $\grec$ is well-sorted if $\aK$ is
  well-sorted.
\end{itemize}


We consider terms up-to $\alpha$-renaming of defined names and
recursion variables.
%
Correspondingly, substitutions are capture avoiding, in the sense that
defined names can be renamed to fresh names before any substitution is
applied to a term.
% \todo{RB: aggiunto}
%
As usual we say that a global type $\aK$ is \emph{closed} when it does
not contain free occurrences of recursion variables $X$ or free
occurrences of names.%, i.e., $\supp{\fn\aK} \cap \varset = \emptyset$.


  
%%% Local Variables:
%%% mode: latex
%%% TeX-master: "main"
%%% End:

\subsection{Some examples}
\label{sec:examples}
% !TEX root =  main.tex

We give a few simple global types (\cref{ex:cs,ex:CS,ex:CSx,ex:CSl})
to highlight some peculiarities of our choreographies as well as a
more complex example (\cref{ex:market}) to illustrate the type of
protocols our choreographies can capture.

\begin{example}\label{ex:cs}
  Consider the following global type that describes the interaction of
  a client $\ptp c$ with a simple service $\ptp s$ that converts
  integers into strings.
  \[
    \aK_\eqref{ex:cs} =
    \apref[\ptp c][\ptp s][{\asort[int]}][{\aloc[l]}]  \prefop
    \apref[\ptp s][\ptp c][{\asort[str]}][{\aloc[l]}]
  \]
  The client $\ptp c$ produces an integer value on the locality
  $\aloc$.
  %
  This tuple must be consumed by the server $\ptp s$, which produces
  back the converted string for the client.
  %
  \finex
\end{example}

Elaborating on the previous example we now discuss a few features of
our setting.

\begin{example}\label{ex:CS}
  Assume that we consider client and server in \cref{ex:cs} as
  multiple instead of unit roles, and write
  \[
    \aK_\eqref{ex:CS} =
    \apref[\amulti {C}][\amulti {S}][{\asort[int]}][{\aloc[l]}] \prefop
    \apref[\amulti {S}][\amulti {C}][{\asort[str]}][{\aloc[l]}]
  \]
  In this case, $\aK_\eqref{ex:CS}$ states that each integer produced by a client
  will be consumed by a server, which will in turn produce a string
  for one of the clients.
\end{example}
The type in \cref{ex:CS} does not ensure that clients consume the
string conversion of the integer they produced, because all tuples are put at the same location $\aloc[l]$.
%
Name binders can be used to correlate tuples.
%
\begin{example}\label{ex:CSx}
  Consider
  \[
    \aK_\eqref{ex:CSx} =
    \apref[\amulti {C}][\amulti {S}][{\nu x:\asort[int]}][{\aloc[l]}] \prefop
    \apref[\amulti {S}][\amulti {C}][{x : {\asort[int]} \cdot {\asort[str]} }][{\aloc[l]}]
  \]
%  $\aK_\eqref{ex:CSx}$ specifies that a fresh identifier $x$ is established between the 
% instances of $\amulti C$ and $\amulti S$ so that they refer to the same integer value.
  The first interaction binds the occurrence of $x$ in the second interaction. 
  The use of $x$ in the second interaction constraints the 
  instances of $\amulti S$ and $\amulti C$ to share a tuple 
  whose integer expression matches the integer shared in the first interaction.
  %
  Despite the identifier is known only to the communicating instances,
  this does not forbid two clients to generate the same integer value.
  %\todo{RB: questo \`e controintuitivo: quando usiamo il $\nu$ sulle locazioni non vogliamo che sia possibile generare due volte la stessa locazione, altrimenti non risolveremmo il problema. Sostituirei $\nu x$ con Solo $x$: mettere $\nu x$ costringerebbe a generare interi diversi, ma questo non \`e desiderabile}
  %
 % Basically, the name $x$ allows to express constraints about the flow
 %of values.
  %
  %In particular, $x : \asort[int]$ in the second interaction states
  %that a server must generate a tuple that contains the consumed
  %integer.
  %
  %Analogously, each client must consume a tuple matching the produced
  %integer.
  %
  \finex
\end{example}
The choreography in \cref{ex:CSx} does not establishes a one-to-one
association between instances of $\amulti C$ and $\amulti S$.
%
In fact, an instance of $\amulti C$ not necessarily interacts with the
same instance of $\amulti S$ in the two communications when two
instances of $\amulti C$ generate the same integer in the first
interaction.
%
A one-to-one correspondence can be achieved by using fresh localities.
\begin{example}\label{ex:CSl}
  Consider
  \[
    \aK_\eqref{ex:CSl} = 
    \apref[\amulti {C}][\amulti {S}][{{\asort[int]} \cdot \nu x: {\asort[loc]}}][{\aloc[l]}]  \prefop
    \apref[\amulti {S}][\amulti {C}][{{\asort[str]}}][{x}]
  \]
  %
  Each client generates a fresh locality identified by $x$, which is then
  used as the locality for the subsequent communication.
  %
  Since the name $x$ is known only to the two communicating instances,
  the second interaction can only take place between the two instances
  that know the locality $x$.
  %
  \finex
\end{example}

% \todo[inline]{{We think we can also deal with situations like the following, in which 
% A creates a private session for B and C. 
% %
% \[
%   \apref[\amulti {A}][\amulti {B}][{\nu y: {\asort[loc]}}][{\aloc[l]}] \prefop
% \]
% \[
%   \apref[\amulti {A}][\amulti {C}][{y : {\asort[loc]}}][{\aloc[l]}] \prefop
% \]
% \[
%   \apref[\amulti {B}][\amulti {C}][\asort][{y}]
% \]
% }}


The following example shows the kind of protocols that our global
types can model in a realistic scenario allowing us to combine
together most of the features of our framework.
%
For readability,  we use the notation
$\grec[@][@][@][1]$ for a recursive protocol where the body $\aK$ is
repeated at least once.
%
Formally,
\[
\grec[@][@][@][1] =
  \aK \sust{X}{\grec} 
\]

%For readability, given $h \geq 0$, we use the notation
%$\grec[@][@][@][h]$ for a recursive protocol where the body $\aK$ is
%repeated at least $h$ times.
%\todo{RB: visto che la notazione ci serve solo per $h=1$ possiamo semplificare definendo solo questo caso (motivandolo col fatto che nell'esempio usiamo una forma di do-while invece che di while-do)}
%
%Formally,
%\[
%\grec[@][@][@][h] =
%\begin{cases}
%  \grec[@][@][@] & \text{if } h = 0
%  \\
%  \aK \sust{X}{\grec[@][@][@][h-1]} & \text{otherwise}
%\end{cases}
%\]
%
% \eMnote{forse non e' un'asta...ma una contrattazione}
\begin{example}[Market place]\label{ex:market}
  % 
  Several sellers and buyers use a market place provided by a
  broker.
  % 
  Sellers can put on sale (several) items and buyers can inspect them.
  %
  When an item of interest is found, the client can starts a negotiation
  with the seller.
  % 
  This protocol can be formalised by the following global type.
  \[
    \begin{array}{l}
      \apref[\ptp {broker}][\amulti {Seller}][{\atuple[start]}][{\aloc[m]}][.] \prefop
      \\
      \grec[][X][{\apref[\amulti {Seller}][@][{{\asort[str]} \cdot {\asort[int]} \cdot {\nu l : \asort[loc]}}][{\aloc[m]}] \prefop X}][1] \seqop
      \\  	
      \grec[\amulti{Buyer}][Y][{\left({
      \begin{array}{l}
        \grec[][Z][{\apref[@][\amulti {Buyer}][{{\asort[str]} \cdot {\asort[int]} \cdot {\asort[loc]}}][{\aloc[m]}][.] \prefop Z}] \seqop
        \\
        \apref[@][\amulti {Buyer}][{{i : \asort[str]} \cdot {p : \asort[int]} \cdot {\nu l : \asort[loc]}}][{\aloc[m]}] \prefop
        \\
        \grec[\amulti{Seller}][W][\left({
        \begin{array}{l}
          \apref[\amulti {Buyer}][\amulti {Seller}][{i : \asort[{str}] \cdot {o : \asort[int]} }][{l}] \prefop
          \\
          \qquad 
          \apref[\amulti {Seller}][\amulti {Buyer}][{\asort[quit]}][{l}] \prefop
          \\
          \qquad 
          \apref[\amulti {Seller}][@][{{i : \asort[str]} \cdot {p : \asort[int]} \cdot {\nu l : \asort[loc]}}][{\aloc[m]}] \prefop
          \\
          \qquad
          Y	
          \\
          \qquad
          \chop
          \\
          \qquad 
          \apref[\amulti {Seller}][\amulti {Buyer}][{\asort[sold]}][{l}]\prefop
          %	\\
          %	\qquad \apref[\amulti {Seller}][\ptp {broker}][i :
          % {\asort[str]} \cdot o : {\asort[int]} ][{\aloc[m]}]\prefop
          % \\ \qquad
          Y	
          \\
          \qquad
          \chop
          \\
          \qquad
          \apref[\amulti {Seller}][\amulti {Buyer}][{\asort[more]}][{l}] \prefop W
          \\
          \chop
          \\
          \apref[\amulti {Buyer}][\amulti {Seller}][{\asort[{noway}]}][{l}] \prefop
          \\\qquad
          \apref[\amulti {Seller}][@][{{i : \asort[str]} \cdot {p : \asort[int]} \cdot {\nu l : \asort[loc]}}][{\aloc[m]}] \prefop
          \\
          \qquad
          Y
        \end{array}
        }\right)][1]
      \end{array}
      }\right)}][1]
    \end{array}
  \]
  % captures the behaviour sketched above.
  % 
  The broker is a unit role that triggers sellers to start
  advertising their items on the marketplace location $\aloc[m]$.
%
Sellers and buyers are modelled as multiple roles.
%
Each seller advertises one or more items at $\aloc[m]$ (see recursion at line~2).
%
Each buyer can inspect the advertised items (line~3) and eventually
start bargaining on a selected item of interest.
%
Note that the consumption at line~4 instantiates a private location
$l$ between the instance of $\amulti{Seller}$ advertising the item and
the instance of $\amulti{Buyer}$ interested in buying it.
\todo{RB: ma nella linea 4 perch\'e serve $\nu$ davanti a $l$? non dovrebbe essere $\apref[@][\amulti {Buyer}][{{i : \asort[str]} \cdot {p : \asort[int]} \cdot {l : \asort[loc]}}][{\aloc[m]}]$? il valore \`e quello settato dal seller come per $i$ e $p$.}
%
Location $l$ is used to perform the bargaining phase.

The seller instance controls the recursion
$\grec[\amulti{Seller}][W][\cdots][1]$; the body of the recursive type
lets the buyer sharing location $l$ to decide whether to stop the
bargaining (by exchanging a $\asort[noway]$ tuple, in which case the
seller re-advertises the unsold item at $\aloc[m]$) or to make an
offer to the seller (which can then decide either to stop the
bargaining, or to struck a deal, or ask for an higher offer).
%
\finex
\end{example}



%%% Local Variables:
%%% mode: latex
%%% TeX-master: "main"
%%% End:

%% !TEX root =  main.tex
%
\subsection{Tuple types}
We consider $\asort$ range over basic sorts which include
$\asort[int]$, $\asort[bool]$, $\asort[str]$ and sort $\asort[loc]$ of
\emph{localities}. We assume $\varset$ to be a set of variables $x, y,
z, \ldots$.
%
The set $\tupleset$ of \emph{(tuple) types} consists of the terms
derived from the following grammar:
\begin{eqnarray*}
  \atuple & \bnfdef & \asort \bnfmid
                      x : \asort \bnfmid
                      \nu x : \asort  \bnfmid
%                      \nu \aloc \bnfmid
%                      \aloc \bnfmid
                      \atuple \cdot \atuple \bnfmid
                      \wildcard
\end{eqnarray*}
Names $x$ in tuples $\nu x : \asort$ are binders. Hence, we talk about
free and defined (sorted) names occurring in tuples. Formally, the functions
$\fn{\_}$ and $\dn{\_}$ returns sets of pairs that associate variables to sorts.
\[
\begin{array}{l@{\hspace{1cm}}l}
\begin{array}{lcl}
  \dn\asort & = & \emptyset
  \\
  \dn{x : \asort} & = & \emptyset
  \\
  \dn{\nu x : \asort} & = & \{x \mapsto \asort \}
  \\
  \dn{\atuple_1 \cdot \atuple_2} & = & \dn{\atuple_1} \cup \dn{\atuple_2}
  \\
  \dn{\wildcard} & = & \emptyset
\end{array}
&
\begin{array}{lcl}
  \fn\asort & = & \emptyset
  \\
  \fn{x : \asort} & = &  \{x \mapsto \asort\}
  \\
  \fn{\nu x : \asort} & = & \emptyset
  \\
  \fn{\atuple_1 \cdot \atuple_2} & = & \fn{\atuple_1} \cup \fn{\atuple_2}
  \\
  \fn{\wildcard} & = & \emptyset
\end{array}
\end{array}
\]
We write $\supp{\_}$ to denote the projection of a set of pairs over its first component. 

We say a tuple $\atuple$ is {\em well-sorted} if the following two
conditions hold:
\begin{itemize}
\item
  $\supp{\fn\atuple} \cap \supp{\dn{\atuple}} = \emptyset$, i.e., free and
  defined names are disjoint; and
\item
  $\atuple = \atuple_1 \cdot \atuple_2$ implies $\atuple_1$ and
  $\atuple_2$ well-sorted and $\supp{\dn{\atuple_1}} \cap \supp{\dn{\atuple_2}} =
  \emptyset$, i.e., defined names are all different.	 
\end{itemize}
%
Hereafter, we assume all tuples to be well-sorted. Note that 
$\fn{\atuple}$ and $\dn{\atuple}$ is a partial function for a well-sorted tuple.

A capture-avoiding substitution of the free occurrences of a variable
$x$ in a (well-sorted) tuple $\atuple$ by a variable $y$, written
$\atuple \sust x y$, is defined as follows
%
\[
\begin{array}{r@{}l@{\ = \ } ll}
%\asort
%&
%\sust x y  
%&  
%\asort
%\\
%(x : \asort)
%& 
%\sust x y  
%& x : \asort
%\\
%z
%&
%\sust x y 
%&  
%z 
%& 
%{\it if} z\neq x
%\\
(x  : \asort)
&
\sust x y  
&  
y : \asort
\\
(\atuple_1 \cdot \atuple_2)
&
\sust x y  
& 
(\atuple_1\sust x y) \cdot (\atuple_2\sust x y) 
%& 
%{\it if} y\not\in\dn{\atuple_1 \cdot \atuple_2}
%\\
%\wildcard
%&
%\sust x y  
%&  
%\wildcard
\end{array}
\]
%
and it is the identity on the remaining cases. Let $\sigma =
\{y_1/x_1,\ldots,y_n/x_n\}$ such that $x_i\neq x_j$ for all $i\neq j$
(i.e., a partial function), we write $\atuple\sigma$ for the
simultaneous substitution of each $x_i$ by $y_i$.
%
We use $\Sigma$ for the set of all substitutions. We write
$\sigma_1\sigma_2$ for the composition of partial functions with
disjoint domain, and $\sigma_1[\sigma_2]$ for the update of $\sigma_1$
with $\sigma_2$.


Tuple types \emph{match} by producing a substitution; formally given
by the partial function ${\matches} : \tupleset \times \tupleset
\rightarrow \Sigma$ defined such as
%\eMnote{add typing env}
\[
  \atuple \matches \atuple' \generates
    \begin{cases}
     \emptyset
    & 
    \text{if  } \atuple = \wildcard \vee \atuple' = \wildcard  \vee (\atuple = \atuple' \land \atuple\in\{\asort, x : \asort\})
    \\
    \sigma_1\sigma_2
    &
    \text{if } \atuple = \atuple_1 \cdot \atuple_2
    \land  \atuple' = \atuple'_1 \cdot \atuple'_2
    \land \atuple_1 \matches \atuple'_1 \generates \sigma_1
    \land \atuple_2 \matches \atuple'_2 \generates \sigma_1
    \\
    \sust x y 
    &
    \text{if  } (\atuple = \nu x : \asort \land \atuple' = y: \asort) \vee  (\atuple' = \nu y : \asort \land \atuple = x: \asort) 
    \\
    \textit{undef} & \textit{otherwise}
   \end{cases}
\]
\todo{double-check}

%
%
%\[
%  \atuple \matches \atuple' \iff
%  \begin{cases}
%    \atuple = \atuple'  
%    & 
%    \text{if $\atuple\in\{\asort, x : \asort, x, \nu \aloc, \aloc\}$}
%    \\
%    \atuple_1 \matches \atuple'_1
%    \land \atuple_2 \matches \atuple'_2
%    &
%    \text{if } \atuple = \atuple_1 \cdot \atuple_2
%    \text{ and }  \atuple' = \atuple'_1 \cdot \atuple'_2
%    \\
%    \atuple = \wildcard \vee \atuple' = \wildcard & \text{otherwise}
%  \end{cases}
%\]


%
We say that $\atuple$ \emph{generates} when in one of its fields there
is a $\nu \aloc$ type. \todo{check}
%

%We define
%\[
%  \atuple \clashes \atuple' \iff \exists \atuple'' \qst \atuple'' \matches \atuple \land \atuple'' \matches \atuple'
%\]
%
%and say that $\atuple$ and $\atuple'$ \emph{are in conflict} when
%$\atuple \clashes \atuple'$.

\subsection{Global types}
We fix two disjoint sets $\participants = \{\ptp p, \ptp q, \ldots\}$
and $\multiroles = \{\amulti P, \amulti Q, \ldots \}$, respectively of
\emph{participants} and \emph{multiple} roles, and define the set of
\emph{roles} $\roleset = \participants \cup \multiroles$
% \cup
%\multiroles^\unknownstarop \cup \multiroles^\unknownop$ (ranged over
%by $\arole, \arole_1, \ldots$) where
%$\multiroles^{\unknownstarop} = \{\unknownstar \sst \amulti P
%\in\multiroles\}$ and
%$\multiroles^{\unknownop} = \{ \unknown \sst \amulti p \in \multiroles
%\}$ account for some flexibility when implementing multiroles: an
%action involving a $\unknownstar$ role can be optionally executed by
%an implementation of role $\amulti P$, while $\unknown$ establishes
%that exactly one implementer must execute that action.
%
We conventionally write multiroles with initial uppercase letter and
unique roles with initial lowercase letter.
%
%We write $\eqR$ for the least equivalence relation on $\roleset$
%satisfying
%\[\amulti P \eqR \unknownstar \hspace{2cm} \amulti P \eqR \unknown\]

Roles are to be thought of as types inhabited by instances of
processes enacting the behaviour specified in a choreography.
%
Participants are unit types while multiroles account for multiple
instances of processes all performing actions according to their role.

We consider a set of localities $\locset$ ranged over by $\aloc$. We use 
$\alocvar$ to range over $\locset\cup\varset$.

Global types $\aK$ have the following syntax
%\eMnote{non e' chiaro se vogliamo $\aK \parop \aK $}
\begin{eqnarray*}
  \aK & \bnfdef & \asum \bnfmid
%                  \aK \parop \aK \bnfmid
	    	 \aK; \aK \bnfmid
                  X \bnfmid
                  \arec
  \\[1.5em]
  \apref & \bnfdef & \apref[\arole][@][@][\alocvar] \bnfmid
                     \apref[\arole][{}][@][\alocvar][.] \bnfmid
                     \apref[{}][\arole][@][\alocvar] \bnfmid
                     \apref[{}][\arole][@][\alocvar][.]
  \\[1em]
        & \bnfmid & \apref[\arole][\arole'][@][\alocvar] \bnfmid
                    \apref[\arole][\arole'][@][\alocvar][.]
\end{eqnarray*}
where $I$ is a finite set of indexes.
% and $\aloc \in \locset$ is a locality.
%
We write $\nil$ for $\asum$ when $I = \emptyset$ and
$\apref_j.\aK_j$ instead of $\asum$ when $I = \{j\}$.
%
We omit trailing occurrences of $\nil$.

We syntactically distinguish two kinds of prefixes $\apref$.
%
The prefixes generated by the first four productions in the grammar of
$\apref$ above are the \emph{autonomous} prefixes, that is those
prefixes that processes can execute directly on a tuple space without
coordinating with other processes; the prefixes generated by the
remaining two productions of the grammar are the \emph{interaction}
prefixes, namely those involving more roles.
%
We define $\roles \apref$ to the set of roles in $\apref$; note that
$\roles \apref$ is a singleton if, and only if, $\apref$ is an
autonomous prefix.
%
Likewise, $\roles \aK \subseteq \participants \cup \multiroles$ is the
set of roles in $\aK$.

We extend the notions of defined and free names to global types as follows:

\[
\begin{array}{ll}
\begin{array}{l@{\ =\ } ll}
 \fn{\apref[\arole][@][@][\alocvar]} 
 & \fn \atuple \cup \{\alocvar \mapsto \asort[loc]\} 
 \\
 \multicolumn{3}{c}{\hspace{1cm}\textit{omitted prefixes are defined analogously}}
 \\ 
 \fn \asum & \bigcup_{i\in I} \fn{\apref_i} \cup (\fn{\aK_i}\setminus\dn{\apref_i})
 \\
  \fn{\aK_1; \aK_2} 
  &
  \fn{\aK_1}\cup\fn{\aK_2}
  \\
  \fn X & \emptyset
  \\
  \fn \arec & \fn {\aK}
\end{array}
\qquad
\begin{array}{l@{\ =\ } ll}
 \dn{\apref[\arole][@][@][\alocvar]} 
 & \dn \atuple 
 \\
 \multicolumn{3}{c}{\hspace{1cm}\ldots}
 \\ 
 \dn \asum & \bigcup_{i\in I} \dn{\apref_i} \cup \dn{\aK_i}
 \\
  \dn{\aK_1; \aK_2} 
  &
  \dn{\aK_1}\cup\dn{\aK_2}
  \\
  \dn X & \emptyset
  \\
  \dn \arec & \dn {\aK}
\end{array}
\end{array}
\]

We write  $\names \_$ for the set of sorted names of a term, i.e., 
 $\names \apref = \fn\apref \cup \dn\apref$ and similarly
  $\names \aK = \fn\aK \cup \dn\aK$. A set $S$ of sorted names 
  is consistent, written $\consistent S$, if 
  $x\mapsto \asort \in S$ and $x\mapsto \asort' \in S$ 
  implies  $\asort = \asort'$.
 
The set of well-sorted terms 
are defined inductively as follows:

\begin{itemize}
\item $\apref$ is well-sorted if $\fn\apref\cap\dn\apref = \emptyset$ and  
$\consistent{\names\apref}$, i.e., there are no clashes/inconsistencies in the sorts of 
the names in $\atuple$ and the locality $\alocvar$.

\item 
$\asum$ is well-sorted if for all ${i\in I}$ both ${\apref_i}$ and ${\aK_i}$ are
well-sorted and $\consistent{\names {\apref_i.\aK_i}}$.
 
\item 
$\aK_1;\aK_2$ is well-sorted if $\aK_1$ and $\aK_2$ are 
well-sorted and $\consistent{\names{\aK_1;\aK_2}}$.

\item $X$ is well-sorted.

\item $\arec$ is well-sorted if $\aK$ is well-sorted.

\end{itemize}


We consider terms up-to $\alpha$-renaming of defined names. As usual we 
say that a global type $\aK$ is {\em closed} when it does not contain free occurrences of 
variables, i.e., $\supp{\fn\aK} \cap \varset = \emptyset$. 


\begin{example}
Consider the following global type that describes the interaction of a client $\ptp c$
with a simple service $\ptp s$ that converts integers into strings. 
\[
 \aK_1 =
    \apref[\ptp c][\ptp s][{\asort[int]}][{\aloc[l]}].    
    \apref[\ptp s][\ptp c][{\asort[str]}][{\aloc[l]}].\nil
\]
The client $\ptp c$ produces a integer value on 
on the locality $\aloc$, which must be consumed by the server $\ptp b$, which 
produces back with the corresponding string for the client. 

Assume now that we consider  client and server as 
multiple roles instead of single participants, and write
\[
 \aK_2 =
    \apref[\amulti {C}][\amulti {S}][{\asort[int]}][{\aloc[l]}].
    \apref[\amulti {S}][\amulti {C}][{\asort[str]}][{\aloc[l]}].\nil
\]
In this case, $\aK_2$ states that each integer produced by 
a client will be consumed by a server, which will in turn produce a 
string for one of the clients. However, the type doesn't ensure
that a client will consume the value sent by the corresponding server. 
Variables can be used to correlate tuples. For instance,
\[
 \aK_3 =
    \apref[\amulti {C}][\amulti {S}][{\nu x:\asort[int]}][{\aloc[l]}].
    \apref[\amulti {S}][\amulti {C}][{x : {\asort[int]} \cdot {\asort[str]} }][{\aloc[l]}].\nil
\]
$\aK_3$ associates a fresh identifier $x$ to each value produced by a 
 client and consumed by a server. Despite the identifier is  known only to 
 the communicating instances,  this does not forbid two clients to 
 generate the same integer value.  Basically, the name $x$ allows to 
 express constraints about the flow of values. In particular,  $x : \asort[int]$ 
in the second interaction states that a server must generate a tuple 
that contains the consumed integer. Analogously, each 
client must consume a tuple matching  
the produced integer. We remark that this does not establishes
a one-to-one association between instances of $\amulti C$ and $\amulti S$.
 In fact, an instance of $\amulti C$ not necessarily interacts with the 
 same instance of $\amulti S$ in the two communications when two instances of 
 $\amulti C$ generate the same intenger in the first interaction. 
 
A one-to-one correspondence can be achieved
by using  fresh localities, e.g..

\[
 \aK_4 = 
    \apref[\amulti {C}][\amulti {S}][{{\asort[int]} \cdot \nu x: {\asort[loc]}}][{\aloc[l]}].
    \apref[\amulti {S}][\amulti {C}][{{\asort[int]}}][{x}].\nil
\]

A client generates a fresh locality identified by $x$, which is then
used as the locality for the subsequent communication. Since the name 
$x$ is known only to the two communicating instances, the second 
interaction can only take place between the two instances that know the 
locality $x$.
\end{example}

\begin{example} I think we can also deal with situations like the following, in which 
A creates a private session for B and C. 

\[
  \begin{array}{l} 
    \apref[\amulti {A}][\amulti {B}][{\nu y: {\asort[loc]}}][{\aloc[l]}].
    \\
    \apref[\amulti {A}][\amulti {C}][{y : {\asort[loc]}}][{\aloc[l]}].
    \\
    \apref[\amulti {B}][\amulti {C}][\asort][{y}]. \nil
\end{array}
\]
\end{example}


\begin{example}[Auction]

\[
  \begin{array}{l}
  \apref[\ptp {broker}][\amulti {Seller}][{\atuple[makeOffer]}][{\aloc[s]}][.].
  \\  	
  \apref[\amulti {Seller}][\amulti {Buyer}][{{\asort[str]} \cdot {\asort[int]} \cdot {\nu \aloc}}][{\aloc[c]}][.].
  \\
  \arec[@][\left({
    	\begin{array}{l}
	\apref[][\amulti {Buyer}][{i:{\asort[str]} \cdot p:{\asort[int]} \cdot \aloc[o]}][{\aloc[c]}][].
	\\
	\apref[\amulti {Buyer}][\amulti {Buyer}][{i \cdot {\asort[booked]}}][{\aloc[c]}][.].
	\\
	\arec[Y][\left({
    	\begin{array}{l}
	\apref[\amulti {Buyer}][\amulti {Seller}][{i \cdot {\asort[offer]} \cdot {\asort[int]}}][{\aloc[o]}][.].
	\\
	\apref[][\amulti {Seller}][{i \cdot \asort[offer]} \cdot  {\asort[int]}][{\aloc[o]}][].
	\\
	\qquad 
	\apref[\amulti {Seller}][\amulti {Buyer}][{i \cdot \asort[sold]}][{\aloc[c]}][.].
	\\
	\qquad\apref[][\amulti {Buyer}][i \cdot {\asort[booked]}][{\aloc[c]}][].
	X
	\\
	\qquad
	+
	\\
	\qquad
	\apref[\amulti {Seller}][\amulti {Buyer}][{i \cdot {\asort[request]}  \cdot \asort[int]}][{\aloc[o]}][.]. 
	\\
	\qquad
	\apref[][\amulti {Buyer}][i \cdot {\asort[request]} \cdot {\asort[int]}][{\aloc[o]}][].
	\\
	\qquad\qquad	
	Y
	\\
	\qquad\qquad	
	+
	\\ 
	\qquad\qquad\apref[][\amulti {Buyer}][i \cdot {\asort[booked]}][{\aloc[c]}][].
	\\
	\qquad\qquad
	\apref[\amulti {Buyer}][\amulti {Seller}][{i\cdot {\asort[quit]}}][{\aloc[o]}][.]. 
	\\
	\qquad\qquad
	\apref[\amulti {Seller}][\amulti {Buyer}][{i \cdot p \cdot {\nu \aloc}}][{\aloc[c]}][.].X
	\\
	\qquad
	+
	\\
	\qquad
	\apref[\amulti {Seller}][\amulti {Buyer}][{i\cdot {\asort[refused]}}][{\aloc[o]}][.]. 
	\\
	\qquad
	\apref[\amulti {Seller}][\amulti {Buyer}][{i \cdot p \cdot {\nu \aloc}}][{\aloc[c]}][.].
	\\
	\qquad
	\apref[][\amulti {Buyer}][{i \cdot \asort[refused]}][{\aloc[o]}][].
	\\
	\qquad\apref[][\amulti {Buyer}][i \cdot {\asort[booked]}][{\aloc[c]}][].
	X
	\end{array}
	}\right)]
	\\
	+
	\\
	\apref[][\amulti {Buyer}][{{\asort[str]} \cdot {\asort[booked]}}][{\aloc[c]}][.]. 
	X
	\\
	+
	\\
	\apref[][\amulti {Buyer}][{{\asort[str]} \cdot {\asort[sold]}}][{\aloc[c]}][.]. 
	X
	\\
	+
	\\
	\nil
  	\end{array}
  }\right)]
  \\
  \end{array}
\]
\finex
\end{example}

\eMnote{un'alternativa}
\begin{example}[Auction]
\[
  \begin{array}{l}
    \apref[\ptp {broker}][\amulti {Seller}][{\atuple[makeOffer]}][{\aloc[s]}][.].
    \\  	
    \arec[X][\left({
    \begin{array}{l}
      \apref[\amulti{Seller}][\amulti {Buyer}][s:{\asort[seller]} \cdot {i:{\asort[str]} \cdot p:{\asort[int]}}][{\aloc[m]}].
      \\\qquad
      X
      \\
      \chop
      \\
      \apref[\amulti {Buyer}][\amulti {Seller}][s \cdot i \cdot {\asort[offer]} \cdot \nu \aloc][{\aloc[m]}][.].
      \\\qquad
      \arec[Y][\left({
      \begin{array}{l}
	\apref[\amulti {Seller}][\amulti {Buyer}][{\asort[quit]}][\aloc].
	\\\qquad
	\apref[\amulti {Seller}][][s \cdot i \cdot p][{\aloc[m]}].
        \\
        \qquad
        X
        \\
        \chop
	\\
        \apref[\amulti{Seller}][\amulti {Buyer}][{\asort[sold]}][\aloc].
        \\\qquad
        \apref[\amulti{Seller}][\ptp{broker}][s \cdot i \cdot \aloc \cdot {\asort[fee]}][{\aloc[s]}].        
        \\
        \qquad
        X
	\\
        \chop
	\\
        \apref[\amulti {Seller}][\amulti {Buyer}][{\asort[more]}][\aloc].
        \\
        \qquad\apref[\amulti {Buyer}][\amulti {Seller}][{\asort[offer]}][{\aloc}][].
	\\
        \qquad	
	Y
      \end{array}
      } \right)]
    \end{array}
    }\right)]
    \\
  \end{array}
\]
\finex
\end{example}

TBC: 

Well-formedness (TBC): Besides the usual conditions about single selector and knowledge of 
choices,  we assume the following conditions hold in any $\aK$.

\begin{itemize}
% \item $\roles {\aK_1} \cap \roles {\aK_2} = \emptyset$ if $\aK = \aK_1 \parop \aK_2$
\item $\neg (\arole \eqR \arole')$ in $\apref[\arole][\arole']$ and
  $\apref[\arole][\arole'][@][\aloc][.]$.
  %
  \todo[inline]{eM: does the restriction above apply only to branches? May be it can be relaxed under the new semantic interpretation...}
\item the prefixes of each branch must generate
  \todo[inline]{eM: may be not}
\end{itemize}
%
\todo[inline]{eM:
  To ensure uniquess of selector in choices we can syntactically restrict the branching operator as
  \[
    \asum[@][@][{\apref[\arole][\arole'_i][\atuple_i]}]
  \]
  A simple way to ensure knowledge of choice is to require that the
  first input actions of passive roles in the branches are
  ``disjoint'' (ie non matching tuples or different locations).
  %
  This is just for simplicity as we could adopt definitions similar
  to the ones in \cite{gt16,gt17}.
  %
  % The notion of well formedness seems to guarantee a much weaker
  % notion of correctness than the usual ones.
  % %
  % Firstly, note that well-formedness here does not imply deadlock
  % freedom (but this is fine since we are not interested in properties
  % of the control of processes).
  % %
  A problem to consider is that, even assuming unique selector, many
  instances of the selector role could exercise choices concurrently.
  %
  This may create confusion if different branches generate matching
  tuples on the a locality.
  %
  % A example could be the following:
  \[
    K_\mathrm{bad} = \apref[A][B][int].K_1 \chop \apref[A][B][str].K_2
  \]
  \[
    K_1 = \apref[B][C][str].\apref[C][B][bool]
  \]
  \[
    K_2 = \apref[B][C][bool]
  \]
  This type of confusion does not seem to introduce deadlocks, but may
  alter the intended data flow (in the example above the instance of
  $C$ executing $K_2$ branch may receive the boolean that the instance
  of $C$ in $K_1$ generates for $B$).
  %
  We can tackle this issue in two different way (at least): one way is
  to statically ensure that tuples generated on one branch do not
  match any other on another branch; another way is to modify the
  semantics of the choice by implicitly inserting an extra field in
  each tuple with a unique identifier of each branch.
  %
  \newline
  %
  Under the current interpretation of our semantics, probably the
  notion of correctness we can guarantee is that any set of instances
  taking a choice will fully execute a branch.
  %
}

%%% Local Variables:
%%% mode: latex
%%% TeX-master: "main"
%%% End:


\subsection{Semantics of global types}
\label{sec:globsem}
% !TEX root =  main.tex

We give semantics to global types using \emph{partially-ordered
  multi-sets} (pomsets for short).
%
Following~\cite{gaifman1987partial}, a \emph{pomset} is an isomorphism
class of labelled partially-ordered sets (lposet) where, fixed a set
of labels $\lset$,
\begin{itemize}
\item an lposet is a triple $(\eset, \leq, \alf)$, with $\eset$ a set
  of events, $\leq$ is a partial order on $\eset$, and
  $\alf: \eset \rightarrow \lset$ a labelling function mapping events
  in $\eset$ to labels in $\lset$;
\item two lposets $(\eset, \leq, \alf)$ and $(\eset', \leq', \alf')$
  are \emph{isomorphic} if there is a bijection
  $\phi: \eset \rightarrow \eset'$ such that
  $\ae \leq \ae' \iff \phi(\ae) \leq' \phi(\ae')$ and
  $\alf = \alf' \circ \phi$.
\end{itemize}
%
Intuitively, the partial order $\leq$ yields a causality relation
among events; for $\ae \neq \ae'$, if $\ae \leq \ae'$ then $\ae'$ is
caused by $\ae$ or, in other words, the occurrence of $\ae'$ must be
preceded by the one of $\ae$ in any execution respecting the order
$\leq$.
%
Note that $\alf$ is not required to be injective: for
$\ae \neq \ae' \in \eset$, $\alf(\ae) = \alf(\ae')$ means that $\ae$
and $\ae'$ model different occurrences of the same action.
%
In the following, $[\eset, \leq, \alf]$ denotes the isomorphism class
of $(\eset, \leq, \alf)$, symbols $\apom,\apom', \dots$ (resp.
$\aR, \aR', \dots$) range over (resp. sets of) pomsets, and we assume
that pomset $\apom$ contains at least one lposet which will possibly
be referred to as $(\esetof \apom$, $\leqof \apom, \alfof \apom)$.
%
The empty pomset is denoted as $\emptypom$.

An event $\ae$ is an \emph{immediate predecessor} of an event $\ae'$
(or equivalently $\ae'$ is an \emph{immediate successor} of $\ae$) in
a pomset $\apom$ if $\ae \neq \ae'$, $\ae \leqof \apom \ae'$, and for
all $\ae'' \in \esetof \apom$ such that
$\ae \leqof \apom \ae'' \leqof \apom \ae'$ either $\ae = \ae''$ or
$\ae' = \ae''$.
%
We will represent pomsets as (a variant\footnote{Edges of Hasse
  diagrams are usually not oriented; here we use arrows so to draw
  order relations between events also horizontally.} of) Hasse
diagrams of the immediate predecessor relation; for instance, the
pomset
%\todo{{RB: anche $(\ae_1,\ae_5)$ nel pomset (non nelle figure ovviamente)?}}
\[
  \left[\{\ae_1,\ae_2,\ae_3,\ae_4,\ae_5\}, \{(\ae_1,\ae_2),(\ae_1,\ae_3),
    (\ae_1,\ae_4), (\ae_1,\ae_5), (\ae_4,\ae_5)\},
    \alf% :
    % \begin{cases}
    %   \ae_1 \mapsto \apref[\ptp p][]
    %   \\
    %   \ae_2,\ae_3 \mapsto \apref[][{\amulti q}]
    %   \\
    %   \ae_4 \mapsto \apref[\amulti R][][@][@][.]
    %   \\
    %   \ae_5 \mapsto \apref[][\amulti R][@][@][.]
    % \end{cases}
  \right]
\]
is more conveniently written as
\[
  \pomsetrep{
    \node (out) {$\ae_1$};
    \node[below left = of out] (in1) {$\ae_2$};
    \node[below right = of out] (in2) {$\ae_3$};
    \node[above right = of in2] (out2) {$\ae_4$};
    \node[below = of out2] (in3) {$\ae_5$};
    \draw[->] (out) -- (in1);
    \draw[->] (out) -- (in2);
    \draw[->] (out) -- (out2);
    \draw[->] (out2) -- (in3);
  }[\alf][][node distance = .5cm and .3cm]
  \qquad\text{or}\qquad
  \pomsetrep{
    \node (out) {$\alf(\ae_1)$}; % {$\apref[{\ptp p}][]$};
    \node[below left = of out] (in1) {$\alf(\ae_2)$}; % {$\apref[][{\amulti q}]$};
    \node[below right = of out] (in2) {$\alf(\ae_3)$}; % {$\apref[][{\amulti q}]$};
    \node[above right = of in2] (out2) {$\alf(\ae_4)$}; % {$\apref[\amulti R][][@][@][.]$};
    \node[below = of out2] (in3) {$\alf(\ae_5)$}; % {$\apref[][\amulti R][@][@][.]$};
    \draw[->] (out) -- (in1);
    \draw[->] (out) -- (in2);
    \draw[->] (out) -- (out2);
    \draw[->] (out2) -- (in3);
  }[][][node distance = .5cm and .1cm]
\]

The labels of the events in our semantics are (decorations of)
autonomous prefixes $\apref$: labels are either just autonomous
prefixes $\apref$ or of the form $\alabel$ with $\apref$ an autonomous
prefix.
%
Intuitively, a label $\alabel[@][][\arole][@][@][.]$
(resp. $\alabel[@][\arole][][@][@][.]$) represents the fact that the
$i^\mathit{th}$ instance of $\arole$ reads (resp. produces) a tuple of
type $\atuple$.
%
Labels $\apref$ not prefixed with $[\_]$ simply specify that the event
can be performed by any instance of the role in $\apref$.
% 
Hereafter, we only deal with pomsets labelled by a decorated
autonomous prefix as described above.
  
% Given an autonomous prefix $\apref$, we let $\pomsetsingle$ to be
% the pomset with a single event labelled by $\apref$.
%
%We will use the auxiliary operations on pomsets described below.
%
% Define the (disjoint) union of two pomsets $\apom$ and $\apom'$ as
% \[
%   \apom \pomsetcup \apom' =
%   [\esetof{\apom} \uplus \esetof{\apom'},
%   \leqof{\apom} \uplus \leqof{\apom'},
%   \alfof{\apom} \uplus \alfof{\apom'}]
% \]
%
The sequential composition of $\apom$ and $\apom'$ composes a
\quo{copy} of $\apom'$ with every maximal event of $\apom$.
%
Formally, for each $\ae \in \max \apom$, let
$\apom'_{\ae} = [\{\ae\} \times \eset_{\apom'}, \leq, \alf]$ where
$(\ae,\ae_1) \leq (\ae,\ae_2) \iff \ae_1 \leqof{\apom'} \ae_2$ and
$\alfof{\apom'_{\ae}} = (\ae,\ae') \mapsto \alfof{\apom'}(\ae')$ for
all $\ae' \in \esetof{\apom'}$.
%
\eMnote{{c'era qualcosa da fare su $\rseq[][]$...}}
Observe that $\apom'_{\ae}$ is isomorphic to $\apom'$, and define
\[
  \rseq[\apom][\apom'] = 
  [\esetof{\apom} \uplus \bigcup_{\ae \in \max \apom}\esetof{\apom'_{\ae}},
  \leq,
  \alfof{\apom} \uplus \bigcup_{\ae \in \max \apom}\alfof{\apom'_{\ae}}]
\]
where $\leq$ is the reflexive-transitive closure of
\todo{AC+RB: sembra che $\leqof{\apom'}$ non possa essere usato perch\'e riguarda gli elementi di $\eset_{\apom'}$ noo quelli delle sue copie. Questo vuol dire che difficilmente possiamo catturare il parallelo...}
\[
\leqof{\apom} \cup \leqof{\apom'} \cup
\bigcup_{\ae \in \max \apom}\leqof{\apom'_{\ae}}
\cup
\bigcup_{\scriptsize
  \begin{array}{c}
\ae \in
\esetof{\apom}
\\
(\ae_1,\ae_2) \in \esetof{\apom'_{\ae}}
  \end{array}
} \{(\ae,(\ae_1,\ae_2)) \sst 
\roles{\alfof{\apom}(\ae)} = \roles{\alfof{\apom'}(\ae_2)}\}
\]
(recall that the labels of events are autonomous prefixes for which
$\roles{}$ is a singleton).
\todo{AC: dove \`e scritto?}
%
\eMnote{{quando $\leq = \leqof{\apom} \uplus \leqof{\apom'}$ abbiamo il parallelo}}
%
% Finally, for $h \geq 1$ and a pomset $\apom$ we define
% \[
%   \apom^h =
%   \begin{cases}
%     r & \text{if } h = 1
%     \\
%     r \pomsetcup r^{h-1} & \text{if } h > 1
%   \end{cases}
% \]
% The operation $\_^h$ extends element-wise to sets of pomsets.

The semantics of a global type builds upon \emph{basic pomsets} $\bp$
of prefixes $\apref$; in fact, the semantics of a global type is built
by taking sets of disjoint unions of basic pomsets of prefixes
sequentially composed together.
%
The intended semantics of an autonomous output is that the generated
tuple type can be accessed by any instance of another role but this is
not compulsory.
%
Hence, a tuple generated with an autonomous output may also not be
accessed at all.
%
To capture this semantics for an autonomous prefix $\apref[@][@][][]$
we define $\bp = \left\{ \pomsetrep{\node {$\alabel$}}[] \right\}$.
%
The semantics interactions is slightly more subtle.
%
Firstly, tuple types generated in an interaction are meant to be
eventually accessed by the some instance of a role unless they are
read-only (in which case they can be also not accessed); secondly, for
consuming interactions we require that the tuple type is actually
consumed by an instance of the receiving role, but allow other an
arbitrary read-only access to such tuple types by other instance of
the receiving role.
%
Therefore, we allow $\asort$ to include element of $\roleset$ and for
$h \geq 1$ and $\arole,\arole' \in \roleset$, we let
\begin{align*}
  \bp[@][{\apref[\arole][\arole'][@][@][.]}] =
  &
    \begin{cases}
      \left\{
        \pomsetrep{
        \node (out) {$\alabel[@][\arole][][\arole' \cdot \atuple][@][.]$};
        \node[below = of out] (in) {$\alabel[@][][\arole'][\arole' \cdot \atuple][@][.]$};
        \draw[->] (out) -- (in);
      }[]\right\}
      &
      \text{if } \arole' \in \participants \text{ or $\atuple$ generates} 
      \\
      \bigcup_{h \geq 1}
      \left\{
        \pomsetrep{
          \node (out) {$\alabel[@][\arole][][\arole' \cdot \atuple][@][.]$};
          \node[below left = of out] (rd1) {$\ae_1$};
          \node[below = of out] (dots) {};
          \node[below right = of out] (rd2) {$\ae_h$};
          \draw[->] (out) -- (rd1);
          \draw[->] (out) -- (rd2);
          \draw[dotted] (rd1) -- (rd2);
        }[{\big(\scriptsize
        \begin{array}[c]{l}
          \ae_j \mapsto \\ \apref[][\arole'][\arole' \cdot \atuple][@][.]
        \end{array}
        \big)_{1 \leq j \leq h}}][][node distance = 1cm and -.25cm]\right\}
      \cup
      \left\{
        \pomsetrep{\node {$\alabel[@][\arole][][\arole' \cdot \atuple][@][.]$}}[]
      \right\}
    &
    \text{otherwise}
    \end{cases}
\end{align*}
%
and
%
\begin{align*}
  \bp[@][{\apref[\arole][\arole']}] =
  &
    \bigcup_{h \geq 1}\left\{
    \pomsetrep{
    \node (out) {$\alabel[@][\arole][][\arole' \cdot \atuple]$};
    \node[below left = of out] (rd1) {$\ae_1$};
    \node[below = of out] (dots) {};
    \node[below right = of out] (rd2) {$\ae_h$};
    \node[below = of dots] (in) {$\alabel[@][][\arole'][\arole' \cdot \atuple]$};
    \draw[->] (out) -- (rd1);
    \draw[->] (out) -- (rd2);
    \draw[->] (rd1) -- (in);
    \draw[->] (rd2) -- (in);
    \draw[dotted] (rd1) -- (rd2);
    }[(\ae_j \mapsto {\apref[][\arole'][\arole' \cdot \atuple][@][.]})_{1 \leq j \leq h}][][node distance = 1cm and -.25cm]\right\}
    \cup
    \left\{
    \pomsetrep{
    \node (out) {$\alabel[@][\arole][][\arole' \cdot \atuple]$};
    \node[below = of out] (in) {$\alabel[@][][\arole'][\arole' \cdot \atuple]$};
    \draw[->] (out) -- (in);
    }[]
    \right\}
\end{align*}
\eMnote{le label delle $\ae_i$ forse dovrebbero essere $\arole''$}

We first give semantics of prefixes.
\begin{align*}
  \ksem{\apref} =
  &
    \begin{cases}
      \bp[1][\apref]
      & \text{if $\apref$ autonomous } \land \roles{\apref} \subseteq \participants
      \\
      \bigcup_{i \geq 1}\bp[@][\apref]
      % \bigcup_{h \geq 1}\big\{\pomsetrep{
      % \node {$\alabel[1]\cdots\alabel[h]$};
      % }[]\big\},
      & \text{if $\apref$ autonomous } \land \roles{\apref} \not\subseteq \participants
    \end{cases}
%  \qquad \text{if $\apref$ autonomous}
  \\
  \ksem{\apref[\arole][\arole'][@][@][.]} =
  &
    \begin{cases}
      \bp[1][{\apref[\arole][\arole'][@][@][.]}]
      & \text{if } \arole \in \participants
      \\
      \bigcup_{i \geq 1}\bp[@][{\apref[\arole][\arole'][@][@][.]}]
      & \text{otherwise}
    \end{cases}
  \\
  \ksem{\apref[\arole][\arole']} =
  &
    \begin{cases}
      \bp[1][{\apref[\arole][\arole']}]
      & \text{if } \arole \in \participants
      \\
      \bigcup_{i \geq 1}\big(\bp[@][{\apref[\arole][\arole']}]\big)
      & \text{otherwise}
    \end{cases}
\end{align*}

We can now give the semantics of (closed) global types.
%
As customary in other choreographic approaches, we define the
semantics of \emph{well-formed} global views of choreographies.
%
The well-formedness of our global views presents some peculiarities
respect to standard notions.
%
In particular, \emph{well-sequencedness} and \emph{well-branchedness}
of our global types need some care.

We tackle well-sequencedness first.
%
Consider the following choreography
\begin{align}\label{eq:seq}
  \apref[\arole_1][\arole_2][{\asort[str] \cdot \wildcard}][\aloc] \seqop
  \apref[\arole_2][\arole_3][{\asort[str] \cdot \asort[int]}][\aloc]
\end{align}
%
where an instance of $\arole_2$ transforms a pair generated by
$\arole_1$ into a pair for $\arole_3$.
%
The choreography specified in \eqref{eq:seq} may be violated when
$\arole_1$ generates a tuple of type $\asort[str] \cdot \asort[int]$.
%
In fact, such a tuple could match the type consumed by $\arole_3$ and
therefore $\arole_3$ could \quo{steal} the tuple from $\arole_2$.
%
The problem is due to the fact that ($i$) the input could match the
tuple sent by $\arole_1$ and ($ii$) $\arole_3$ can consume the tuple
sent by $\arole_1$ because it is on the same locality $\aloc$.
%
More generally, write $(\atuple, \aloc) \in \aK$ when there is a
prefix in $\aK$ whose tuple is $\atuple$ and whose location is
$\aloc$; we say that $(\atuple, \aloc)$ is \emph{local} to $\aK$ if
either of the following holds:
\begin{itemize}
\item $\aK = \asum[@][@][][]$ there is $i \in I$ such that either
  $(\atuple, \aloc)$ is local to $\aK_i$ or $\apref_i$ outputs
  $\atuple$ at $\aloc$ and there is $\atuple'$ in an input from
  $\aloc$ in $\aK$
\todo{AC+RB: $\aK$ o $\aK_i$?}
   and $\atuple \matches \atuple'$
\item $\aK = \aK_1 \seqop \aK_2$ either $(\atuple, \aloc)$ is local to
  $\aK_1$ or   $(\atuple, \aloc)$ is local to $\aK_2$
\item $\aK = \grec[@][@][@][\aK']$ and $(\atuple, \aloc)$ is local to $\aK'$
\todo{AC+RB: sintassi della ricorsione cambiata? A cosa serve $\aK'$?}
\end{itemize}
%
A choreography $\aK_1 \seqop \aK_2$ is \emph{conflict-free} if for
$i \neq j \in \{1,2\}$
%
\begin{itemize}
\item for all $(\atuple,\aloc)$ local to $\aK_i$ and for all
  $(\atuple', \aloc) \in \aK_j$, $\atuple \matches \atuple'$ implies
  $(\atuple', \aloc)$ is in a read-only prefix in $\aK_j$
\item for all $(\atuple,\aloc)$ in a non-local consuming-input prefix
  of $\aK_i$ and for all $(\atuple', \aloc)$ in an output prefix of
  $\aK_j$, $\atuple \matches \atuple'$ implies $(\atuple',\aloc)$ is
  in a consuming output prefix in $\aK_j$.
\end{itemize}
% such that $\atuple_i$ occurs in a prefix of $\aK_i$ both at a locality $\aloc$, \eMnote{Problema: $\aloc$ vs $\alocvar$}

Finally, we give the semantic equations of global types.
%
For the recursive global type $\grec$ we introduce some auxiliary
functions:
\begin{align*}
  \mathit{STOP}(\arole, \aK, \seq y) = & \apref[\arole][\arole_1][{\asort[stop]}][y_1] \seqop \ldots \seqop \apref[\arole][\arole_n][{\asort[stop]}][y_n]
  \\
  \mathit{LOOP}(\arole, \aK, \seq y, \seq y') = & \apref[\arole][\arole_1][{\nu y'_1:\asort[loc]}][y_1)] \seqop \ldots \seqop \apref[\arole][\arole_n][{\nu y'_n:\asort[loc]}][y_n]
\end{align*}
where $\roles \aK = \{\arole, \arole_1, \ldots, \arole_n\}$ with
$\arole \not\in \{\arole_1, \ldots, \arole_n\}$ and
$\seq y = y_1 \cdots y_n$ and 
$\seq y' = y'_1 \cdots y'_n$.
%
Then, we define
\begin{align*}
  \ksem{\grec} =
  &
    \begin{cases}
      \bigcup_{h \geq 0} \ksem{\unfold h \grec {\fn{\aK}} {\seq y}  {\seq y'}} & \text{if } \ws[\aK\sust X \nil][\aK\sust X \nil]
      \\ & \text{ and } \seq y \cap \fn{\aK} = \emptyset
      \\
      \mathit{undefined} & \text{otherwise}
    \end{cases}
\end{align*}
where
\[
  \unfold h \grec L {\seq y} {\seq y'} =
  \begin{cases}
    \mathit{STOP}(\arole,\seq y) & \text{if } h = 0
    \\
    \mathit{LOOP}(\arole,\aK, \seq y, \seq y') \seqop \aK \sust X {\aK'} & \text{otherwise}
  \end{cases}
\]
%
where $\aK' = \unfold{h-1}{\grec}{L \cup \seq y \cup \seq y'}{\seq y'}{\seq y''}$
with $\seq y''$ fresh.

The remaining equations are
%
\begin{align*}
  % 
  \ksem{\asum[@][@][][]} =
  &
    \begin{cases}
      \{ \emptypom \} & \text{if } I = \emptyset
      \\
      \bigcup_{\apom \in \ksem{\apref_i},\apom' \in \ksem{\aK_i}} \rseq[\apom][\apom']
      & \wb[{\big\{\bigcup_{i \in I}\apref_i\mkop{.}\aK_i\big\}}]
      \\
      \mathit{undefined} & \text{otherwise}
    \end{cases}
  \\
  \ksem{\aK_1 \seqop \aK_2} =
  &
    \begin{cases}
      \rseq[\ksem{\aK_1}][\ksem{\aK_2}] & \text{if } \ws[\aK_1][\aK_2]
      \\
      \mathit{undefined} & \text{otherwise}
    \end{cases}
\end{align*}
where predicates $\wb[]$ and $\ws[][]$ respectively check the
\emph{well-branchedness} and \emph{well-sequencedness} conditions.
%
The former is defined below whereas $\ws[\aK_1][\aK_2]$ holds when
$\aK_1 \seqop \aK_2$ is conflict-free and any participant $\ptp a$
performing actions in $\aK_2$ acts in $\aK_1$ as well and all the
first actions of $\ptp a$ in $\aK_2$ causally depend on all last
actions of $\ptp a$ in $\aK_1$ in the pomset
$\rseq[\ksem{\aK_1}][{\ksem{\aK_2}}]$.

% To avoid this problem we introduce \emph{run-time} tuples
We now consider well-branchedness, the other condition of
well-formedness.
%
Well-branchedness requires two conditions: single selector and knowledge of 
choices.
%
This can be formalised by requiring that one process in the choice is
\emph{active}, namely it selects the branch to take, while the others
are \emph{passive}, namely they are informed of the chosed branch by
inputting some information that unambiguously identifies each branch
of the choice.
%
%
We syntactically enforce uniqueness of selectors; a choice with
several branches, takes the form
\begin{align}
  \asum\label{eq:ch}
\end{align}
namely the instance of $\arole$ act a \emph{unique} selectors.
%
This is just for simplicity as we could adopt definitions similar to
the ones in \cite{gt16,gt17} at the cost of higher technical
complexity.

Intuitively, a passive instance (for example one enacting role
$\arole_i$) in \eqref{eq:ch} has to be able to ascertain which branch
the selector decided when the choice was taken.
%
A simple way to ensure this is to require that the first input actions
of their roles in the branches are \quo{disjoint} (\ie\ non matching
tuples or different locations).
%
This is just for simplicity as we could adopt more general definitions
similar to the one based on divergence points in \cite{gt16,gt17}.

The conditions on active and passive processes alone are not enough:
in our framework, the notion of well-branchedness is slightly
complicated by the presence of multiroles.
%
For instance, even assuming unique selectors, many instances of a
selector role could exercise choices concurrently.
%
This may create confusion if different branches generate matching
tuples on a locality as illustrated by next example.
\begin{example}\label{ex:nonwb}
  Let
  \begin{align*}
    \aK_\mathrm{bad} = & \apref[\amulti A][\amulti B][{\asort[int]}][\aloc] \mkop{.} K_1 \chop \apref[\amulti A][\amulti B][{\asort[str]}][\aloc] \mkop{.} K_2
    \\
    \aK_1 = & \apref[\amulti B][\amulti C][{\asort[str]}][\aloc] \mkop{.} \apref[\amulti C][\amulti B][{\asort[bool]}][\aloc]
    \\
    \aK_2 = & \apref[\amulti B][\amulti C][{\asort[bool]}][\aloc]
  \end{align*}
  In $\aK_\mathrm{bad}$ confusion may arise that may alter the
  intended data flow when two groups of instances of $\amulti A$,
  $\amulti B$, and $\amulti C$ execute the choice so that each group
  takes a different branch.
  %
  In fact, the instance of $\amulti C$ executing $K_2$ in the second
  branch may receive the boolean that the instance of $\amulti C$ in
  $K_1$ executing the first branch generates for $\amulti B$.
  %
  \eMnote{This type of confusion does not seem to introduce deadlocks}
  %
  \finex
\end{example}
%
Therefore we require that tuples types in different branches of a
choice do not match when they are at the same locality and that if a
branch of a choice involves a participant then none of the branches of
the choice involves multiroles.
%
This condition, dubbed \emph{confusion-free branching} ensures that
different \quo{groups} of instances involved in concurrent resolutions
of a choice do not \quo{interfere} with each other and such groups can
execute the choice many times only if none of the processes implements
a participant.
%
We remark that the above condition is not a limitation; in fact, we
can pre-process branches of choices by adding an extra field in all
tuples of the branch so to unequivocally identify on which branch the
tuple type is used.

To sum up, a choice as in~\eqref{eq:ch} is \emph{well-branched},
written $\wb[{\big\{\bigcup_{i \in I}\apref_i\mkop{.}\aK_i\big\}}]$,
when it is confusion free, there is a unique active role, all other
roles are passive.


  %
% The notion of well formedness seems to guarantee a much weaker
% notion of correctness than the usual ones.
% %
% Firstly, note that well-formedness here does not imply deadlock
% freedom (but this is fine since we are not interested in properties
% of the control of processes).
% %
%
% A example could be the following:
%
%
\eMnote{We can tackle this issue in two different way (at least): one way is
to statically ensure that tuples generated on one branch do not match
any other tuple on another branch; another way is to modify the
semantics of the choice by implicitly inserting an extra field in each
tuple with a unique identifier of each branch.
\\
Under the current interpretation of our semantics, probably the notion
of correctness we can guarantee is that any set of instances taking a
choice will fully execute a branch.
}




%%% Local Variables:
%%% mode: latex
%%% TeX-master: "main"
%%% End:


\subsection{Local types}
\label{sec:locsem}
% !TEX root =  main.tex
%

A {\em local type} $\aL$, which describes the interaction from the
perspective of a single role, is a term generated by the following
grammar.
%
\todo{RB: per i tipi locali $\seqop$ \`e davvero la composizione
  sequenziale: meglio usare un simbolo diverso, ad esempio ;? Notare
  che poi nella definizione di $\proj$ (caso 7) si usa ;
}
\todo{eM \& HM: ci avevamo pensato, ma poi abbiamo optato per consistenza; se volete cambiate pure.}
\begin{eqnarray*}
  \aLpref & \bnfdef &
%                  \aout[] \bnfmid
                  \arout[]\bnfmid
                  \ain[] \bnfmid
                  \ard[] 
\\
  \aL & \bnfdef &
                  \aLsum \bnfmid
                  \aL \seqop \aL \bnfmid
                   \alrec \bnfmid
  	        \alvar%[{\alvar[@][x_1,\ldots,x_n]}][\aL]
\end{eqnarray*}
%\todo{RB: nella formula sopra compare un carattere strano che mi causa errore di compilazione: l'ho messo tra parentesi, si pu\`o togliere?, HM: adesso?}

Prefixes $\arout[]$, $\ain[]$ and $\ard[]$ respectively stand for the
production, consumption and read of a tuple $\atuple$ at the locality
$\alocvar$. Differently from global types, local types do not
distinguish the generation of read-only tuples from the ones that can
be consumed.

Formation rules for branching and sequential local types $\aL$ are
exactly the same as for global types; analogously we write $\nil$ for
an empty sum.  The syntax of recursive local types deviates from
global types to make explicit the localities used for coordinating the
execution; consequently, process variables are parametric (the syntax
for recursive types is borrowed from~\cite{bhty10}).  The term $\alrec$
defines a process variable $X$ with parameters $\seq x$ to be used in
$\aL$; the initial values of $\seq x$ are given by
$\seq\alocvar$. Accordingly, the usage of a process variable is
parameterised, i.e., $\alvar$.  For any $\alrec$, we assume that
$|\seq x| = |\seq \alocvar|$ and $|\seq x| = |\seq {\alocvar'}|$ for
any bound occurrence of $\alvar[@][\alocvar']$ in $\aL$.


The notions of free and defined names, well-sorted and closed terms
are straightforwardly extended to local types; in $\alrec$, $X$ and
$\seq x$ act as binders for the occurrence in $\aL$.
%
%
%The definition of $\fn\_$ is extended to local types 
%\begin{eqnarray*}
%  \fn {\aout[]}  & = & \fn {\arout[]} = \fn {\ain[]} = \fn {\ard[]} = \fn {\atuple} \cup \{\aloc\}  
%  \\
%  \fn {\aLsum} & = & \bigcup_{i\in I} \fn{\aLpref_i}\cup (\fn {\aL_i}\setminus \dn{\aLpref_i}) 
%  \\
%  \fn {X} & = & \emptyset 
%  \\
%  \fn {\arec[@][\aL]} & = & \fn {\aL} 
%\end{eqnarray*}
%
Substitution on local types is defined as follows.
% 
% \todo{RB: ma nella somma la sostituzione non si applica al prefisso $\aLpref_{i}$? Oppure il termine nella sommatoria dovrebbe leggersi $({\aLpref_{i}.\aL_{i}})\sust x y$? (ma la sostituzione non \`e definita per il prefisso)}
% \todo{eM: check now; a me sembra che la sostituzione non \`e definita per il prefisso (prime 3 equaz.)}
% \todo{RB: cos\`\i\ ok per me}
 \[
\begin{array}{rl@{\hspace{.3cm}}l}
  (\arout[]) \sust x y  = &  \arout[][\atuple\sust x y][(\alocvar\sust x y)] &  {\it if}\  x\not\in\dn\atuple
  \\
  (\ain[]) \sust x y  = &  \ain[][\atuple\sust x y][(\alocvar\sust x y)] &  {\it if}\  x\not\in\dn\atuple
  \\
  (\ard[]) \sust x y  = &  \ard[][\atuple\sust x y][(\alocvar\sust x y)] &  {\it if}\  x\not\in\dn\atuple
%  \\ 
%  \fn {\arout[]} = & 
%  \\
%  \fn {\ain[]}  = &
%  \\
%  \fn {\ard[]}  = &
  \\
   (\aLsum)\sust x y  = & \displaystyle{\sum_{i\in I} {(\aLpref_{i}\sust x y).(\aL_{i}\sust x y)}}
   &  {\it if}\  \forall i. x\not\in\dn{\aLpref_i}
  \\
  (\aL_1\seqop\aL_2)\sust x y  = &  \aL_1\sust x y\seqop  \aL_2\sust x y 
  \\
  \alvar\sust x y  = & \alvar[@][\seq\alocvar\sust x y]  
  \\
  (\alrec[@][\seq z])\sust x y  = & \alrec[@][\seq z][\aL\sust x y][\seq\alocvar\sust x y] &  {\it if}\  \{x,y\}\cap\seq z = \emptyset
\end{array}
\]
%
%
%Let $s$ and $t$ in $\locset$  local types up-to $\alpha$-renaming of binders. 
%
%
%For a local type $\aL = \aLsum$, we write $\selectors \aL$ for the selecting roles the choice, i.e.,
%$\selectors \aL = \cup_{i\in I}{\{\arole_i\}}$.
%%
%Moreover, we assume
%\begin{itemize}
%\item $\aL_i \neq\aLsum[j][J]$, for all $i \in I$.
%\item $\arole_i \eqR \arole_j$ for all $i,j \in I$.
%\end{itemize}
%
As for global types, we consider terms up-to $\alpha$-renaming.
 
We consider the following syntax for the run-time semantics of a set
of local types running on a tuple space, dubbed \emph{specification}.
%
\begin{eqnarray*}
  \envmv & \bnfdef & \emptyset \bnfmid
                  \envmv, \arole :  \aL \bnfmid
%                  \envmv, \atupleat \bnfmid
                  \envmv, \artupleat[@][@][\aloc]                  
\end{eqnarray*}
%
% \todo{eM: $\alocvar$ non dovrebbe comparire nelle tuple locate delle specifiche}
% \todo{RB: credo che possa succedere: solo a livello delle etichette delle transizioni le variabili sono istanziate a locazioni}
%
A specification is a multiset containing two kind of pairs:
$\arole : \aL$ associates a role with a local type; while $\artupleat[@][@][\aloc]$
indicates that a tuple of type $\atuple$ is available at locality
$\aloc$.
%In $\arole :  \aL$  we assume
%$\arole \in \participants \cup \multiroles$  and $\arole \eqR \arole'$ for any $\arole'$
%occurring in $\aL$.
%
We sometimes write $\envtuple$ to denote a specification containing only
terms of the form $\artupleat[@][@][\aloc]$. 

The definition of $\fn{\_}$ is straightforwardly  extended to specifications. 

We give an operational semantics to local types defined inductively by 
the rules in \cref{fig:local-types-sem}, where labels $\alpha$ are of the form 
$\arole : \aLpref$. Rule $\rulename{LOut}$ 
accounts  for the behaviour of a role $\rho$ that  generates a tuple type $\atuple$ at the 
locality $\aloc$. The operational semantics for the generation of a tuple $\atuple$ that 
contains binders ensures that  each defined name  is substituted by a fresh free variable 
 (i.e., a variable that does not occur free in $\envmv, \arole : \arout[][@][\aloc].\aL$).
This is achieved by requiring  (i) all bound names in $\atuple$ to be fresh (i.e., $\dn\atuple$  fresh) and 
(ii)  the generated tuple $\eraseB\atuple$ is the binder-free version of $\atuple$. 
Rule $\rulename{LIn}$ handles the case in which a role $\arole$
consumes  a tuple specified as $\atuple$ from locality $\aloc$. In order for the consumption to
take place, the  requested tuple $\atuple$ should match  a tuple $\atuple'$ available 
at the locality $\aloc$. 
Note that the substitution $\sigma$ generated from the match is applied 
to the continuation $\aL$ associated with the role $\arole$; the
consumed tuple is eliminated from the locality $\aloc$. 
%
Rule $\rulename{LRd}$ is analogous to $\rulename{LIn}$, but the 
read tuple is not removed from the tuple space.
%
Rule $\rulename{LSum}$ accounts for a role that follows by choosing one of 
its enabled branches. 
%
The semantics of a recursive term $\alrec[@][@][@][\seq \aloc]$ is
given by the rule $\rulename{LRec}$, which unfolds the definition,
i.e., $\aL\sust X{\abasiclrec}$ and substitutes the formal parameters
$\seq x$ of the recursive definition by the actual parameters
$\seq\aloc$, i.e., it applies the substitution
$\sust {\seq x}{\seq \aloc}$.
 

%\eMnote{nelle etichette delle conclusioni di Lin e LRd non dovrebbe essere $\atuple$ invece di $\atuple'$?} 
\begin{figure}[t]
\[
\begin{array}{l@{\quad\quad}l}
%\mathrule{\dn\atuple\ {\it fresh}}
%	{\envmv, \arole : \aout[][@][\aloc].\aL 
%	 \red[\arole : {\aout[][\eraseB\atuple][\aloc]}]
%	 \envmv, \arole : \aL , \atupleat[@][\eraseB\atuple][\aloc]}
%	{LOut_1} 
%\\[25pt]
\mathrule{\dn\atuple\ {\it fresh}}
	{\envmv, \arole : \arout[][@][\aloc].\aL 
	 \red[\arole : {\arout[][\eraseB\atuple][\aloc]}]
	 \envmv, \arole : \aL , \artupleat[@][\eraseB\atuple][\aloc]}
	{LOut}
&
\mathrule
	{\atuple \matches \atuple' \generates \sigma}
	{\envmv, \arole : \ain[][@][\aloc].\aL, \artupleat[][\atuple'][\aloc]
	 \red[\arole : {\ain[][\atuple'][\aloc]}]
	 \envmv, \arole : \aL\sigma}
	{LIn}
\\[25pt]
%\mathrule
%	{\atuple \matches \atuple' \generates \sigma}
%	{\envmv, \arole : \ard[][@][\aloc].\aL, \atupleat[][\atuple'][\aloc] 
%	 \red[\arole : {\ard[][\atuple'\sigma][\aloc]}]
%	 \envmv, \arole : \aL\sigma,  \atupleat[][\atuple'][\aloc] }
%	{LRd_1}
%\\[25pt]
\mathrule
	{\atuple \matches \atuple'\generates \sigma}
	{\envmv, \arole : \ard[][@][\aloc].\aL, \artupleat[][\atuple'][\aloc] 
	 \red[\arole : {\ard[][\atuple'][\aloc]}]
	 \envmv, \arole : \aL\sigma,  \artupleat[][\atuple'][\aloc] }
	{LRd}
&
\mathrule
	{\envtuple, \arole : \aLpref_j.\aL_j \red[\alpha] \envmv'}
	{\envmv, \envtuple, \arole : \aLsum
	 \red[\alpha]
	 \envmv,\envmv'}
	{LSum} \qquad j\in I
\\[25pt]
\mathrule
	{\envmv, \arole : \aL_1  \red[\alpha] \envmv', \arole : \aL'_1 }
	{\envmv, \arole : \aL_1\seqop\aL_2
	 \red[\alpha]
	 \envmv', \arole : \aL'_1\seqop\aL_2}
	{LSeq_1} 
&
\mathrule
	{\envmv, \arole : \aL_1  \red[\alpha] \envmv', \arole : \nil }
	{\envmv, \arole : \aL_1\seqop\aL_2
	 \red[\alpha]
	 \envmv', \arole : \aL_2}
	{LSeq_2} 
  \\[25pt]
  \multicolumn 2 c {
  \mathrule
  {\envmv, \arole : \aL\sust X{\abasiclrec}\sust {\seq x}{\seq \aloc} \red[\alpha] \envmv'}
  {\envmv, \arole : \alrec[@][@][@][\seq \aloc] \red[\alpha] \envmv'}
  {LRec}
  }
\end{array}
\]
\caption{Semantics of local types}
\label{fig:local-types-sem}
\end{figure}


\subsection{Obtaining local types out of global types}
\label{sec:proj}

The projection of a global type $\aK$ over a role $\arole$, 
written $\proj[@][@][\ ]$,  denotes the local type that specifies 
the behaviour of  $\arole$ in $\aK$. 
% 
Our projection operation is fairly standard but for the case of
recursive types, which coordinate their execution by communicating
over dedicated locations.
%
%
Note that the semantics of recursive global types $\grec$ introduces
auxiliary interactions to coordinate their execution (see
$\mathit{STOP}(\arole, \aK, \seq y)$ and
$\mathit{LOOP}(\arole, \aK, \seq y, \seq y')$ in \cref{sec:globsem}).
%
However, there is not such an implicit mechanism in the 
execution of local types, where recursion is standard. 
%
Consequently, those auxiliary interactions need to be defined
explicitly in local types; and consequently, they are introduced by
projection (similarly to the approach in~\cite{bmt14}).
%
Another subtle aspect of the semantics of a recursive global type is that 
each iteration  is parametric with respect to the set of localities used 
for coordination. In fact,  $\mathit{LOOP}(\arole, \aK, \seq y, \seq y')$
generates a set of fresh localities  that are 
used by the next iteration. Such behaviour is mimicked by local types by 
relying on parameterised process variables. As a consequence,
projection depends on the locations that are chosen as parameters  
of process variables. Hence, $\proj[@][@][\ ]$ is defined in terms 
of $\proj$, where $\eta$ is a partial function that maps 
process variables into sequences of locations, i.e., 
$\eta X = \seq\alocvar$; and $\proj[@][@][\ ] = \proj[@][@][\emptyset]$.
\begin{figure}[t]
 \[
  \proj =\left\{
  \begin{array}{l@{\quad}l@{\ \ }l}
    \nil & \text{if }\ \arole \not \in \roles \aK
    \\[10pt]
    \proj[\aK']
    &
    \multicolumn{2}{l}{
    \text{if }\ \aK = \apref.\aK' \text{ and } \rho\not\in\roles{\apref}}
%    \\
%    \aout[].(\proj[\aK'])
%    &
%    \text{if } \aK = \apref[\arole].\aK'
    \\[10pt]
    \arout[].(\proj[\aK'])
    &
    \text{if }\  \aK = \apref[\arole][{}][@][@][.].\aK'  
    & 
    \text{or }\ \aK = \apref[\arole][\arole'][@][\alocvar][.].\aK'
    \\
    &
    \text{or }\ \aK = \apref[\arole].\aK'    
    &
    \text{or }\ \aK = \apref[\arole][\arole'].\aK'  
%    \\
%    \ain[].(\proj[\aK'])
%    &
%    \text{if } \aK = \apref[{}][\arole].\aK'
    \\[10pt]
    \ain[].(\proj[\aK'])
    &
     \text{if }\  \aK = \apref[{}][\arole].\aK' 
     &
     \text{or }\ \aK = \apref[\arole'][\arole].\aK'
    \\[10pt]
    \ard[].(\proj[\aK'])
    &
    \text{if }\  \aK = \apref[{}][\arole][@][@][.].\aK' 
    &
    \text{or }\ \aK = \apref[\arole'][\arole][@][\alocvar][.].\aK'
    \\[10pt]
    \displaystyle{\sum_{i\in I}{\proj[(\apref_i.\aK_i)]}}
    &
     \multicolumn{2}{l}{
     \text{if }\ \aK = \asum[@][@][][]
     }
    \\[10pt]
    \proj[\aK_1];\proj[\aK_2] 
    & 
    \multicolumn{2}{l}{
     \text{if }\  \aK = \aK_1 \seqop \aK_2
     }
     \\[10pt]
%    \aout[].(\proj[\aK'])
%    &
%    \text{if } \aK = \apref[\arole][\arole'].\aK'
%    \\
%    \ain[].(\proj[\aK'])
%    &
%    \text{if } \aK = \apref[\arole'][\arole].\aK' \text{ or }
%    \\
%    \arout[].(\proj[\aK'])
%    &
%    \text{if } \aK = \apref[\arole][\arole'][@][\aloc][.].\aK'
%    \\
%    \ard[].(\proj[\aK'])
%    &
%    \text{if } \aK = \apref[\arole'][\arole][@][\aloc][.].\aK'
%    \\
    \multicolumn{3}{l}{
     \alrec [@][x][{
    	\ain[][{\asort[stop]}][x].\nil 
	+ 
	(\ain[][{\nu y:\asort[loc]}][x]. \proj[\aK'][@][\eta,X\mapsto y])}] [\aphi \arole]}
     \\
     &
      \multicolumn{2}{l}{
    \text{if }\ \aK = \grec[\arole'][@][\aK'][\aphi], \ \arole \neq \arole', \text{ and }\ \{x, y\}\cap (\fn{\aK'} \cup \cod \eta) = \emptyset
    }
    \\[10pt]
    \multicolumn{3}{l}{
     \alrec [@][@][{
    	\arout[][{\asort[stop]}][x_1]\ldots\arout[][{\asort[stop]}][x_n].\nil 
	+ 
	\arout[][{\nu y_1:\asort[loc]}][x]\ldots\arout[][{\nu y_n:\asort[loc]}][x]. \proj[\aK'][@][\eta,X\mapsto \seq y]}] [\aphi \arole_1\ldots\aphi \arole_n]}
\\
    &
    \multicolumn{2}{l}{
    \text{if }\ \aK = \grec[\arole'][@][\aK'][\aphi], \dom{\aphi} = \{\rho_1, \ldots, \rho_n\}, \seq x = x_1\ldots x_n,
    }
    \\
    & 
     \multicolumn{2}{l}{
     \seq y = y_1\ldots y_n, \text{ and }  (\seq x\cup \seq y) \cap (\fn{\aK'} \cup \cod \eta) = \emptyset 
      }
     \\[10pt]
      \alvar [@][\eta X]
    & \text{if }\ \aK = X
  \end{array}
  \right.
\]
\caption{Projection}
\label{fig:projection}
\end{figure}
%
We now comment on the definition of $\proj$ in \cref{fig:projection}. 
%
As usual, 
the local type corresponding to a role $\arole$ that is not part of $\aK$  
is $\nil$. 
%
The projection of a prefix $\apref$ depends on the role played by 
$\arole$ in $\apref$: it is omitted when $\arole$ does not participate on $\apref$; it is 
 the production of a tuple when $\apref$ is an interaction
 or an autonomous output and $\rho$ is the producer; it is the consumption of a tuple   
 when $\apref$ is an autonomous input or a consuming interaction and $\rho$ is the consumer;
 or else it is the read of a tuple. 
%
Projection is homomorphic with respect to choices and 
sequential composition. 

 A global type $\grec[@][@][@][\ ]$ is projected as a recursive 
local type $\alrec$ where the formal parameters $\seq x$ stand 
for the locations used for coordination and 
$\seq \alocvar$ are the initial values. 
%
Note that $\grec[@][@][@][\ ]$ does not make explicit the set of 
initial locations but they are so in local types. 
%
For this reason, we
define projection for a decorated version of global types, where 
each recursive sub-term $\grec$ is annotated by a function
$\aphi:\roleset \mapsto \locset$
defined such that $\dom\aphi = \roles\aK\setminus\{\arole\}$ and 
 for all $\arole\in\dom\aphi$,  $\aphi(\arole)$ is globally fresh.
%
Such annotations can be  automatically added by pre-processing  
global types so to associate a fresh set of locations to each recursive
process. 
%
Then, the projection of 
 $\grec[\arole'][@][\aK'][\aphi]$ onto $\arole$ depends on whether $\arole$
 coordinates the recursion  (i.e., $\arole = \arole'$) or not. 
%
When $\rho$ is not 
 the coordinator, the recursive process needs just one location $x$ to await for 
 either $\asort[stop]$ or a new location $y$ for the next iteration.
%
Note 
 that the body of the recursion $\aK'$ is then projecting by considering an extended 
 version of $\eta$ where process variable $X$ is parameterised with the received 
 location $y$. 
%
The initial value of  $x$ is fixed  according to  
 by $\phi$  (i.e., $\phi\arole$). 
%
Differently, when $\arole$ coordinates the recursion, the projection generates 
a process variable that has several parameters, i.e., one location $x_i$ for 
each passive role. 
%
In this case the body of the recursion consists of two 
branches: one that communicates the termination of the recursion to 
each participant, and the other one executes the body of the recursion after
distributing fresh localities to each participants. 
%
Recursion parameters are
initialised analogously. 
%
Finally, a process variable $X$ is projected as its parameterised version 
$\alvar [@][\eta X]$, where the value of parameters are established according 
to $\eta$.


%\begin{example}
%
% This protocol can be formalised by the following global type.
%  \[
%    \begin{array}{l}
%      \ard[\amulti {Seller}][{\atuple[start]}][{\aloc[m]}] \prefop
%      \\
%      \arout[\amulti {Seller}][{{\asort[str]} \cdot {\asort[int]} \cdot {\nu l : \asort[loc]}}][{\aloc[m]}] \prefop
%      \alrec[X][][(\nil + {\arout[\amulti {Seller}][{{\asort[str]} \cdot {\asort[int]} \cdot {\nu l : \asort[loc]}}][{\aloc[m]}]} \prefop X)][]
%      \seqop
%      \\  	
%      \grec[\amulti{Buyer}][Y][{\left({
%      \begin{array}{l}
%        \grec[][Z][{\apref[@][\amulti {Buyer}][{{\asort[str]} \cdot {\asort[int]} \cdot {\asort[loc]}}][{\aloc[m]}][.] \prefop Z}] \seqop
%        \\
%        \apref[@][\amulti {Buyer}][{{i : \asort[str]} \cdot {p : \asort[int]} \cdot {\nu l : \asort[loc]}}][{\aloc[m]}] \prefop
%        \\
%        \grec[\amulti{Seller}][W][\left({
%        \begin{array}{l}
%          \apref[\amulti {Buyer}][\amulti {Seller}][{i : \asort[{str}] \cdot {o : \asort[int]} }][{l}] \prefop
%          \\
%          \qquad 
%          \apref[\amulti {Seller}][\amulti {Buyer}][{\asort[quit]}][{l}] \prefop
%          \\
%          \qquad 
%          \apref[\amulti {Seller}][@][{{i : \asort[str]} \cdot {p : \asort[int]} \cdot {\nu l : \asort[loc]}}][{\aloc[m]}] \prefop
%          \\
%          \qquad
%          Y	
%          \\
%          \qquad
%          \chop
%          \\
%          \qquad 
%          \apref[\amulti {Seller}][\amulti {Buyer}][{\asort[sold]}][{l}]\prefop
%          %	\\
%          %	\qquad \apref[\amulti {Seller}][\ptp {broker}][i :
%          % {\asort[str]} \cdot o : {\asort[int]} ][{\aloc[m]}]\prefop
%          % \\ \qquad
%          Y	
%          \\
%          \qquad
%          \chop
%          \\
%          \qquad
%          \apref[\amulti {Seller}][\amulti {Buyer}][{\asort[more]}][{l}] \prefop W
%          \\
%          \chop
%          \\
%          \apref[\amulti {Buyer}][\amulti {Seller}][{\asort[{noway}]}][{l}] \prefop
%          \\\qquad
%          \apref[\amulti {Seller}][@][{{i : \asort[str]} \cdot {p : \asort[int]} \cdot {\nu l : \asort[loc]}}][{\aloc[m]}] \prefop
%          \\
%          \qquad
%          Y
%        \end{array}
%        }\right)][1]
%      \end{array}
%      }\right)}][1]
%    \end{array}
%  \]
%
%\end{example}

%For recursive terms  $\alrec$, we assume $|\seq x| = |\seq \alocvar|$, 
% variables in $\seq x$ pair-wise different, 
%$\aphi$ injective, $\dom(\aphi) = \roles \aK$, ${\it range}(\aphi) = [0..|\seq x|]$ 
%and $\aphi(\rho) = 0$.
%

%%% Local Variables:
%%% mode: latex
%%% TeX-master: "main"
%%% End:

  
\section{Conclusions}
\label{sec:disc}\label{sec:conc}
% !TEX root =  main.tex


This paper, a modest attempt to thank Rocco for his work and
friendship, addresses the following question:
%
\begin{quote}
  What notion of behavioural types corresponds to Linda-based
  coordination mechanisms?
\end{quote}
%
To answer such question we advocate Klaim-based global and local
types, dubbed klaimographies.
%
Klaim has been designed to program distributed systems consisting of
processes interacting via multiple distributed tuple spaces.

For simplicity, we have neglected code mobility, a distinctive feature
of Klaim.
%
Accommodating the mobility mechanism of Klaim would require to control
the multiplicity of running instances and to generalise the
well-formedness conditions to dynamically spawned processes.
%
A further challenge, would be to include mobility of
processes-as-values featured by Klaim, which shares many
similarities with session delegation.
%
However, this can be associated to control-driven problems.
%
These challenges are scope for future work.

We have also not considered parallel types.
%
A simple way to compose klaimographies in parallel would be to follow standard
approaches restricting roles on single threads and disjoint tuple
spaces.
%
We consider this not very interesting, and plan to explore more expressive
settings for parallel types such as the one in~\cite{gt16,gt17}.
%
In particular, we conjecture that to add parallel composition
$\aK \mid \aK'$ of klaimographies it is enough to require that
$\neg(\atuple \matches \atuple')$ for all
$(\atuple, \aloc) \in \aK, (\atuple', \aloc) \in \aK'$.
%
This condition is the counterpart of the \emph{well-forkedness}
condition of~\cite{gt16,gt17}, that requires that different threads of
a choregraphy have disjoing input actions.

Klaim has been extended with several features designed on theoretical
foundations and implemented in a suite of
prototypes~\cite{klaim}.
%
On the one hand, klaimographies share similarities with standard
behavioural types centred on point-to-point channel based
communications, on the other hand they also have some peculiarities,
some of which we highlighted here.

The closest work to our is~\cite{chjny19}, which develops the initial
proposal on parameterised choreographies in~\cite{ydbh10,dybh12}.
%
Notably,~\cite{chjny19} is the first work to support indexed roles and
to statically infer the participants inhabiting them.
%
The main difference with the approach in~\cite{chjny19} is that
klaimographies do not focus on processes, but rather on data.
%
We envisage behavioural types as specifications of how to guarantee
general properties of tuple spaces.
%
For instance, take the marketplace example (cf. \cref{ex:market}),
one would like to check properties such as
\begin{quote}
  for each tuple type
  $\atuple = {{i : \asort[str]} \cdot {p : \asort[int]} \cdot {\nu l :
      \asort[loc]}}$ consumed from locality $\aloc[m]$ either a tuple type
  $\asort[sold]$ is eventually generated at locality $l$ or $\atuple$
  is eventually generated at $\aloc[m]$.
\end{quote}
%
Such property does not concern typical properties
controlled by behavioural types (e.g., progress of processes, message
orphanage, or unspecified reception).

As scope for future work, we aim to characterise the (classes of)
properties of interest that klaimographies enforce.
%
We conjecture that the well-formedness conditions defined here
are strong enough to guarantee the property above.
%
Another interesting line of research is to identify typing principles
for Klaim processes.
%
We believe that klaimographies can enable the possibility that a same
process enacts different roles.
%
For instance, considering again the marketplace example, a process
can act both as seller and as buyer.

We have adopted a few simplifying assumptions.
%
Other variants seem rather interesting.
%
For instance, guards of sums could be autonomous inputs and not just
consuming interactions, or even read-only access prefixes.
%
Relaxing the constraint that read-only tuples cannot generate, would
lead to a sort of multi-cast mechanism of fresh localities.
%
We plan to study those variants in future work.


%%% Local Variables:
%%% mode: latex
%%% TeX-master: "main"
%%% End:


\bibliographystyle{abbrv}
\bibliography{biblio}

\iffinal
\else
 \newpage
 \setcounter{tocdepth}{2}
 \listoffixmes
\fi

\end{document}

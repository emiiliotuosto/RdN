\documentclass[runningheads,a4paper]{llncs}

% !TEX root =  main.tex
\usepackage{hyperref}
\usepackage{amsmath}
\usepackage{amssymb}
\usepackage{mathtools}
\usepackage{stmaryrd}
\usepackage{graphicx}
\usepackage{subfigure}
\usepackage{multicol}
\usepackage[usenames,dvipsnames,svgnames,table]{xcolor} % must be loaded before tikz %% use monochrome option to remove colors
\usepackage{tikz}
\usepackage{xstring}
\usepackage{graphics,framed}
\usepackage[capitalise]{cleveref}
\crefformat{enumi}{condition~#2#1#3}
\crefname{fact}{Fact}{Facts}
\Crefname{fact}{Fact}{Facts}
\crefformat{fact}{Fact~#2#1#3}
\usepackage{xargs}
\usepackage{titlecaps}
\usepackage{listings}
\usepackage{etoolbox}
\usepackage{bussproofs}
\usepackage{etoolbox}
\usepackage{extarrows}
%\usepackage{minted}
%\AtBeginEnvironment{minted}{\fontsize{7.1}{7.1}\selectfont}
\usepackage{wrapfig}
\usepackage{lineno}

%\usepackage{xypic}
\usepackage{xspace}

\usepackage[inline]{enumitem}

% managing draft and final version (uncomment finaltrue to get final)
\newif\iffinal
% \finaltrue

\usepackage[textsize=tiny%
\iffinal
,disable%
\fi
]{todonotes}

\iffinal
\else
% for todo notes
\setlength{\marginparwidth}{2.2cm}
\fi



%%% Local Variables:
%%% mode: latex
%%% TeX-master: "main"
%%% End:

\input{macros_util}
% !TEX root =  main.tex
%%% Macros for Klaimographies ;-)

\def\finex{{\unskip\nobreak\hfil
\penalty50\hskip1em\null\nobreak\hfil$\diamond$
\parfillskip=0pt\finalhyphendemerits=0\endgraf}}
\newcommand{\dummypar}[1]{\ifempty{#1}{\_}{#1}}
%\newcommand{\undef}{\textit{undef}}

\newcommand{\asort}[1][s]{\mathtt{#1}}
\newcommand{\wildcard}{\star}
\newcommandx{\atuple}[1][1 = t]{\texttt{#1}}
\newcommandx{\aloc}[1][1=l]{\texttt{\color{blue}#1}}
\newcommand{\locset}{\mathcal{L}\!\mathit{oc}}
\newcommand{\locsort}{\mathcal{L}\!\mathit{oc}}
\newcommand{\varset}{\mathcal{V}}
\newcommand{\tupleset}{\mathcal{T}}
\newcommand{\mkop}[1]{\textcolor{ForestGreen}{#1}}
\DeclareMathOperator{\at}{\mkop{\tiny @}}
\DeclareMathOperator{\outop}{\mkop{!}}
\DeclareMathOperator{\parop}{\mkop{\mid}}
\DeclareMathOperator{\chop}{\mkop{+}}
\DeclareMathOperator{\inop}{\mkop{?}}
\DeclareMathOperator{\toop}{\mkop{\to}}
\DeclareMathOperator{\seqop}{\mkop{\prec}}
\DeclareMathOperator{\lseqop}{\mkop{\fatsemi}}
\DeclareMathOperator{\prefop}{\mkop{.}}
\newcommand{\rec}{\mkop{\mu}\xspace}
\newcommand{\nil}{\mkop{\textbf{0}}}
\newcommandx{\artupleat}[3][1={}, 2=\atuple, 3=\alocvar, usedefault=@]{
  \ifempty{#1}{{#2} \at {#3}}{{#2} \at {#3}}
}
\newcommandx{\atupleat}[3][1={}, 2=\atuple, 3=\alocvar, usedefault=@]{
  \ifempty{#1}{{(#2)} \at {#3}}{{(#2)} \at {#3}}
}
\newcommand{\clashes}{\sharp}
\newcommand{\matches}{\bowtie}
\newcommand{\arole}{\rho}
\newcommand{\ptp}[1]{{\mathsf{\MakeLowercase{#1}}}}
\newcommand{\amulti}[1]{{\mathsf{\titlecap{#1}}}}
\newcommand{\roleset}{\mathcal{R}}
\newcommand{\participants}{\mathcal{U}}
\newcommand{\multiroles}{\mathcal{M}}
\newcommand{\unknownstarop}{{\mkop{\ast}}}
\newcommand{\unknownop}{{\mkop{\odot}}}
\newcommandx{\unknownstar}[1][1=P]{{\amulti {#1}}^{\unknownstarop}}
\newcommandx{\unknown}[1][1= P]{{\amulti {#1}}^{\unknownop}}
\newcommand{\aK}[1][K]{\texttt{\textcolor{BrickRed}#1}}
\newcommand{\aL}[1][L]{\texttt{\textcolor{orange}#1}}
\newcommandx{\proj}[3][1=\aK,2=\arole,3=\eta,usedefault=@]{
  \dummypar{#1} \downharpoonleft_{\dummypar{#2}}^{\dummypar{#3}}
}
\newcommandx{\aint}[3][1=\arole, 2=\arole', 3=\aK, usedefault=@]{
  {#1} \toop {#2} \, \mkop{:} \, {#3}
}
\newcommandx{\apref}[5][1={},2={},3=\atuple,4=\alocvar,5={},usedefault=@]{
  \ifempty{#1}{
    \ifempty{#2}{\pi}{
      \ifempty{#5}{\ain[{#2}][{#3}][{#4}]}{\ard[{#2}][{#3}][{#4}]}
    }
  }{
    \ifempty{#2}{
      \ifempty{#5}{\aout[{#1}][({#3})][{#4}]}{\aout[{#1}][{#3}][{#4}]}
    }{
      {#1} \toop {#2} \, \mkop{:} \, \ifempty{#5}{({#3})\at{#4}}{{#3}\at{#4}}
    }
  }
}
\newcommandx{\alabel}[6][1=i,2={},3={},4=\atuple,5=\alocvar,6={},usedefault=@]{
  \ifempty{#1}{}{^{[#1]}}\apref[#2][#3][#4][#5][#6]
}
\newcommandx{\asum}[5][1=i,2=I,3=\arole,4=\arole,5=\aK,usedefault=@]{
  \displaystyle{\sum_{#1 \ifempty{#2}{}{\in #2}}}{
    \ifempty{#4}{\apref_{#1}}{
      \apref[#3][{#4_{#1}}][\atuple_{#1}][\alocvar_{#1}]
    }
    \ifempty{#5}{}{\mkop{.} #5_{#1}}
  }
}
\newcommandx{\gtrace}[3][1=1,2=n,3=\dummypar{},usedefault=@]{
  \alabel[{#3}]_{#1} \cdots \alabel[{#3}]_{#2}
}
\newcommandx{\ltrace}[2][1=1,2=n,usedefault=@]{
  \alpha_{#1} \cdots \alpha_{#2}
}

\newcommandx{\arout}[3][1=\arole,2=\atuple,3=\alocvar,usedefault=@]{
  \ifempty{#1}{#2 \outop #3}{{#1} \outop {#2} \at {#3}}
}
\newcommandx{\aout}[3][1=\arole,2=\atuple,3=\alocvar,usedefault=@]{
  \ifempty{#1}{(#2) \outop #3}{{#1} \outop {#2} \at {#3}}
}
\newcommandx{\ain}[3][1=\arole,2=\atuple,3=\alocvar,usedefault=@]{
  \ifempty{#1}{(#2) \inop #3}{{#1} \inop {(#2)} \at {#3}}
}
\newcommandx{\ard}[3][1=\arole,2=\atuple,3=\alocvar,,usedefault=@]{
  \ifempty{#1}{#2 \inop #3}{{#1} \inop {#2} \at {#3}}
}
\newcommandx{\grec}[4][1=\arole, 2=X, 3=\aK, 4={}, usedefault=@]{
  \rec^{#4}_{#1}\ #2 \mkop{.} #3
}
\newcommandx{\arec}[6][1=X,2=\aK,3={\arole},4={\aphi},5={\seq\alocvar}, 6={\seq x},usedefault=@]{
  {\big(\rec_{#3}^{#4}\ #1({#6}) \mkop{.} #2}\big)\langle#5\rangle
}
\newcommand{\aphi}{\phi}

\newcommandx{\aLpref}{\kappa}
\newcommandx{\aLsum}[4][1=i,2=I,3 = \aLpref,4=\aL,usedefault=@]{
    \displaystyle{\sum_{#1 \ifempty{#2}{}{\in #2}}}{{#3_{#1}.#4_{#1}}}
}

\newcommand{\selectors}[1]{\it{sel}(#1)}
\newcommand{\eqR}{\sim}

\newcommand{\irule}[2]{\frac{\textstyle\rule[-1.3ex]{0cm}{3ex}#1}{\textstyle\rule[-.5ex]{0cm}{3ex}#2}}

\newcommand{\rulename}[1]{\mathsf{[#1]}}

\def \mathaxiom #1#2{
  \begin{array}{l}%
    \ifempty{#2}{}{\hspace{0em}\mbox{\footnotesize$\mathsf{[#2]}$}\\}
    {#1}
  \end{array}
}

\def \mathrule #1#2#3{
  \begin{array}{l}%
    \ifempty{#3}{}{\hspace{0em}\mbox{\footnotesize$\mathsf{[#3]}$}\\}
    \irule{#1}{#2}
  \end{array}
}

\newcommand{\envmv}{\textcolor{cyan}{\Delta}}
\newcommand{\envtuple}{\textcolor{cyan}{\Gamma}}

\newcommandx{\red}[1][1={}]{\xlongrightarrow{#1}}

\newcommand{\roles}[1]{\mathsf{roles}\ifempty{#1}{}{(#1)}}

%%%%%%%%%%%%%%%%%%%%%%%%%%%%%%%%%%%%%%%%%%%%%%%%%%%%%%%%%%%%%%%%%%%%%%%%%%%%%
%%%                        START POMSETS MACROS                           %%%
%%%%%%%%%%%%%%%%%%%%%%%%%%%%%%%%%%%%%%%%%%%%%%%%%%%%%%%%%%%%%%%%%%%%%%%%%%%%%
\newcommand{\apom}{r}
\newcommand{\emptypom}{\epsilon}
\newcommand{\alf}{\lambda}
\newcommand{\projpom}[2]{{#1}\!\!\downharpoonright_{#2}}

%%% chosem macros to add to ggmacros
\newcommand{\eset}{\mathcal{E}}
\newcommand{\aR}[1][R]{{\colorR{#1}}}
\newcommand{\efst}[1]{\pi_1\ifempty{#1}{}{({#1})}}
\newcommand{\aConf}{s}
\newcommand{\alfof}[1]{\alf_{#1}}
\newcommand{\minev}[1]{\mathtt{min}_{[#1]}}
\newcommand{\esetof}[1]{\eset_{#1}}
\newcommand{\leqof}[1]{\leq_{#1}}
\newcommandx{\detM}[1][1=\aCM,usedefault=@]{\Delta({#1})}

\tikzset{
  pomset/.style={
    scale = .7,
    transform shape,
    smooth
  }
}
%
\newcommandx{\pomsetrep}[4][2=\alf, 3={}, 4={}, usedefault=@]{
  \left[
    \begin{array}[c]{c}
      \begin{tikzpicture}[every node/.style={pomset},#4]
        {#1;}
      \end{tikzpicture}
    \end{array}\right]_{#2}^{#3}
}
\DeclareMathOperator{\pomsetcup}{\sqcup}
\newcommand{\pomsetsingle}[1][\apref]{\pomsetrep{\node {$#1$}}[]}

\def\colorMsg{\color{BrickRed}}    
\def\colorR{\color{OliveGreen}}
\newcommand{\msg}[1][m]{\mathsf{\colorMsg{#1}}}
\newcommand{\lset}{\mathcal{L}}
\def\colorE{\color{orange}}
\newcommand{\al}[1][l]{{\colorE{#1}}}
\def\colorE{\color{orange}}
\newcommand{\aE}[1][E]{{\colorE{#1}}}
\renewcommand{\ae}[1][e]{{\colorE{#1}}}
\newcommandx{\rseq}[2][1=\apom,2={\aapom'},usedefault=@]{
  \gfun{seq}(\ifempty{#1}{\_}{#1},\ifempty{#2}{\_}{#2})
}
\def\colorFun{\color{black}}
\newcommand{\gfun}[1]{\ensuremath{\mathsf{\colorFun #1}}}
\newcommandx{\bp}[2][1={i},2=\apref,usedefault=@]{\gfun{bp}(\ifempty{#1}{\_}{#1}, \ifempty{#2}{\_}{#2})}
\newcommandx{\unique}[1][1=\apref,usedefault=@]{\gfun{unique}(\ifempty{#1}{\_}{#1})}
\newcommand{\ksem}[1]{\llbracket \ifempty{#1}{\_}{#1} \rrbracket}


%%%%%%%%%%%%%%%%%%%%%%%%%%%%%%%%%%%%%%%%%%%%%%%%%%%%%%%%%%%%%%%%%%%%%%%%%%%%%
%%% 			LOCAL TYPES
%%%%%%%%%%%%%%%%%%%%%%%%%%%%%%%%%%%%%%%%%%%%%%%%%%%%%%%%%%%%%%%%%%%%%%%%%%%%%

\newcommand{\dn}[1]{{\it dn}(#1)}
\newcommand{\fn}[1]{{\it fn}(#1)}
\newcommand{\names}[1]{{\it n}(#1)}
\newcommand{\sust}[2]{\{^{#2} / _{#1}\}}
%\newcommand{\generates}{\rightsquigarrow}
\newcommand{\generates}{=}
\newcommand{\alocvar}{{\color{red}\ell}}
\newcommand{\supp}[1]{\llcorner#1\lrcorner}
\newcommand{\consistent}[1]{#1\downarrow}

\newcommand{\eraseB}[1]{\downarrow\!{#1}}
\newcommandx{\ws}[2][1={\aK},2={\aK'},usedefault=@]{ws(\ifempty{#1}{\_}{#1}, \ifempty{#2}{\_}{#2})}
\newcommandx{\wb}[1][1={},usedefault=@]{wb(\ifempty{#1}{\_}{#1})}
\newcommand{\unfold}[5]{\text{unfold}_{#1}(\dummypar{#2}, \dummypar{#3}, \dummypar{#4}, \dummypar{#5})}

\newcommand{\seq}[1]{\widetilde{#1}}

\newcommandx{\alvar}[2][1=X,2={\seq\alocvar},usedefault=@]{#1\langle#2\rangle}
\newcommandx{\alrec}[5][1=X,2=\seq x,3=\aL,4=\seq\alocvar,5=,usedefault=@]{
{\big(\rec_{#5}#1({#2}) \prefop #3}\big)\langle#4\rangle}

\newcommandx{\abasiclrec}[3][1=X,2=\seq x,3=\aL,usedefault=@]{
{\big(\rec #1({#2}) \prefop #3}\big)}

%%%%%%%%%%%%%%%%%%%%%%%%%%%%%%%%%%%%%%%%%%%%%%%%%%%%%%%%%%%%%%%%%%%%%%%%%%%%% 
%%%                        END POMSETS MACROS                             %%%
%%%%%%%%%%%%%%%%%%%%%%%%%%%%%%%%%%%%%%%%%%%%%%%%%%%%%%%%%%%%%%%%%%%%%%%%%%%%%


%%% Local Variables:
%%% mode: latex
%%% TeX-master: "main"
%%% End:


\FXRegisterAuthor{eM}{aeM}{\color{orange} {\underline{eM}}}
%
% co-authors: please define your own macro for metacomments as the one
% above

\begin{document}

\mainmatter  % start of an individual contribution

% first the title is needed
\title{
  \iffinal
  Data-driven choreographies \`a la Klaim
  \else
  \fcolorbox{black!15}{yellow!20}{Data-driven choreographies \`a la Klaim}
  \fi
}

% a short form should be given in case it is too long for the running head

% the name(s) of the author(s) follow(s) next
%
% NB: Chinese authors should write their first names(s) in front of
% their surnames. This ensures that the names appear correctly in
% the running heads and the author index.
%

\author{
  Roberto Bruni \inst{1} \and
  Andrea Corradini \inst{1} \and
  Fabio Gadducci \inst{1} \and
  Hern\'an Melgratti \inst{2} \and
  Ugo Montanari \inst{1} \and
  Emilio Tuosto\inst{3}
}
\institute{
  Universit\`a di Pisa, Italy
  \and
  Departamento de Computaci\'on, FCEyN, Universidad de Buenos Aires - Conicet, Argentina
  \and
  University of Leicester, UK
}

\iffinal
\else
\authorrunning{\fcolorbox{black!15}{yellow!20}{Check list of fixmes}}
\fi

\maketitle

\begin{abstract}
  % Klaim is one of the many contributions of Rocco.
  % %
  % It mixes together many flavours and borders many research areas.
  % %
  % One can appreciate Klaim as a basic calculus alternative to other
  % name passing calculi such as the pi, join, or ambient.
  % %
  % Or one can see Klaim as a coordination model bringing the generative
  % mechanisms of Linda in the distributed setting.
  % %
  % Or else one can envisage Klaim as a programming paradigm.
  % 
  We propose Klaim as a suitable base for a novel choreographic
  framework.
  %
  More precisely we advocate Klaim as a suitable language onto which
  to project \emph{data-driven} global specifications based on
  distributed tuple spaces.
  %
  These specifications, akin behavioural types, describe the coordination
  from a global point of view.
  %
  Differently from behavioural types though, our specifications
  express the data flow across distributed tuple spaces rather than
  detailing the communication pattern of processes.
  %
  We devise a typing system to validate Klaim programs against projections
  of our global specifications.
  %
  An interesting feature of our typing approach is that well-typed
  systems have arbitrary number of participants.
  %
  In standard approaches based on behavioural types, this is often
  achieved at the cost of tremendous technical complications.
\end{abstract}



\section{Introduction}
\label{sec:intro}
% !TEX root =  main.tex

Communication-centered programming is playing a tremendous role in the production of nowadays software. Programming peers that need to exchange information is an error-prone activity and the behaviour of even small system is subject to a combinatorial blow-up as the number of peers increases.
Therefore well-structured principles and rigorous foundations are needed to develop well-engineered, trustworthy software. 
One possibility is to exploit some sort of behavioural types~\cite{} to manage abstract descriptions of peers and formally study their properties such as communication safety, absence of deadlocks, progress or session fidelity: given the types of the peers the emerging behaviour of their composition is analysed.
In the seminal paper~\cite{DBLP:conf/popl/HondaYC08}, recently nominated the \emph{most influential POPL paper (Award 2018)}, the authors push forward an abstract notion of global type of interaction that represents a sort of contract between the communicating peers. This is paired with the notion of local type that gives an abstract description of the behaviour of each peer, as taken in isolation.
Interestingly, local types can be obtained for free by projection from global types, while the properties of interest can be studied and guaranteed just at the level of global types, without the need of studying the composition of local types. The conformance of peers implementation w.r.t. the global type can be studied instead at the level of local types, allowing a more efficient form of type checking. Roughly this means that properties are stated globally but checked locally. Global types have been inspired by choreography languages in service oriented computing, where complex interactions are modelled from the point of view of the global sequence of events that must take place in order to successfully complete the computation.

In the literature, global/local types have been studied mostly in the context of dyadic, message-passing interactions. This means that the main action in a choreography is the sending of one message from one peer to another (on a given channel of a given type). In this paper we explore a different setting, where interaction over tuple-spaces replaces message passing, in the style of Linda-like languages~\cite{DBLP:journals/toplas/Gelernter85}.
Instead of primitives for sending and receiving messages, here there are primitive for inserting a tuple on a tuple space, for reading (without consuming) a tuple from a tuple space or for retrieving a tuple from a tuple space. We call these interactions data-driven, as decisions will be taken on the basis of the type of the tuples that are manipulated. We coined the term Klaimographies in honour of the process language Klaim~\cite{DBLP:journals/tse/NicolaFP98}, one of the main contributions of Rocco De Nicola in the field of tuple-space programming. Inspired by Klaim, Klaimographies exploit the notion of distributed tuple-space localities to separate the access to data on the basis of the interactions that are carried out. Localities can be communicated in tuples, allowing name mobility.

There are several major advances w.r.t. to the literature on global types.
First, Klaimographies can have an arbitrary number of participants, while in global types the number of participant is usually fixed a priori.
Second, interactions are not dyadic because each tuple can be read many times, while messages have exactly one producer and one consumer, even when asynchronous communication is considered.
Third, all interactions involve a tuple space locality instead of a channel name.
Fourth, Klaimographies are data-driven.

The main contribution of this paper is to set up the formal setting of Klaimographies and prepare the ground for several interesting research directions: we fix the syntax of global and local type description and define the projection from global type to local types as typical of choreographic frameworks.
Global types are equipped with a partial order semantics of events and local types with an ordinary operational semantics. Then the conditions under which the behaviour of projected local types are faithful to the semantics of global types are spelled out. 

Shifting the focus from control to data in choreographic framework has
several implications.
%
Firstly, the emphasis is no longer on properties related to
computational actors.
%
For instance, Klaimographies admit computations where some processes
may not terminate and is left waiting for some data.
%
In standard choreographic framework those would be undesired behaviour
to rule out with suitable typing disciplines.
%
Nonetheless, we claim that in some application domain computations with
deadlocked processes have to be considered non-erroneous.
%
For instance, in reactive systems based on event-notification
frameworks some \quo{listener} components must be kept waiting for
events to occur.
%
Our work paves the way to the formal study of properties of data, like consumption, persistence and availability, in a choreographic setting.

Another main innovation of klaimographies is that they allow us to easily
represent protocols where an arbitrary number of components can
play a role in the choreography.
%
We give an example of such protocol in \cref{sec:examples}.
%
Remarkably, those protocols can be specified in some existing
choreographic frameworks~\cite{ydbh10,chjny19}, but in a less abstract way
that requires the explicit quantification on components.

\paragraph{Structure of the paper}
In Section~\ref{sec:klaimographies} we define Klaimographies as global types and give some examples.
In Section~\ref{sec:globsem} we define the semantics of global types and give the adequacy conditions for projecting global types to local types.
In Section~\ref{sec:locsem} we define the operational semantics of local types and state its correspondence with the global semantics.
Some concluding remarks together with the discussion of related and future work are in Section~\ref{sec:conc}.

%%% Local Variables:
%%% mode: latex
%%% TeX-master: "main"
%%% End:



\section{Klaim-inspired Choreographies}
\label{sec:klaimographies}
\subsection{Tuple types}
% !TEX root =  main.tex
%

We consider a set of variables $\varset$  ranged over by $x$
and a set of localities $\locset$ ranged over by $\aloc$ (and
use $\alocvar$ to range over $\locset\cup\varset$) and we let $\asort$
range over basic sorts which include $\asort[int]$, $\asort[bool]$,
$\asort[str]$ and sort $\asort[loc]$ of \emph{localities}.
%
The set $\tupleset$ of \emph{(tuple) types} consists of the terms
derived from the following grammar:
\begin{eqnarray*}
  \atuple & \bnfdef & \asort \bnfmid
                        \wildcard\bnfmid
                      x : \asort \bnfmid
                      \nu x : \asort  \bnfmid
%                      \nu \aloc \bnfmid
%                      \aloc \bnfmid
                      \atuple \cdot \atuple 
\end{eqnarray*}
Tuple types are trees  $\atuple \cdot \atuple$  where leaves are
 either a sort $\asort$, any type $\wildcard$,  a sorted variable $x : \asort$,
or a fresh sorted variables $\nu x : \asort$ 
(the difference between $x : \asort$ and $\nu x : \asort$ is clarified in \cref{sec:examples}).
Note that $\nu x : \asort$ are binders that \emph{define}
$x \in \varset$. 
%
Hence, we talk about \emph{free} and \emph{defined} (sorted) names occurring in
tuples.
%
%We assume that in a tuple there is at most one occurrence of $\nu x$,
%for each $x \in \varset$.
%
The functions $\fn{\_}$ and $\dn{\_}$ return sets of pairs
$x \mapsto \asort$ assigning sort $\asort$ to $x \in \varset$.
\[
\begin{array}{l@{\hspace{1cm}}l}
\begin{array}{lcl}
  \dn\asort & = & \emptyset
  \\
  \dn{x : \asort} & = & \emptyset
  \\
  \dn{\nu x : \asort} & = & \{x \mapsto \asort \}
  \\
  \dn{\atuple_1 \cdot \atuple_2} & = & \dn{\atuple_1} \cup \dn{\atuple_2}
  \\
  \dn{\wildcard} & = & \emptyset
\end{array}
&
\begin{array}{lcl}
  \fn\asort & = & \emptyset
  \\
  \fn{x : \asort} & = &  \{x \mapsto \asort\}
  \\
  \fn{\nu x : \asort} & = & \emptyset
  \\
  \fn{\atuple_1 \cdot \atuple_2} & = & \fn{\atuple_1} \cup \fn{\atuple_2}
  \\
  \fn{\wildcard} & = & \emptyset
\end{array}
\end{array}
\]
We write $\supp{\_}$ to denote the projection of a set of pairs over its first component. 
%
We say a tuple $\atuple$ is {\em well-sorted} if the following two
conditions hold:
\begin{itemize}
\item
  $\supp{\fn\atuple} \cap \supp{\dn{\atuple}} = \emptyset$, i.e., free and
  defined names are disjoint; and
\item
  $\atuple = \atuple_1 \cdot \atuple_2$ implies $\atuple_1$ and
  $\atuple_2$ well-sorted and  names are disjoint, namely, 
  $\supp{\dn{\atuple_1}} \cap \supp{\dn{\atuple_2}} =  \emptyset$
  and 
  $\supp{\fn{\atuple_1}} \cap \supp{\fn{\atuple_2}} =  \emptyset$.
\end{itemize}
%
Hereafter, we assume all tuples to be well-sorted.
%
Note that $\fn{\atuple}$ and $\dn{\atuple}$ are partial functions (from names to sorts)
for well-sorted tuples.

A substitution of the free occurrences of a variable
$x$ in a (well-sorted) tuple $\atuple$ such that $x\not\in \dn{\atuple}$
 by a variable $y\not\in \dn{\atuple}$, written
$\atuple \sust x y$, is defined by\todo{AC: a substitution can make the tuple non-well-sorted. Is this a problem?}
%
\[
\begin{array}{r@{}l@{\ = \ } ll}
%\asort
%&
%\sust x y  
%&  
%\asort
%\\
%(x : \asort)
%& 
%\sust x y  
%& x : \asort
%\\
%z
%&
%\sust x y 
%&  
%z 
%& 
%{\it if} z\neq x
%\\
(x  : \asort)
&
\sust x y  
&  
y : \asort
\\
(\atuple_1 \cdot \atuple_2)
&
\sust x y  
& 
(\atuple_1\sust x y) \cdot (\atuple_2\sust x y) 
%& 
%{\it if} y\not\in\dn{\atuple_1 \cdot \atuple_2}
%\\
%\wildcard
%&
%\sust x y  
%&  
%\wildcard
\end{array}
\]
%
and it is the identity on the remaining cases. Let
$\sigma = \{y_1/x_1,\ldots,y_n/x_n\}$ such that $x_i\neq x_j$ for all
$i\neq j$ (i.e., a partial endo-function on $\varset$), we write
$\atuple\sigma$ for the simultaneous substitution of each $x_i$ by
$y_i$.
%
We use $\Sigma$ for the set of all substitutions. We write
$\sigma_1\sigma_2$ for the composition of partial functions with
disjoint domain, and $\sigma_1[\sigma_2]$ for the update of $\sigma_1$
with $\sigma_2$.


Tuple types $\atuple$ and $\atuple'$ s.t.
$\dn{\atuple}\cap\dn{\atuple'} = \emptyset$
% \todo{AC:aggiunto $=\emptyset$}
can \emph{match} by producing a substitution; formally
this is realised by the partial function
$\matches : \tupleset \times \tupleset \to \Sigma$ below
%\eMnote{add typing env}
\[
  \atuple \matches \atuple' \generates
    \begin{cases}
     \emptyset
    & 
    \text{if  } \atuple = \wildcard \vee \atuple' = \wildcard  \vee (\atuple = \atuple' \land \atuple\in\{\asort, x : \asort\})
%    \\
%    \sigma_1\sigma_2
%    &
%    \text{if } \atuple = \atuple_1 \cdot \atuple_2
%    \land  \atuple' = \atuple'_1 \cdot \atuple'_2
%    \land \atuple_1 \matches \atuple'_1 \generates \sigma_1
%    \land \atuple_2 \matches \atuple'_2 \generates \sigma_2
    \\
    \sigma
    &
    \text{if } \atuple = \atuple_1 \cdot \atuple_2
    \land  \atuple' = \atuple'_1 \cdot \atuple'_2
    \land \atuple_1 \matches \atuple'_1 \generates \sigma_1
    \land \atuple_2\sigma_1 \matches \atuple'_2\sigma_1 \generates \sigma
    \\
    \sust x y 
    &
    \text{if  } (\atuple = \nu y : \asort \land \atuple' = x: \asort) \vee  (\atuple' = \nu y : \asort \land \atuple = x: \asort) 
    \\
    \textit{undef} & \textit{otherwise}
   \end{cases}
\]
%
We write $\atuple \matches \atuple'$ when
$\atuple \matches \atuple' = \sigma$ for a substitution
$\sigma \in \Sigma$.
%
%
%
%\[
%  \atuple \matches \atuple' \iff
%  \begin{cases}
%    \atuple = \atuple'  
%    & 
%    \text{if $\atuple\in\{\asort, x : \asort, x, \nu \aloc, \aloc\}$}
%    \\
%    \atuple_1 \matches \atuple'_1
%    \land \atuple_2 \matches \atuple'_2
%    &
%    \text{if } \atuple = \atuple_1 \cdot \atuple_2
%    \text{ and }  \atuple' = \atuple'_1 \cdot \atuple'_2
%    \\
%    \atuple = \wildcard \vee \atuple' = \wildcard & \text{otherwise}
%  \end{cases}
%\]
%%

\noindent
We say that $\atuple$ \emph{generates} when in one of its fields there
is a $\nu x: \asort[loc]$ type. 
%\todo{RB: meglio dire $\nu \aloc: \asort[loc]$}
%

%We define
%\[
%  \atuple \clashes \atuple' \iff \exists \atuple'' \qst \atuple'' \matches \atuple \land \atuple'' \matches \atuple'
%\]
%
%and say that $\atuple$ and $\atuple'$ \emph{are in conflict} when
%$\atuple \clashes \atuple'$.


%%% Local Variables:
%%% mode: latex
%%% TeX-master: "main"
%%% End:

\subsection{Global types}
% !TEX root =  main.tex
%

We fix two disjoint sets $\participants = \{\ptp p, \ptp q, \ldots\}$
and $\multiroles = \{\amulti P, \amulti Q, \ldots \}$, respectively of
\emph{unit} role and \emph{multiple} roles, and define the set of
\emph{roles} $\roleset = \participants \cup \multiroles$.
% \cup \multiroles^\unknownstarop \cup
% \multiroles^\unknownop$ (ranged over by $\arole, \arole_1,
% \ldots$) where $\multiroles^{\unknownstarop} = \{\unknownstar \sst
% \amulti P \in\multiroles\}$ and $\multiroles^{\unknownop} = \{
% \unknown \sst \amulti p \in \multiroles
% \}$ account for some flexibility when implementing multiroles: an
% action involving a
% $\unknownstar$ role can be optionally executed by an implementation
% of role $\amulti P$, while
% $\unknown$ establishes that exactly one implementer must execute
% that action.
%
We conventionally write multiple roles with initial uppercase letter and
unique roles with initial lowercase letter.
%
%We write $\eqR$ for the least equivalence relation on $\roleset$
%satisfying
%\[\amulti P \eqR \unknownstar \hspace{2cm} \amulti P \eqR \unknown\]

Roles are to be thought of as types inhabited by instances of
processes enacting the behaviour specified in a choreography.
%
Unit roles are unit types while multiple roles account for multiple
instances of processes all performing actions according to their role.
%
%\todo{RB: forse \`e importante dire che tutti i multiruoli sono disgiunti, giusto?}
%\todo{eM: direi di no; in the long run vorremmo che un processo possa coprire piu' ruoli}

%
Let us first define the \emph{prefixes} used in global types
with the following grammar:
% \todo{RB: ``consuming output'' mi sembra fuorviante... ma non ho proposte migliori}
%
\[\begin{array}{lcl@{\qquad\qquad}l}
  \apref & \bnfdef
  & \apref[\arole][@][@][\alocvar] & \text{(autonomous) output}
  \\ & \bnfmid
  & \apref[\arole][{}][@][\alocvar][.] & \text{(autonomous) read-only output}
  \\ & \bnfmid
  & \apref[{}][\arole][@][\alocvar] & \text{(autonomous) input}
  \\ & \bnfmid
  & \apref[{}][\arole][@][\alocvar][.] & \text{(autonomous) read}
  \\ & \bnfmid
  & \apref[\arole][\arole'][@][\alocvar] & \text{consuming interaction}
  \\ & \bnfmid
  & \apref[\arole][\arole'][@][\alocvar][.]  & \text{read-only interaction}
  \end{array}
\]
%
The set $\roles \apref \subseteq \roleset$ of roles in $\apref$ is
defined in the obvious way; note that $\roles \apref$ is a singleton
if, and only if, $\apref$ is an autonomous prefix.
%
We syntactically distinguish two kinds of prefixes.
%
The prefixes generated by the first four productions in the grammar of
$\apref$ above are the \emph{autonomous} prefixes, that is those
prefixes that processes can execute directly on a tuple space without
coordinating with other processes.
%
The prefixes generated by the remaining two productions of the grammar
are the \emph{interaction} prefixes, namely those involving a role
generating the tuple and one accessing them.
%
Inspired by Klaim, processes can access tuple types according to two
modalities syntactically distinguished by the round brackets around
the tuple in prefixes.
%
More precisely, when a prefix surrounds a tuple $\atuple$ with round
brackets then $\atuple$ is meant to be consumed otherwise it is meant
to be read-only.
%
% We call \emph{read-only outputs} and \emph{reads} the autonomous
% read-only prefixes while those prescribing the consumption of tuples,
% and consuming (resp. read-only) interactions the interations
% prescribing inputs (resp. reads).
%
We assume that tuple types used in read-only modalities do not
generate.
%
\eMnote{Una variante interessante e' quella dove questo vincolo e'
  rilassato.}

Global types $\aK$ have the following syntax
% \eMnote{non e' chiaro se vogliamo $\aK \parop \aK $}
\begin{eqnarray*}
  \aK & \bnfdef & \asum[@][@][][]
                  \bnfmid
                  % \aK \parop \aK \bnfmid
                  \aK \seqop \aK \bnfmid
                  X \bnfmid
                  \grec
\end{eqnarray*}
% \todo{RB: avendo $\seqop$ direi che il parallelo non \`e necessario}
where $I$ is a finite set of indexes; we write $\nil$ for
$\asum[@][@][][]$ when $I = \emptyset$ (we omit trailing occurrences
of $\nil$) and $\apref_j.\aK_j$ instead of $\asum[@][@][][]$ when
$I = \{j\}$.
%
The set $\roles \aK$ of roles of $\aK$ is the set of roles that are mentioned in $\aK$ and it is defined in the obvious way.

The syntax of global types features prefix guarded choices, sequential
composition, and recursion.
%
To handle recursive behaviour, 
%\todo{AC+RB: il comportamento della ricorsione si intuisce solo dopo aver visto la semantica. Forse bisognerebbe tentare di spiegarlo intuitivamente?}
%\todo{eM: ora?}
the construct $\grec$ singles out a role $\arole \in \roles \aK$
%
% and specifies an injective function
% $\aphi : \roles \aK \setminus \{\arole\} \to \locset$ such that the
% locations in $\cod(\aphi)$ do not occur in $\aK$; intuitively,
% $\arole$
deciding when the recursion ends.
%\todo{RB: pi\`u sotto viene usata la notazione $\alvar$ invece di $X$: bisogna cambiare la grammatica delle coreografie?}
% and, for all $\arole' \neq \arole$ in $\aK$, the location
% $\aphi(\arole')$ is used to communicate the decision of $\arole$ to
% $\arole'$.
%
Intuitively, in $\grec$ role  $\arole$ non-deterministically decides whether 
to repeat the execution of  the body $\aK$  or (if ever) to end it.
%
To achieve this,  $\arole$ notifies the decision by generating tuple types for the
other roles (this is formally defined in \cref{sec:globsem}).
%
We omit the decoration $\arole$ when $\roles \aK = \{\arole\}$.

We extend the notions of defined and free names to global types as
follows:
%\todo{RB: ma la $\nu$ pu\`o essere usata anche in una tupla di input/read (senza output)?}
%\todo{eM: si, ma poi non sincronizza}
\[
 \fn{\apref[\arole][@][@][\alocvar]}
 = \fn \atuple \cup \{\alocvar \mapsto \asort[loc]\} 
\qquad
 \dn{\apref[\arole][@][@][\alocvar]} 
 = \dn \atuple 
\]
omitted prefixes are defined analogously.
\[
  \begin{array}{ll}
    \begin{array}{l@{\ =\ } ll}
      \fn{\asum[@][@][][]} & \displaystyle{\bigcup_{i\in I}} \fn{\apref_i} \cup (\fn{\aK_i}\setminus\dn{\apref_i})
      \\
      \fn{\aK_1 \seqop \aK_2} 
                &
                  \fn{\aK_1}\cup\fn{\aK_2}
      \\
      \fn X & (\seq \alocvar \cap \varset)
      \\
      \fn \grec & \fn {\aK}
    \end{array}
    \begin{array}{l@{\ =\ } ll}
      \dn{\asum[@][@][][]} & \displaystyle{\bigcup_{i\in I}} \dn{\apref_i} \cup \dn{\aK_i}
      \\
      \dn{\aK_1 \seqop \aK_2} 
                           &
                             \dn{\aK_1}\cup\dn{\aK_2}
      \\
      \dn X & \emptyset
      \\
      \dn \grec & \dn {\aK}
    \end{array}
  \end{array}
\]
%
We write $\names \_$ for the set of sorted names of a term, i.e.,
$\names \apref = \fn\apref \cup \dn\apref$ and similarly
$\names \aK = \fn\aK \cup \dn\aK$. A set $S$ of sorted names is
consistent, written $\consistent S$, if $x\mapsto \asort \in S$ and
$x\mapsto \asort' \in S$ implies $\asort = \asort'$.
 
The set of well-sorted terms are defined inductively as follows:

\begin{itemize}
\item $\apref$ is well-sorted if $\fn\apref\cap\dn\apref = \emptyset$ and  
$\consistent{\names\apref}$, i.e., there are no clashes/inconsistencies in the sorts of 
the names in the component $\atuple$ of $\pi$ and the locality $\alocvar$ mentioned in $\pi$;
\item $\asum[@][@][][]$ is well-sorted if for all ${i\in I}$ both
  $\apref_i$ and ${\aK_i}$ are well-sorted and
  $\consistent{\names {\apref_i.\aK_i}}$;
\item $\aK_1 \seqop \aK_2$ is well-sorted if $\aK_1$ and $\aK_2$ are
  well-sorted and $\consistent{\names{\aK_1 \seqop \aK_2}}$;
\item $X$ is well-sorted and $\grec$ is well-sorted if $\aK$ is
  well-sorted.
\end{itemize}


We consider terms up-to $\alpha$-renaming of defined names and
recursion variables.
%
Correspondingly, substitutions are capture avoiding, in the sense that
defined names can be renamed to fresh names before any substitution is
applied to a term.
% \todo{RB: aggiunto}
%
As usual we say that a global type $\aK$ is \emph{closed} when it does
not contain free occurrences of recursion variables $X$ or free
occurrences of names.%, i.e., $\supp{\fn\aK} \cap \varset = \emptyset$.


  
%%% Local Variables:
%%% mode: latex
%%% TeX-master: "main"
%%% End:

\subsection{Some examples}
\label{sec:examples}
We give a few simple global types (\cref{ex:cs,ex:CS,ex:CSx,ex:CSl})
to highlight some peculiarities of our choreographies as well as a
more complex example (\cref{ex:auction}) to illustrate the type of
protocol our choreographies can capture.

\begin{example}\label{ex:cs}
  Consider the following global type that describes the interaction of
  a client $\ptp c$ with a simple service $\ptp s$ that converts
  integers into strings.
  \[
    \aK_\eqref{ex:cs} =
    \apref[\ptp c][\ptp s][{\asort[int]}][{\aloc[l]}]  \prefop
    \apref[\ptp s][\ptp c][{\asort[str]}][{\aloc[l]}]
  \]
  The client $\ptp c$ produces an integer value on the locality
  $\aloc$.
  %
  This tuple must be consumed by the server $\ptp s$, which produces
  back the converted string for the client.
  %
  \finex
\end{example}

Elaborating on the previous example we now discuss a few features of
our setting.

\begin{example}\label{ex:CS}
  Assume that we consider client and server in \cref{ex:cs} as
  multiroles instead of single participants, and write
  \[
    \aK_\eqref{ex:CS} =
    \apref[\amulti {C}][\amulti {S}][{\asort[int]}][{\aloc[l]}] \prefop
    \apref[\amulti {S}][\amulti {C}][{\asort[str]}][{\aloc[l]}]
  \]
  In this case, $\aK_\eqref{ex:CS}$ states that each integer produced by a client
  will be consumed by a server, which will in turn produce a string
  for one of the clients.
\end{example}
The type in \cref{ex:CS} does not ensure that clients consume the
string conversion of the integer they produced, because all tuples are put at the same location $\aloc[l]$.
%
Name binders can be used to correlate tuples.
%
\begin{example}\label{ex:CSx}
  Consider
  \[
    \aK_\eqref{ex:CSx} =
    \apref[\amulti {C}][\amulti {S}][{\nu x:\asort[int]}][{\aloc[l]}] \prefop
    \apref[\amulti {S}][\amulti {C}][{x : {\asort[int]} \cdot {\asort[str]} }][{\aloc[l]}]
  \]
  $\aK_\eqref{ex:CSx}$ associates a fresh identifier $x$ to each value produced by
  a client and consumed by a server.
  %
  Despite the identifier is known only to the communicating instances,
  this does not forbid two clients to generate the same integer value.
  %
  Basically, the name $x$ allows to express constraints about the flow
  of values.
  %
  In particular, $x : \asort[int]$ in the second interaction states
  that a server must generate a tuple that contains the consumed
  integer.
  %
  Analogously, each client must consume a tuple matching the produced
  integer.
  %
  \finex
\end{example}
The choreography in \cref{ex:CSx} does not establishes a one-to-one
association between instances of $\amulti C$ and $\amulti S$.
%
In fact, an instance of $\amulti C$ not necessarily interacts with the
same instance of $\amulti S$ in the two communications when two
instances of $\amulti C$ generate the same integer in the first
interaction.
%
A one-to-one correspondence can be achieved by using fresh localities.
\begin{example}\label{ex:CSl}
  Consider
  \[
    \aK_\eqref{ex:CSl} = 
    \apref[\amulti {C}][\amulti {S}][{{\asort[int]} \cdot \nu x: {\asort[loc]}}][{\aloc[l]}]  \prefop
    \apref[\amulti {S}][\amulti {C}][{{\asort[str]}}][{x}]
  \]
  %
  A client generates a fresh locality identified by $x$, which is then
  used as the locality for the subsequent communication.
  %
  Since the name $x$ is known only to the two communicating instances,
  the second interaction can only take place between the two instances
  that know the locality $x$.
  %
  \finex
\end{example}

\todo[inline]{We think we can also deal with situations like the following, in which 
A creates a private session for B and C. 
%
\[
  \apref[\amulti {A}][\amulti {B}][{\nu y: {\asort[loc]}}][{\aloc[l]}] \prefop
\]
\[
  \apref[\amulti {A}][\amulti {C}][{y : {\asort[loc]}}][{\aloc[l]}] \prefop
\]
\[
  \apref[\amulti {B}][\amulti {C}][\asort][{y}]
\]
}

\eMnote{{definire $\grec[@][@][i]$ and fare un forward ref a ws}}
\eMnote{rimossa interazione col broker nella scelta; utile esempio per
  far vedere che non catturiamo tutti i systemi 'buoni' con wb}
\begin{example}[Auction]\label{ex:auction}
\todo[inline]{ma \`e lecito usare $\nu$ in $\apref[@][\amulti {Buyer}][{{i : \asort[str]} \cdot {p : \asort[int]} \cdot {\nu l : \asort[loc]}}][{\aloc[m]}]$ o dovrebbe essere $\apref[@][\amulti {Buyer}][{{i : \asort[str]} \cdot {p : \asort[int]} \cdot {l : \asort[loc]}}][{\aloc[m]}]$?}
\[
  \begin{array}{l}
  \apref[\ptp {broker}][\amulti {Seller}][{\atuple[makeOffer]}][{\aloc[m]}][.] \prefop
  \\  	
  \grec[][X][{\apref[\amulti {Seller}][@][{{\asort[str]} \cdot {\asort[int]} \cdot {\nu l : \asort[loc]}}][{\aloc[m]}] \prefop X}] \seqop
  \\  	
  \grec[\amulti{Buyer}][Y][{\left({
      	\begin{array}{l}
          \grec[][Z][{\apref[@][\amulti {Buyer}][{{\asort[str]} \cdot {\asort[int]} \cdot {\asort[loc]}}][{\aloc[m]}][.] \prefop Z}] \seqop
          \\
          \apref[@][\amulti {Buyer}][{{i : \asort[str]} \cdot {p : \asort[int]} \cdot {\nu l : \asort[loc]}}][{\aloc[m]}] \prefop
          \\
          \grec[\amulti{Seller}][W][\left({
    	\begin{array}{l}
	\apref[\amulti {Buyer}][\amulti {Seller}][{i : \asort[{str}] \cdot {o : \asort[int]} }][{l}] \prefop
	\\
	\qquad 
	\apref[\amulti {Seller}][\amulti {Buyer}][{\asort[quit]}][{l}] \prefop
	\\
	\qquad 
	\apref[\amulti {Seller}][@][{{i : \asort[str]} \cdot {p : \asort[int]} \cdot {\nu l : \asort[loc]}}][{\aloc[m]}] \prefop
	\\
	\qquad
	Y	
	\\
	\qquad
	\chop
	\\
	\qquad 
	\apref[\amulti {Seller}][\amulti {Buyer}][{\asort[sold]}][{l}]\prefop
%	\\
%	\qquad 
%	\apref[\amulti {Seller}][\ptp {broker}][i : {\asort[str]} \cdot o : {\asort[int]} ][{\aloc[m]}]\prefop
%	\\ \qquad
	Y	
	\\
	\qquad
	\chop
	\\
	\qquad
	\apref[\amulti {Seller}][\amulti {Buyer}][{\asort[more]}][{l}] \prefop W
	\\
	\chop
	\\
        \apref[\amulti {Buyer}][\amulti {Seller}][{\asort[{noway}]}][{l}] \prefop
        \\\qquad
        \apref[\amulti {Seller}][@][{{i : \asort[str]} \cdot {p : \asort[int]} \cdot {\nu l : \asort[loc]}}][{\aloc[m]}] \prefop
          \\\qquad
          Y
	\end{array}
	}\right)]
     	\end{array}
  }\right)}]
  \end{array}
\]
%
\finex
\end{example}



%%% Local Variables:
%%% mode: latex
%%% TeX-master: "main"
%%% End:

%%
Let $\locset$ be a set of \emph{localities} and $\asort$ range over
basic sorts which include $\asort[int]$, $\asort[bool]$,
$\asort[str]$, etc.
%
Tuple types have the following syntax:
\begin{eqnarray*}
  \atuple & \bnfdef & \asort \bnfmid
                      \nu \aloc \bnfmid
                      x : \asort \bnfmid
                      \atuple \cdot \atuple \bnfmid
                      \wildcard \bnfmid
\end{eqnarray*}
%
Tuple types \emph{match} according to the relation $\_ \matches \_$ defined as
\[
  \atuple \matches \atuple' \iff
  \begin{cases}
    \atuple = \atuple'  & \text{if $\atuple$ and $\atuple'$ are sorts}
    \\
    \atuple_1 \matches \atuple'_1
    \land \atuple_2 \matches \atuple'_2  & \text{if } \atuple = \atuple_1 \cdot \atuple_2 \text{ and }  \atuple' = \atuple'_1 \cdot \atuple'_2
    \\
    \atuple = \wildcard \vee \atuple' = \wildcard & \text{otherwise}
  \end{cases}
\]

We define
\[
  \atuple \clashes \atuple' \iff \exists \atuple'' \qst \atuple'' \matches \atuple \land \atuple'' \matches \atuple'
\]
%
and say that $\atuple$ and $\atuple'$ \emph{are in conflict} when
$\atuple \clashes \atuple'$.

We fix two disjoint sets $\unitroles$ and $\multiroles$, respectively
of \emph{unique} and \emph{multiple} roles, and define the set of
\emph{roles}
$\roleset = \unitroles \cup \multiroles \cup \{\unknownstar,
\unknown\}$ (ranged over by $\arole, \arole', \ldots$) where
$\unknownstar$ and $\unknown$ are 'wildcards' representing any roles
and at least one role.
%
We conventionally write multiroles with initial uppercase letter and
unique roles with initial lowercase letter.

Global types $\aK$ have the following syntax
\eMnote{non e' chiaro se vogliamo $\aK \parop \aK $}
\begin{eqnarray*}
  \aK & \bnfdef & \asum \bnfmid
                  \aK \parop \aK \bnfmid
                  X \bnfmid
                  \arec
  \\
  \apref & \bnfdef & \apref[\arole] \bnfmid
                         \apref[{}][\arole] \bnfmid
                         \apref[{}][\arole][@][@][.] \bnfmid
                         \apref[\arole][\arole'] \bnfmid
                         \apref[\arole][\arole'][@][\aloc][.]
\end{eqnarray*}
where $I$ is a finite set of indexes and $\aloc \in \locset$ is a
locality.

Local types $\aL$ have the following syntax
\begin{eqnarray*}
  \aL & \bnfdef & \nil \bnfmid
                  \aL \chop \aL \bnfmid
                  \aout[].\aL \bnfmid
                  \ain[].\aL \bnfmid
                  \ard[].\aL \bnfmid
                  X \bnfmid
                  \arec[@][\aL]
\end{eqnarray*}

The projection $\proj$ of a global type $\aK$ on a role $\arole$ is
defined as
\[
  \proj =
  \begin{cases}
    & \text{if } \\
    & \text{if } \\
    & \text{if } \\
    X & \text{if } \aK = X \\
    \arec[@][{\proj[\aK']}] & \text{if } \aK = \arec[@][\aK']
  \end{cases}
\]


%%% Local Variables:
%%% mode: latex
%%% TeX-master: "main"
%%% End:


\subsection{Semantics of global types}
\label{sec:globsem}
% !TEX root =  main.tex

We give semantics to global types using \emph{partially-ordered
  multi-sets} (pomsets for short).
%
Following~\cite{gaifman1987partial}, a \emph{pomset} is an isomorphism
class of labelled partially-ordered sets (lposet) where, fixed a set
of labels $\lset$,
\begin{itemize}
\item an lposet is a triple $(\eset, \leq, \alf)$, with $\eset$ a set
  of events, $\leq$ is a partial order on $\eset$, and
  $\alf: \eset \rightarrow \lset$ a labelling function mapping events
  in $\eset$ to labels in $\lset$;
\item two lposets $(\eset, \leq, \alf)$ and $(\eset', \leq', \alf')$
  are \emph{isomorphic} if there is a bijection
  $\phi: \eset \rightarrow \eset'$ such that
  $\ae \leq \ae' \iff \phi(\ae) \leq' \phi(\ae')$ and
  $\alf = \alf' \circ \phi$.
\end{itemize}
%
Intuitively, the partial order $\leq$ yields a causality relation
among events; for $\ae \neq \ae'$, if $\ae \leq \ae'$ then $\ae'$ is
caused by $\ae$ or, in other words, the occurrence of $\ae'$ must be
preceded by the one of $\ae$ in any execution respecting the order
$\leq$.
%
Note that $\alf$ is not required to be injective: for
$\ae \neq \ae' \in \eset$, $\alf(\ae) = \alf(\ae')$ means that $\ae$
and $\ae'$ model different occurrences of the same action.
%
In the following, $[\eset, \leq, \alf]$ denotes the isomorphism class
of $(\eset, \leq, \alf)$, symbols $\apom,\apom', \dots$ (resp.
$\aR, \aR', \dots$) range over (resp. sets of) pomsets, and we assume
that pomset $\apom$ contains at least one lposet which will possibly
be referred to as $(\esetof \apom$, $\leqof \apom, \alfof \apom)$.
%
The empty pomset is denoted as $\emptypom$.

An event $\ae$ is an \emph{immediate predecessor} of an event $\ae'$
(or equivalently $\ae'$ is an \emph{immediate successor} of $\ae$) in
a pomset $\apom$ if $\ae \neq \ae'$, $\ae \leqof \apom \ae'$, and for
all $\ae'' \in \esetof \apom$ such that
$\ae \leqof \apom \ae'' \leqof \apom \ae'$ either $\ae = \ae''$ or
$\ae' = \ae''$.
%
We will represent pomsets as (a variant\footnote{Edges of Hasse
  diagrams are usually not oriented; here we use arrows so to draw
  order relations between events also horizontally.} of) Hasse
diagrams of the immediate predecessor relation; for instance, the
pomset
%\todo{{RB: anche $(\ae_1,\ae_5)$ nel pomset (non nelle figure ovviamente)?}}
\[
  \left[\{\ae_1,\ae_2,\ae_3,\ae_4,\ae_5\}, \{(\ae_1,\ae_2),(\ae_1,\ae_3),
    (\ae_1,\ae_4), (\ae_1,\ae_5), (\ae_4,\ae_5)\},
    \alf% :
    % \begin{cases}
    %   \ae_1 \mapsto \apref[\ptp p][]
    %   \\
    %   \ae_2,\ae_3 \mapsto \apref[][{\amulti q}]
    %   \\
    %   \ae_4 \mapsto \apref[\amulti R][][@][@][.]
    %   \\
    %   \ae_5 \mapsto \apref[][\amulti R][@][@][.]
    % \end{cases}
  \right]
\]
is more conveniently written as
\[
  \pomsetrep{
    \node (out) {$\ae_1$};
    \node[below left = of out] (in1) {$\ae_2$};
    \node[below right = of out] (in2) {$\ae_3$};
    \node[above right = of in2] (out2) {$\ae_4$};
    \node[below = of out2] (in3) {$\ae_5$};
    \draw[->] (out) -- (in1);
    \draw[->] (out) -- (in2);
    \draw[->] (out) -- (out2);
    \draw[->] (out2) -- (in3);
  }[\alf][][node distance = .5cm and .3cm]
  \qquad\text{or}\qquad
  \pomsetrep{
    \node (out) {$\alf(\ae_1)$}; % {$\apref[{\ptp p}][]$};
    \node[below left = of out] (in1) {$\alf(\ae_2)$}; % {$\apref[][{\amulti q}]$};
    \node[below right = of out] (in2) {$\alf(\ae_3)$}; % {$\apref[][{\amulti q}]$};
    \node[above right = of in2] (out2) {$\alf(\ae_4)$}; % {$\apref[\amulti R][][@][@][.]$};
    \node[below = of out2] (in3) {$\alf(\ae_5)$}; % {$\apref[][\amulti R][@][@][.]$};
    \draw[->] (out) -- (in1);
    \draw[->] (out) -- (in2);
    \draw[->] (out) -- (out2);
    \draw[->] (out2) -- (in3);
  }[][][node distance = .5cm and .1cm]
\]

  

% Given an autonomous prefix $\apref$, we let $\pomsetsingle$ to be
% the pomset with a single event labelled by $\apref$.
%
%We will use the auxiliary operations on pomsets described below.
%
% Define the (disjoint) union of two pomsets $\apom$ and $\apom'$ as
% \[
%   \apom \pomsetcup \apom' =
%   [\esetof{\apom} \uplus \esetof{\apom'},
%   \leqof{\apom} \uplus \leqof{\apom'},
%   \alfof{\apom} \uplus \alfof{\apom'}]
% \]


In the definition of our semantics we follow a principle that
distinguishes the nature of autonomous and interaction prefixes.
\begin{itemize}
\item A tuple type $\atuple$ generated by an autonomous output can
  be accessed by any instance of \emph{any} other role.
  %
  However, there is no obligation to access the tuple $\atuple$, hence
  our semantics has to contemplate the cases where no read or input of
  $\atuple$ happens.
\item Interactions are slightly more subtle.
  %
  Firstly, a tuple type $\atuple$ in an read-only interaction is meant
  to be eventually accessed by (an instance of) the receiving role.
  %
  Secondly, the tuple type $\atuple$ of a consuming interaction must
  be eventually consumed by an instance of the receiving role.
  %
  Thirdly, if $\atuple$ is in a consuming interaction any instance of
  the receiving role is allowed to read $\atuple$.
\end{itemize}

To capture this semantics we label events with 
autonomous prefixes $\apref$, possibly decorated as $\alabel$.
%
Intuitively, e.g., a label $\alabel[@][][\arole][@][@][.]$
(resp. $\alabel[@][\arole][][@][@][.]$) represents the fact that the
$i^\mathit{th}$ instance of $\arole$ reads (resp. produces) a tuple of
type $\atuple$.
%
Labels $\apref$ not prefixed with $[\_]$ simply specify that the event
can be performed by any instance of the role in $\apref$.
% 
Hereafter, we only deal with pomsets labelled  as  above.
%
\todo{eM: qui}
%
%
% Finally, for $h \geq 1$ and a pomset $\apom$ we define
% \[
%   \apom^h =
%   \begin{cases}
%     r & \text{if } h = 1
%     \\
%     r \pomsetcup r^{h-1} & \text{if } h > 1
%   \end{cases}
% \]
% The operation $\_^h$ extends element-wise to sets of pomsets.
%
%
Also, we assign \emph{basic pomsets} $\bp$ to prefixes $\apref$, a
basic pomset yields the causal relations of $\apref$ imposed by the
design principle described above.
% 
For an autonomous prefix $\apref[@][@][][]$ we define
$\bp = \left\{ \pomsetrep{\node {$\alabel$}}[] \right\}$.
%
For interaction prefixes we define
%
\begin{align*}
  \bp[@][{\apref[\arole][\arole'][@][@][.]}] =
  &
    \bigcup_{h \geq 1}
    \left\{
    \pomsetrep{
    \node (out) {$\alabel[@][\arole][][\arole' \cdot \atuple][@][.]$};
    \node[below left = of out] (rd1) {$\ae_1$};
    \node[below = of out] (dots) {};
    \node[below right = of out] (rd2) {$\ae_h$};
    \draw[->] (out) -- (rd1);
    \draw[->] (out) -- (rd2);
    \draw[dotted] (rd1) -- (rd2);
    }[\alf][][node distance = 1cm and -.25cm]\right\}
  \\
  \bp[@][{\apref[\arole][\arole']}] =
  &
    \bigcup_{h \geq 1}\left\{
    \pomsetrep{
    \node (out) {$\alabel[@][\arole][][\arole' \cdot \atuple]$};
    \node[below left = of out] (rd1) {$\ae_1$};
    \node[below = of out] (dots) {};
    \node[below right = of out] (rd2) {$\ae_h$};
    \node[below = of dots] (in) {$\alabel[@][][\arole'][\arole' \cdot \eraseB\atuple]$};
    \draw[->] (out) -- (rd1);
    \draw[->] (out) -- (rd2);
    \draw[->] (rd1) -- (in);
    \draw[->] (rd2) -- (in);
    \draw[dotted] (rd1) -- (rd2);
    }[\alf][][node distance = 1cm and -.25cm]\right\}
    \cup
    \left\{
    \pomsetrep{
    \node (out) {$\alabel[@][\arole][][\arole' \cdot \atuple']$};
    \node[below = of out] (in) {$\alabel[@][][\arole'][\arole' \cdot \eraseB\atuple]$};
    \draw[->] (out) -- (in);
    }[]
    \right\}
\end{align*}
where each read-only event $\ae_j$ (with $1 \leq j \leq h$) is labelled as  
$\alf (\ae_j) = \apref[][\arole'][\arole' \cdot \eraseB\atuple][@][.]$  
%
%$: {\big(\ae_j \mapsto \apref[][\arole'][\arole' \cdot \atuple][@][.]\big)_{1 \leq j \leq h}}$
%
with 
$\eraseB\atuple$ the binder-free version of $\atuple$. 
Formally, $\eraseB{\_}$ is defined  such that 
$\eraseB{(\nu x:\asort)} = x : \asort$,  it is the identity on $\asort$, $\wildcard$ and $x : \asort$ and it behaves 
homomorphically over $\_\cdot\_$. 
%
Note that the tuples in the labels of the events are \quo{prefixed} by the 
role $\arole'$ meant to access  them; this requires to extend $\asort$ so to
include  $\roleset$.



We can now give the semantics of prefixes as follows
\begin{align*}
  \ksem{\apref} =
  &
    \begin{cases}
      \bp[1][\apref]
      & \text{if $\apref$ autonomous } \land \roles{\apref} \subseteq \participants
      \\
      \bigcup_{i \geq 1}\bp[@][\apref]
      % \bigcup_{h \geq 1}\big\{\pomsetrep{
      % \node {$\alabel[1]\cdots\alabel[h]$};
      % }[]\big\},
      & \text{if $\apref$ autonomous } \land \roles{\apref} \not\subseteq \participants
    \end{cases}
%  \qquad \text{if $\apref$ autonomous}
  \\
  \ksem{\apref[\arole][\arole'][@][@][.]} =
  &
    \begin{cases}
      \bp[1][{\apref[\arole][\arole'][@][@][.]}]
      & \text{if } \arole \in \participants
      \\
      \bigcup_{i \geq 1}\bp[@][{\apref[\arole][\arole'][@][@][.]}]
      & \text{otherwise}
    \end{cases}
  \\
  \ksem{\apref[\arole][\arole']} =
  &
    \begin{cases}
      \bp[1][{\apref[\arole][\arole']}]
      & \text{if } \arole \in \participants
      \\
      \bigcup_{i \geq 1}\big(\bp[@][{\apref[\arole][\arole']}]\big)
      & \text{otherwise}
    \end{cases}
\end{align*}

As customary in other
choreographic approaches~\cite{},  the semantics of (closed) global types
considers only  \emph{well-formed} global
types, namely those enjoying \emph{well-sequencedness} and
\emph{well-branchedness}.
%
With respect to standard notions, however, these concepts have some
peculiarities which we now discuss.

%
% 
%\eMnote{{quando $\leq = \leqof{\apom} \uplus \leqof{\apom'}$ abbiamo il parallelo}}
%
% Finally, for $h \geq 1$ and a pomset $\apom$ we define
% \[
%   \apom^h =
%   \begin{cases}
%     r & \text{if } h = 1
%     \\
%     r \pomsetcup r^{h-1} & \text{if } h > 1
%   \end{cases}
% \]
% The operation $\_^h$ extends element-wise to sets of pomsets.

\bigskip

The key points of well-sequencedness are highlighted in 
the following type
\begin{align}\label{eq:seq}
  \apref[\arole_1][\arole_2][{\asort[str] \cdot \wildcard}][\aloc] \seqop
  \apref[\arole_2][\arole_3][{\asort[str] \cdot \asort[int]}][\aloc]
\end{align}
%
where an instance of $\arole_2$ transforms a pair generated by
$\arole_1$ into a pair for $\arole_3$.
%
The choreography \eqref{eq:seq} may be violated when
$\arole_1$ generates a tuple of type $\asort[str] \cdot \asort[int]$.
%
In fact, such a tuple could match the type consumed by $\arole_3$ and
therefore $\arole_3$ could \quo{steal} the tuple from $\arole_2$.
%
The problem is due to the fact that the tuples are generated on the same location and they match each other.
%
More generally, the problem arises when different interactions introduce races on tuple types. Formally, write $(\atuple, \aloc) \in \aK$ when there is a
prefix in $\aK$ whose tuple type is $\atuple$ and whose location is
$\aloc$; we say that $(\atuple, \aloc)$ is \emph{local} to $\aK$ if
either of the following holds:
\begin{itemize}
\item $\aK = \asum[@][@][][]$ and there is $i \in I$ such that either
  $(\atuple, \aloc)$ is local to $\aK_i$ or $\apref_i$ outputs
  $\atuple$ at $\aloc$, there is $\atuple'$ in an input from
  $\aloc$ in $\aK$,
\todo{RB: $\aK_i$?}
 and $\atuple \matches \atuple'$
\item $\aK = \aK_1 \seqop \aK_2$ and either $(\atuple, \aloc)$ is local to
  $\aK_1$ or   $(\atuple, \aloc)$ is local to $\aK_2$
\item $\aK = \grec[@][@][\aK']$ and $(\atuple, \aloc)$ is local to
  $\aK'$.
% \todo{AC+RB: sintassi della ricorsione cambiata? A cosa serve $\aK'$?}
\end{itemize}
%
Our notion of well-sequencedness requires that there are no races
on tuple types: a choreography $\aK_1 \seqop \aK_2$ is
\emph{conflict-free} if for $i \neq j \in \{1,2\}$
%
\begin{itemize}
\item for all $(\atuple,\aloc)$ local to $\aK_i$ and for all
  $(\atuple', \aloc) \in \aK_j$, $\atuple \matches \atuple'$ implies
  $(\atuple', \aloc)$ is in a read-only prefix in $\aK_j$
\item for all $(\atuple,\aloc)$ in an autonomous input prefix of $\aK_i$ and for all $(\atuple', \aloc)$ in an
  output prefix of $\aK_j$, $\atuple \matches \atuple'$ implies
  $(\atuple',\aloc)$ is in a consuming
\todo{RB: togliere ``consuming''? (la terminologia \`e cambiata)}
 output prefix in $\aK_j$.
\end{itemize}
% such that $\atuple_i$ occurs in a prefix of $\aK_i$ both at a
% locality $\aloc$, \eMnote{Problema: $\aloc$ vs $\alocvar$}

We introduce the auxiliary operation $\rseq[][]$ that sequentially
composes pomsets.
%
The sequential composition of $\apom$ and $\apom'$ has to make the
actions of a role in $\apom$ to precede its actions in $\apom'$ and
to make
autonomous outputs in $\apom$ to precede autonomous
matching accesses in $\apom'$.
% \quo{copy} of $\apom'$ with every maximal event of $\apom$.
%
Formally, an \emph{unmatched output of a pomset $\apom$} is an
event $\ae \in \esetof \apom$ such that $\alfof \apom(\ae)$ is an
output with tuple type $\atuple$ and for all $\ae'$ successors of
$\ae$ in $\apom$, if $\alfof{\apom}(\ae')$ is an input with tuple type
$\atuple'$ then $\atuple \matches \atuple' = \textit{undef}$;
similarly, call \emph{unmatched input of $\apom$} an event
$\ae \in \esetof \apom$ such that $\alfof \apom(\ae)$ is an output
with tuple type $\atuple$ and for all $\ae' \leqof \apom \ae$ in
$\apom$, if $\alfof \apom (\ae')$ is an input with tuple type
$\atuple'$ then $\atuple \matches \atuple' = \textit{undef}$.
%
% for each $\ae \in \max \apom$, let
% $\apom'_{\ae} = [\{\ae\} \times \eset_{\apom'}, \leq, \alf]$ where
% $(\ae,\ae_1) \leq (\ae,\ae_2) \iff \ae_1 \leqof{\apom'} \ae_2$ and
% $\alfof{\apom'_{\ae}} = (\ae,\ae') \mapsto \alfof{\apom'}(\ae')$ for
% all $\ae' \in \esetof{\apom'}$.
% %
% Observe that $\apom'_{\ae}$ is isomorphic to $\apom'$, and
Let $\apom$ and $\apom'$ such that
$\esetof \apom \cap \esetof{\apom'} = \emptyset$,
recall that the
labels of events are autonomous prefixes for which $\roles{}$ is a
singleton,
%
\todo{AC: dove \`e scritto?}
%
and define
\[
  \rseq[\apom][\apom'] = 
  \bigcup_{\scriptsize
    \begin{array}{c}
      \ae \text{ unmatched output of $\atuple$},
      \\
      \ae' \text{ unmatched input of $\atuple'$},
      \\
      \atuple \matches \atuple'
    \end{array}
  } \big\{
  [\esetof{\apom} \uplus \esetof{\apom'},
  \leq_{\ae,\ae'},
  \alfof{\apom} \uplus \alfof{\apom'}]
  \big\}
\]
where $\leq_{\ae,\ae'}$ is the reflexo-transitive closure of
$\leqof{\apom} \cup \leqof{\apom'} \cup \{(e,e')\}$.
%
Finally,
\begin{align*}
  \ksem{\aK_1 \seqop \aK_2} =
  &
    \begin{cases}
      \rseq[\ksem{\aK_1}][\ksem{\aK_2}] & \text{if } \ws[\aK_1][\aK_2]
      \\
      \mathit{undef} & \text{otherwise}
    \end{cases}
\end{align*}
where the well-sequencedness condition $\ws[\aK_1][\aK_2]$ holds when
$\aK_1 \seqop \aK_2$ is conflict-free and any unit role $\ptp a$
performing actions in $\aK_2$ acts in $\aK_1$ as well and all the
first actions of $\ptp a$ in $\aK_2$ causally depend on all last
actions of $\ptp a$ in $\aK_1$ in each pomset in
$\rseq[\ksem{\aK_1}][{\ksem{\aK_2}}]$.

\bigskip

% To avoid this problem we introduce \emph{run-time} tuples
We now consider well-branchedness, the other condition of
well-formedness.
%
As usual, well-branchedness requires two conditions: single selector
and knowledge of choices.
%
This can be formalised by requiring that one process in the choice is
\emph{active}, namely it selects the branch to take, while the others
are \emph{passive}, namely they are informed of the chosen branch by
inputting some information that unambiguously identifies each branch
of the choice.
%
We syntactically\footnote{
%
  This is just for simplicity as we could adopt definitions similar to
  the ones in \cite{gt16,gt17} at the cost of higher technical
  complexity.
%
} enforce uniqueness of selectors; a choice with several branches,
takes the form
\begin{align}
  \asum\label{eq:ch}
\end{align}
namely the instance of $\arole$ acts as \emph{unique} selectors.
%
Intuitively, a passive instance (for example one enacting role
$\arole_i$) in \eqref{eq:ch} has to be able to ascertain which branch
the selector decided when the choice was taken.
%
A simple way to ensure this is to require that the first input actions
of each passive roles are pairwise \quo{disjoint} (\ie\ non matching
tuples or different locations) among branches.
%
% This is just for simplicity as we could adopt more general
% definitions similar to the one based on divergence points in
% \cite{gt16,gt17}.

The conditions on active and passive processes alone are not enough:
in our framework, the notion of well-branchedness is slightly
complicated by the presence of multiple roles.
%
For instance, even assuming unique selectors, many instances of a
selector role could exercise choices concurrently.
%
This may create confusion if different branches generate matching
tuples on a locality as illustrated by next example.
\begin{example}\label{ex:nonwb}
  Let
  \begin{align*}
    \aK_\mathrm{bad} = & \apref[\amulti A][\amulti B][{\asort[int]}][\aloc] \mkop{.} K_1 \chop \apref[\amulti A][\amulti B][{\asort[str]}][\aloc] \mkop{.} K_2
    \\
    \aK_1 = & \apref[\amulti B][\amulti C][{\asort[str]}][\aloc] \mkop{.} \apref[\amulti C][\amulti B][{\asort[bool]}][\aloc]
    \\
    \aK_2 = & \apref[\amulti B][\amulti C][{\asort[bool]}][\aloc]
  \end{align*}
  In $\aK_\mathrm{bad}$ confusion may arise that may alter the
  intended data flow when two groups of instances of $\amulti A$,
  $\amulti B$, and $\amulti C$ execute the choice so that each group
  takes a different branch.
  %
  In fact, the instance of $\amulti C$ executing $K_2$ in the second
  branch may receive the boolean that the instance of $\amulti C$ in
  $K_1$ executing the first branch generates for $\amulti B$.
  %
  \eMnote{This type of confusion does not seem to introduce deadlocks}
  %
  \finex
\end{example}
%
Therefore we require that tuple types in different branches of a
choice do not match when they are at the same locality and that if a
branch of a choice involves a unit role then none of the branches of
the choice involves multiple roles.
%
This condition, dubbed \emph{confusion-free branching} ensures that
different \quo{groups} of instances involved in concurrent resolutions
of a choice do not \quo{interfere} with each other. If a unit role is involved,
only one group can resolve the choice. 
%
We remark that the above condition is not a limitation; in fact, we
can pre-process branches of choices by adding an extra field in all
tuples of the branch so to unequivocally identify on which branch the
tuple type is used.

To sum up, a choice as in~\eqref{eq:ch} is \emph{well-branched},
written $\wb[{\big\{\bigcup_{i \in I}\apref_i\mkop{.}\aK_i\big\}}]$,
when it is confusion-free, there is a unique active role, all other
roles are passive.
%
So we define
%
\begin{align*}
  % 
  \ksem{\asum[@][@][][]} =
  &
    \begin{cases}
      \{ \emptypom \} & \text{if } I = \emptyset
      \\
      \bigcup_{\apom \in \ksem{\apref_i},\apom' \in \ksem{\aK_i}} \rseq[\apom][\apom']
      & 
      \text{if } \wb[{\big\{\bigcup_{i \in I}\apref_i\mkop{.}\aK_i\big\}}]
      \\
      \mathit{undef} & \text{otherwise}
    \end{cases}
  \\
\end{align*}

\bigskip



  %
% The notion of well formedness seems to guarantee a much weaker
% notion of correctness than the usual ones.
% %
% Firstly, note that well-formedness here does not imply deadlock
% freedom (but this is fine since we are not interested in properties
% of the control of processes).
% %
%
% A example could be the following:
%
%
\eMnote{We can tackle this issue in two different way (at least): one way is
to statically ensure that tuples generated on one branch do not match
any other tuple on another branch; another way is to modify the
semantics of the choice by implicitly inserting an extra field in each
tuple with a unique identifier of each branch.
\\
Under the current interpretation of our semantics, probably the notion
of correctness we can guarantee is that any set of instances taking a
choice will fully execute a branch.
}

\bigskip

Finally, the semantic equation for the recursive global type $\grec$
require some auxiliary functions:
\begin{align*}
  \mathit{STOP}(\arole, \aK, \seq y) = & \apref[\arole][\arole_1][{\asort[stop]}][y_1] \seqop \ldots \seqop \apref[\arole][\arole_n][{\asort[stop]}][y_n]
  \\
  \mathit{LOOP}(\arole, \aK, \seq y, \seq y') = & \apref[\arole][\arole_1][{\nu y'_1:\asort[loc]}][y_1)] \seqop \ldots \seqop \apref[\arole][\arole_n][{\nu y'_n:\asort[loc]}][y_n]
\end{align*}
where $\roles \aK = \{\arole, \arole_1, \ldots, \arole_n\}$ with
$\arole \not\in \{\arole_1, \ldots, \arole_n\}$ and
$\seq y = y_1 \cdots y_n$ and 
$\seq y' = y'_1 \cdots y'_n$.
%
Then, we define
\begin{align*}
  \ksem{\grec} =
  &
    \begin{cases}
      \bigcup_{h \geq 0} \ksem{\unfold h \grec {\fn{\aK}} {\seq y}  {\seq y'}} & \text{if } \ws[\aK\sust X \nil][\aK\sust X \nil]
      \\ & \text{ and } \seq y \cap \fn{\aK} = \emptyset
      \\
      \mathit{undef} & \text{otherwise}
    \end{cases}
\end{align*}
where
\[
  \unfold h \grec L {\seq y} {\seq y'} =
  \begin{cases}
    \mathit{STOP}(\arole,\seq y) & \text{if } h = 0
    \\
    \mathit{LOOP}(\arole,\aK, \seq y, \seq y') \seqop \aK \sust X {\aK'} & \text{otherwise}
  \end{cases}
\]
%
where $\aK' = \unfold{h-1}{\grec}{L \cup \seq y \cup \seq y'}{\seq y'}{\seq y''}$
with $\seq y''$ fresh.

%%% Local Variables:
%%% mode: latex
%%% TeX-master: "main"
%%% End:


\subsection{Local types}
\label{sec:locsem}

Local types $\aL$ have the following syntax
\begin{eqnarray*}
  \aLpref & \bnfdef &
                  \aout[] \bnfmid
                  \arout[]\bnfmid
                  \ain[] \bnfmid
                  \ard[] 
\\
  \aL & \bnfdef &
                  \aLsum \bnfmid
                  X \bnfmid
                  \arec[@][\aL]
\end{eqnarray*}
As for global types, we write $\nil$ for a sum with an empty set $I$ of branches. 
%
%For a local type $\aL = \aLsum$, we write $\selectors \aL$ for the selecting roles the choice, i.e.,
%$\selectors \aL = \cup_{i\in I}{\{\arole_i\}}$.
%%
%Moreover, we assume
%\begin{itemize}
%\item $\aL_i \neq\aLsum[j][J]$, for all $i \in I$.
%\item $\arole_i \eqR \arole_j$ for all $i,j \in I$.
%\end{itemize}
%

We consider the following syntax for run-time semantics of a set of local types and tuple spaces, 
called {\em specification}.

\begin{eqnarray*}
  \envmv & \bnfdef & \emptyset \bnfmid
                  \envmv, \arole :  \aL \bnfmid
                  \envmv, \atupleat \bnfmid
                  \envmv, \artupleat                  
\end{eqnarray*}

%In $\arole :  \aL$  we assume
%$\arole \in \participants \cup \multiroles$  and $\arole \eqR \arole'$ for any $\arole'$
%occurring in $\aL$.
%
We write $\envtuple$ for a specification that only contains terms of the form $\atupleat$. 
 
\[
\begin{array}{l@{\hspace{1cm}}l}
\mathaxiom
	{\envmv, \arole : \aout[].\aL 
	 \red[\arole : {\aout[]}]
	 \envmv, \arole : \aL , \atupleat}
	{LOut_1}
\\[25pt]
\mathaxiom
	{\envmv, \arole : \arout[].\aL 
	 \red[\arole : {\arout[]}]
	 \envmv, \arole : \aL , \artupleat}
	{LOut_2}
\\[25pt]
\mathrule
	{\atuple \matches \atuple'}
	{\envmv, \arole : \ain[].\aL, \atupleat[][\atuple'] 
	 \red[\arole : {\ain[][\atuple']}]
	 \envmv, \arole : \aL}
	{LIn}
\\[25pt]
\mathrule
	{\atuple \matches \atuple'}
	{\envmv, \arole : \ard[].\aL, \atupleat[][\atuple'] 
	 \red[\arole : {\ard[][\atuple']}]
	 \envmv, \arole : \aL,  \atupleat[][\atuple'] }
	{LRd_1}
\\[25pt]
\mathrule
	{\atuple \matches \atuple'}
	{\envmv, \arole : \ard[].\aL, \artupleat[][\atuple'] 
	 \red[\arole : {\ard[][\atuple']}]
	 \envmv, \arole : \aL,  \artupleat[][\atuple'] }
	{LRd_2}
\\[25pt]
\mathrule
	{j\in I \qquad \envtuple, \arole : \aLpref_j.\aL_j \red[\alpha] \envmv'}
	{\envmv, \envtuple, \arole : \aLsum
	 \red[\alpha]
	 \envmv,\envmv'}
	{LSum}
\end{array}
\]
\todo{not sure which decoration do we want in LSum $\arole$ or $\arole_i$.}

The projection $\proj$ of a global type $\aK$ on a role
$\arole \in \participants \cup \multiroles$ is defined as
\[
  \proj =
  \begin{cases}
    \nil & \text{if } \arole \not \in \roles \aK
    \\
    \displaystyle{\sum_{i\in I}{\proj[(\apref_i.\aK_i)]}}
    &
    \text{if } \aK = \asum
    \\
    \proj[\aK_1]
    &
    \text{if } \aK = \apref.\aK_1 \text{ and } \rho\not\in\roles{\apref}
    \\
    \aout[].(\proj[\aK_1])
    &
    \text{if } \aK = \apref[\arole].\aK_1
    \\
    \arout[].(\proj[\aK_1])
    &
    \text{if } \aK = \apref[\arole][{}][@][@][.].\aK_1
    \\
    \ain[].(\proj[\aK_1])
    &
    \text{if } \aK = \apref[{}][\arole].\aK_1
    \\
    \ard[].(\proj[\aK_1])
    &
    \text{if }  \aK = \apref[{}][\arole][@][@][.].\aK_1
    \\
    \aout[].(\proj[\aK_1])
    &
    \text{if } \aK = \apref[\arole][\arole'].\aK_1
    \\
    \ain[].(\proj[\aK_1])
    &
    \text{if } \aK = \apref[\arole'][\arole].\aK_1
    \\
    \arout[].(\proj[\aK_1])
    &
    \text{if } \aK = \apref[\arole][\arole'][@][\aloc][.].\aK_1
    \\
    \ard[].(\proj[\aK_1])
    &
    \text{if } \aK = \apref[\arole'][\arole][@][\aloc][.].\aK_1
    \\
    X
    & \text{if } \aK = X
    \\
    \arec[@][{(\proj[\aK_1])}]
    &
    \text{if } \aK = \arec[@][\aK_1]
  \end{cases}
\]


%%% Local Variables:
%%% mode: latex
%%% TeX-master: "main"
%%% End:

  
\section{Conclusions}
\label{sec:disc}\label{sec:conc}
% !TEX root =  main.tex

Klaim has been designed to program distributed systems consisting of
processes interacting via multiple distributed tuple spaces.
%
Klaim has been extended with several features designed on theoretical
foundations~\cite{klaim} and implemented in a suite of prototypes.



%%% Local Variables:
%%% mode: latex
%%% TeX-master: "main"
%%% End:


\bibliographystyle{abbrv}
\bibliography{biblio}

\iffinal
\else
 \newpage
 \setcounter{tocdepth}{2}
 \listoffixmes
\fi

\end{document}

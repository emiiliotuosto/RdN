% !TEX root =  main.tex

Communication-centered programming is playing a tremendous role in the production of nowadays software. Programming peers that need to exchange information is an error-prone activity and the behaviour of even small systems is subject to a combinatorial blow-up as the number of peers increases.
Therefore well-structured principles and rigorous foundations are needed to develop well-engineered, trustworthy software. 
One possibility is to exploit some sort of behavioural types~\cite{DBLP:journals/csur/HuttelLVCCDMPRT16,dd09} to manage abstract descriptions of peers and formally study their properties such as communication safety, absence of deadlocks, progress or session fidelity: given the types of the peers the emerging behaviour of their composition is analysed.
In the seminal paper~\cite{DBLP:conf/popl/HondaYC08}, recently nominated the \emph{most influential POPL paper (Award 2018)}, the authors push forward an abstract notion of global type of interaction that represents a sort of contract between the communicating peers. This is paired with the notion of local type that gives an abstract description of the behaviour of each peer, as taken in isolation.
Interestingly, local types can be obtained for free by projection from global types, while the properties of interest can be studied and guaranteed just at the level of global types, without the need of studying the composition of local types. The conformance of peers implementation w.r.t. the global type can be studied instead at the level of local types, allowing a more efficient form of type checking. Roughly this means that properties are stated globally but checked locally. Global types have been inspired by choreography languages in service oriented computing, where complex interactions are modelled from the point of view of the global sequence of events that must take place in order to successfully complete the computation.

In the literature, global/local types have been studied mostly in the
context of point-to-point channel-based interactions. This means that
the main action in a choreography is the sending of a message from one
peer to another on a specific channel (of a given type). In this paper we
explore a different setting, where interaction over tuple-spaces
replaces message passing, in the style of Linda-like
languages~\cite{DBLP:journals/toplas/Gelernter85}.  Instead of
primitives for sending and receiving messages, here there are
primitives for inserting a tuple on a tuple space, for reading
(without consuming) a tuple from a tuple space or for retrieving a
tuple from a tuple space. We call these interactions data-driven, as
decisions will be taken on the basis of the type of the tuples that
are manipulated. We coined the term Klaimographies in honour of the
process language Klaim~\cite{DBLP:journals/tse/NicolaFP98,klaim}, one
of the main contributions of Rocco De Nicola in the fields of process
algebras and distributed programming. Inspired by Klaim,
Klaimographies exploit the notion of distributed tuple-space
localities to separate the access to data on the basis of the
interactions that are carried out. Localities can be communicated in
tuples, allowing name mobility.

There are several major advances w.r.t. to the literature on global types.
First, Klaimographies can have an arbitrary number of participants, while in global types the number of participants is usually fixed a priori.
Second, interactions are not point-to-point because each tuple can be read many times, while messages have exactly one producer and one consumer, even when asynchronous communication is considered.
Third, all interactions involve a tuple space locality instead of a channel name.
Fourth, Klaimographies are data-driven.

The main contribution of this paper is to set up the formal setting of Klaimographies and prepare the ground for several interesting research directions: we fix the syntax of global and local type description and define the projection from global type to local types as typical of choreographic frameworks.
Global types are equipped with a partial order semantics of events and local types with an ordinary operational semantics in the SOS style. Then the conditions under which the behaviour of projected local types are faithful to the semantics of global types are spelled out. 

Shifting the focus from control to data in choreographic framework has
several implications.
%
Firstly, the emphasis is no longer on properties related to
computational actors.
%
For instance, Klaimographies admit computations where some processes
may not terminate and is left waiting for some data.
%
In standard choreographic framework those would be undesired behaviour
to rule out with suitable typing disciplines.
%
Nonetheless, we claim that in some application domain computations with
deadlocked processes have to be considered non-erroneous.
%
For instance, in reactive systems based on event-notification
frameworks some \quo{listener} components must be kept waiting for
events to occur.
%
Our work paves the way to the formal study of properties of data, like consumption, persistence and availability, in a choreographic setting.

Another main innovation of Klaimographies is that they allow us to easily
represent protocols where an arbitrary number of components can
play a role in the choreography.
%
We give an example of such protocol in \cref{sec:examples}.
%
Remarkably, those protocols can be specified in some existing
choreographic frameworks~\cite{ydbh10,chjny19}, but in a less abstract way
that requires the explicit quantification on components.

\paragraph{Structure of the paper}
After some preliminaries in Section~\ref{sec:tuples}, we define Klaimographies as global types in Section~\ref{sec:gt} and give some examples in Section~\ref{sec:examples}.
In Section~\ref{sec:globsem} we define the semantics of global types and give the adequacy conditions for projecting global types to local types.
In Section~\ref{sec:locsem} we define the operational semantics of local types and state its correspondence with the global semantics.
Some concluding remarks together with the discussion of related and future work are in Section~\ref{sec:conc}.

%%% Local Variables:
%%% mode: latex
%%% TeX-master: "main"
%%% End:


% !TEX root =  main.tex


This paper, a modest attempt to thank Rocco for his work and
friendship, addresses the following question:
%
\begin{quote}
  What notion of behavioural types corresponds to Linda-based
  coordination mechanisms?
\end{quote}
%
To answer such question we advocate Klaim-based global and local
types, dubbed klaimographies.
%
Klaim has been designed to program distributed systems consisting of
processes interacting via multiple distributed tuple spaces.
%
Klaim has been extended with several features designed on theoretical
foundations and implemented in a suite of
prototypes~\cite{klaim}.
%
On the one hand, klaimographies share similarities with standard
behavioural types centred on point-to-point channel based
communications, on the other hand they also have some peculiarities,
some of which we highlighted here.



The closest work to our is~\cite{chjny19} which, develops the initial
proposal on parameterised choreographies in~\cite{ydbh10,dybh12}.
%
Notably,~\cite{chjny19} is the first work to support indexed roles and
to statically infer the participants inhabiting them.
%
The main difference with the approach in~\cite{chjny19} is that
klaimographies do not focus on processes rather on data.
%
We envisage behavioural types as specifications of to guarantee
general properties of tuple spaces.
%
For instance, take the market place example (cf. \cref{ex:market} in
\cref{sec:examples}),
%
One would like to check properties such as
\begin{quote}
  for each tuple type
  $\atuple = {{i : \asort[str]} \cdot {p : \asort[int]} \cdot {\nu l :
      \asort[loc]}}$ consumed from locality $\aloc[m]$ either a tuple type
  $\asort[sold]$ is eventually generated at locality $l$ or $\atuple$
  is eventually generated at $\aloc[m]$.
\end{quote}
%
Remarkably, such property does not concern typical properties
controlled by behavioural types (e.g., progress of processes, message
orphanage, or unspecified reception).

As scope for future work, we aim to characterise the (classes of)
properties of interest klaimographies enforce.
%
We conjecture that the well-formedness conditions defined here
are enough to guarantee the property above.
%
Another interesting line of research is to identify typing principles
for klaim systems.
%
We believe that klaimographies can enable the possibility that a same
process enacts different roles.
%
For instance, considering again the market place example, a process
can act both as seller and as buyer.

In this paper, we adopted a few simplifying assumptions.
%
Other variants appear to be rather interesting.
%
For instance, guards of sums could be autonomous inputs and not just
consuming interactions, or even read-only access prefixes.
%
Relaxing the constraint that read-only tuples cannot generate, would
lead to a sort of multi-cast mechanism of fresh localities.
%
We plan to study those variants in future work.


%%% Local Variables:
%%% mode: latex
%%% TeX-master: "main"
%%% End:

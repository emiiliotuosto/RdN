% !TEX root =  main.tex
%

We consider 

We consider a set of localities $\locset$ ranged over by $\aloc$ (and
use $\alocvar$ to range over $\locset\cup\varset$) and we let $\asort$
range over basic sorts which include $\asort[int]$, $\asort[bool]$,
$\asort[str]$ and sort $\asort[loc]$ of \emph{localities}.
%
The set $\tupleset$ of \emph{(tuple) types} consists of the terms
derived from the following grammar:
\begin{eqnarray*}
  \atuple & \bnfdef & \asort \bnfmid
                      x : \asort \bnfmid
                      \nu x : \asort  \bnfmid
%                      \nu \aloc \bnfmid
%                      \aloc \bnfmid
                      \atuple \cdot \atuple \bnfmid
                      \wildcard
\end{eqnarray*}
Fields $\nu x : \asort$ are binders that \emph{define}
$x \in \varset$.
%
Hence, we talk about \emph{free} and \emph{defined} (sorted) names occurring in
tuples.
%
%We assume that in a tuple there is at most one occurrence of $\nu x$,
%for each $x \in \varset$.
%
Formally, the functions $\fn{\_}$ and $\dn{\_}$ returns sets of pairs
$x \mapsto \asort$ assigning sort $\asort$ to $x \in \varset$.
\[
\begin{array}{l@{\hspace{1cm}}l}
\begin{array}{lcl}
  \dn\asort & = & \emptyset
  \\
  \dn{x : \asort} & = & \emptyset
  \\
  \dn{\nu x : \asort} & = & \{x \mapsto \asort \}
  \\
  \dn{\atuple_1 \cdot \atuple_2} & = & \dn{\atuple_1} \cup \dn{\atuple_2}
  \\
  \dn{\wildcard} & = & \emptyset
\end{array}
&
\begin{array}{lcl}
  \fn\asort & = & \emptyset
  \\
  \fn{x : \asort} & = &  \{x \mapsto \asort\}
  \\
  \fn{\nu x : \asort} & = & \emptyset
  \\
  \fn{\atuple_1 \cdot \atuple_2} & = & \fn{\atuple_1} \cup \fn{\atuple_2}
  \\
  \fn{\wildcard} & = & \emptyset
\end{array}
\end{array}
\]
We write $\supp{\_}$ to denote the projection of a set of pairs over its first component. 

We say a tuple $\atuple$ is {\em well-sorted} if the following two
conditions hold:
\begin{itemize}
\item
  $\supp{\fn\atuple} \cap \supp{\dn{\atuple}} = \emptyset$, i.e., free and
  defined names are disjoint; and
\item
  $\atuple = \atuple_1 \cdot \atuple_2$ implies $\atuple_1$ and
  $\atuple_2$ well-sorted and $\supp{\dn{\atuple_1}} \cap \supp{\dn{\atuple_2}} =
  \emptyset$, i.e., defined names are all different.
%  \todo{RB: non \`e conseguenza del fatto che ``We assume that in a tuple there is at most one occurrence of $\nu x$, for each $x \in \varset$''}
\end{itemize}
%
Hereafter, we assume all tuples to be well-sorted.
%
Note that $\fn{\atuple}$ and $\dn{\atuple}$ are partial functions for
well-sorted tuples.
\todo{RB: in che senso funzioni parziali? perch\'e assegnano sort a nomi? HM: Si, in questo senso.}

A capture-avoiding
\todo{RB: toglierei capture-avoiding aggiungendo condizioni esplicite su $x,y,\atuple$.}
substitution of the free occurrences of a variable
$x$ in a (well-sorted) tuple $\atuple$ such that $x\not\in \dn{\atuple}$
\todo{RB: aggiunto}
 by a variable $y\not\in \dn{\atuple}$, written
$\atuple \sust x y$, is defined such that
%
\[
\begin{array}{r@{}l@{\ = \ } ll}
%\asort
%&
%\sust x y  
%&  
%\asort
%\\
%(x : \asort)
%& 
%\sust x y  
%& x : \asort
%\\
%z
%&
%\sust x y 
%&  
%z 
%& 
%{\it if} z\neq x
%\\
(x  : \asort)
&
\sust x y  
&  
y : \asort
\\
(\atuple_1 \cdot \atuple_2)
&
\sust x y  
& 
(\atuple_1\sust x y) \cdot (\atuple_2\sust x y) 
%& 
%{\it if} y\not\in\dn{\atuple_1 \cdot \atuple_2}
%\\
%\wildcard
%&
%\sust x y  
%&  
%\wildcard
\end{array}
\]
%
and it is the identity on the remaining cases. Let
$\sigma = \{y_1/x_1,\ldots,y_n/x_n\}$ such that $x_i\neq x_j$ for all
$i\neq j$ (i.e., a partial endo-function on $\varset$), we write
$\atuple\sigma$ for the simultaneous substitution of each $x_i$ by
$y_i$.
%
We use $\Sigma$ for the set of all substitutions. We write
$\sigma_1\sigma_2$ for the composition of partial functions with
disjoint domain, and $\sigma_1[\sigma_2]$ for the update of $\sigma_1$
with $\sigma_2$.


Tuple types \emph{match} by producing a substitution; formally given
by the partial function
$\matches : \tupleset \times \tupleset \to \Sigma$ defined such as
%\eMnote{add typing env}
\[
  \atuple \matches \atuple' \generates
    \begin{cases}
     \emptyset
    & 
    \text{if  } \atuple = \wildcard \vee \atuple' = \wildcard  \vee (\atuple = \atuple' \land \atuple\in\{\asort, x : \asort\})
%    \\
%    \sigma_1\sigma_2
%    &
%    \text{if } \atuple = \atuple_1 \cdot \atuple_2
%    \land  \atuple' = \atuple'_1 \cdot \atuple'_2
%    \land \atuple_1 \matches \atuple'_1 \generates \sigma_1
%    \land \atuple_2 \matches \atuple'_2 \generates \sigma_2
    \\
    \sigma
    &
    \text{if } \atuple = \atuple_1 \cdot \atuple_2
    \land  \atuple' = \atuple'_1 \cdot \atuple'_2
    \land \atuple_1 \matches \atuple'_1 \generates \sigma_1
    \land \atuple_2\sigma_1 \matches \atuple'_2\sigma_1 \generates \sigma
    \\
    \sust x y 
    &
    \text{if  } (\atuple = \nu x : \asort \land \atuple' = y: \asort) \vee  (\atuple' = \nu y : \asort \land \atuple = x: \asort) 
    \\
    \textit{undef} & \textit{otherwise}
   \end{cases}
\]
\todo{RB: cambiato caso $\atuple = \atuple_1 \cdot \atuple_2$}

%
%
%\[
%  \atuple \matches \atuple' \iff
%  \begin{cases}
%    \atuple = \atuple'  
%    & 
%    \text{if $\atuple\in\{\asort, x : \asort, x, \nu \aloc, \aloc\}$}
%    \\
%    \atuple_1 \matches \atuple'_1
%    \land \atuple_2 \matches \atuple'_2
%    &
%    \text{if } \atuple = \atuple_1 \cdot \atuple_2
%    \text{ and }  \atuple' = \atuple'_1 \cdot \atuple'_2
%    \\
%    \atuple = \wildcard \vee \atuple' = \wildcard & \text{otherwise}
%  \end{cases}
%\]


%
We say that $\atuple$ \emph{generates} when in one of its fields there
is a $\nu x: \asort[loc]$ type. 
%\todo{RB: meglio dire $\nu \aloc: \asort[loc]$}
%

%We define
%\[
%  \atuple \clashes \atuple' \iff \exists \atuple'' \qst \atuple'' \matches \atuple \land \atuple'' \matches \atuple'
%\]
%
%and say that $\atuple$ and $\atuple'$ \emph{are in conflict} when
%$\atuple \clashes \atuple'$.


%%% Local Variables:
%%% mode: latex
%%% TeX-master: "main"
%%% End:

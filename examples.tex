% !TEX root =  main.tex

We give a few simple global types (\cref{ex:cs,ex:CS,ex:CSx,ex:CSl})
to highlight some peculiarities of Klaimographies as well as a
more complex example (\cref{ex:market}) to illustrate the kind of
protocols our global types can capture.

\begin{example}\label{ex:cs}
  Consider the following global type that describes the interaction of
  a client $\ptp c$ with a simple service $\ptp s$ that converts
  integers into strings.
  \[
    \aK_\eqref{ex:cs} =
    \apref[\ptp c][\ptp s][{\asort[int]}][{\aloc[l]}]  \prefop
    \apref[\ptp s][\ptp c][{\asort[str]}][{\aloc[l]}]
  \]
  The client $\ptp c$ produces an integer value on the locality
  $\aloc$.
  %
  This tuple must be consumed by the server $\ptp s$, which produces
  back the converted string for the client.
  %
  \finex
\end{example}

Elaborating on the previous example we now discuss a few features of
our setting.

\begin{example}\label{ex:CS}
  Assume that we consider client and server in \cref{ex:cs} as
  multiple instead of unit roles, and write
  \[
    \aK_\eqref{ex:CS} =
    \apref[\amulti {C}][\amulti {S}][{\asort[int]}][{\aloc[l]}] \prefop
    \apref[\amulti {S}][\amulti {C}][{\asort[str]}][{\aloc[l]}]
  \]
  In this case, $\aK_\eqref{ex:CS}$ states that each integer produced by a client
  will be consumed by a server, which will in turn produce a string
  for one of the clients.
\end{example}
The type in \cref{ex:CS} does not ensure that clients consume the
string conversion of the integer they produced, because all tuples are put at the same location $\aloc[l]$.
%
Name binders can be used to correlate tuples.
%
\begin{example}\label{ex:CSx}
  Consider
  \[
    \aK_\eqref{ex:CSx} =
    \apref[\amulti {C}][\amulti {S}][{\nu x:\asort[int]}][{\aloc[l]}] \prefop
    \apref[\amulti {S}][\amulti {C}][{x : {\asort[int]} \cdot {\asort[str]} }][{\aloc[l]}]
  \]
%  $\aK_\eqref{ex:CSx}$ specifies that a fresh identifier $x$ is established between the 
% instances of $\amulti C$ and $\amulti S$ so that they refer to the same integer value.
  The first interaction binds the occurrence of $x$ in the second interaction. 
  The use of $x$ in the second interaction constraints the 
  instances of $\amulti S$ and $\amulti C$ to share a tuple 
  whose integer expression matches the integer shared in the first interaction.
  %
  Despite the identifier is known only to the communicating instances,
  this does not forbid two clients to generate the same integer value.
  %\todo{RB: questo \`e controintuitivo: quando usiamo il $\nu$ sulle locazioni non vogliamo che sia possibile generare due volte la stessa locazione, altrimenti non risolveremmo il problema. Sostituirei $\nu x$ con Solo $x$: mettere $\nu x$ costringerebbe a generare interi diversi, ma questo non \`e desiderabile}
  %
 % Basically, the name $x$ allows to express constraints about the flow
 %of values.
  %
  %In particular, $x : \asort[int]$ in the second interaction states
  %that a server must generate a tuple that contains the consumed
  %integer.
  %
  %Analogously, each client must consume a tuple matching the produced
  %integer.
  %
  \finex
\end{example}
The Klaimography in \cref{ex:CSx} does not establishes a one-to-one
association between instances of $\amulti C$ and $\amulti S$.
%
In fact, an instance of $\amulti C$ not necessarily interacts with the
same instance of $\amulti S$ in the two communications when two
instances of $\amulti C$ generate the same integer in the first
interaction.
%
\begin{example}\label{ex:CSl}
  A one-to-one correspondence can be achieved by using defined names
  for localities.
  %
  Consider
  \[
    \aK_\eqref{ex:CSl} = 
    \apref[\amulti {C}][\amulti {S}][{{\asort[int]} \cdot \nu x: {\asort[loc]}}][{\aloc[l]}]  \prefop
    \apref[\amulti {S}][\amulti {C}][{{\asort[str]}}][{x}]
  \]
  %
  As in \cref{ex:CSx}, client and server instances establish a common
  fresh identity $x$ in the first interaction; this time the identity
  is a locality meant to share tuples in subsequent communication: the
  second interaction can only take place between the two instances
  sharing $x$ since such locality is known only to them.
  %
  \finex
\end{example}

% \todo[inline]{{We think we can also deal with situations like the following, in which 
% A creates a private session for B and C. 
% %
% \[
%   \apref[\amulti {A}][\amulti {B}][{\nu y: {\asort[loc]}}][{\aloc[l]}] \prefop
% \]
% \[
%   \apref[\amulti {A}][\amulti {C}][{y : {\asort[loc]}}][{\aloc[l]}] \prefop
% \]
% \[
%   \apref[\amulti {B}][\amulti {C}][\asort][{y}]
% \]
% }}


The following example focuses on a more realistic scenario, allowing us to combine
together most of the features of our framework.
%
For readability,  we use the notation
$\grec[@][@][@][1]$ for a recursive protocol where the body $\aK$ is
repeated at least once.
%
Formally,\footnote{The reader should not be confused by the fact that the meaning of $\grec$ is different from that of $\aK \sust{X}{\grec}$: this is because iteration and termination require some implicit interactions driven by $\arole$ towards the other roles in $\aK$.}
\[
\grec[@][@][@][1] =
  \aK \sust{X}{\grec} 
\]

%For readability, given $h \geq 0$, we use the notation
%$\grec[@][@][@][h]$ for a recursive protocol where the body $\aK$ is
%repeated at least $h$ times.
%\todo{RB: visto che la notazione ci serve solo per $h=1$ possiamo semplificare definendo solo questo caso (motivandolo col fatto che nell'esempio usiamo una forma di do-while invece che di while-do)}
%
%Formally,
%\[
%\grec[@][@][@][h] =
%\begin{cases}
%  \grec[@][@][@] & \text{if } h = 0
%  \\
%  \aK \sust{X}{\grec[@][@][@][h-1]} & \text{otherwise}
%\end{cases}
%\]
%
% \eMnote{forse non e' un'asta...ma una contrattazione}
\begin{example}[Market place]\label{ex:market}
  % 
  Several sellers and buyers use a market place provided by a
  broker.
  % 
  Sellers can put on sale (several) items and buyers can inspect them.
  %
  When an item of interest is found, the client can start a negotiation
  with the seller.
  % 
  This protocol can be formalised by the following global type.
  \[
    \begin{array}{l}
      \apref[\ptp {broker}][\amulti {Seller}][{\atuple[start]}][{\aloc[m]}][.] \prefop
      \\
      \grec[][X][{\apref[\amulti {Seller}][@][{{\asort[str]} \cdot {\asort[int]} \cdot {\nu l : \asort[loc]}}][{\aloc[m]}] \prefop X}][1] \seqop
      \\  	
      \grec[\amulti{Buyer}][Y][{\left({
      \begin{array}{l}
        \grec[][Z][{\apref[@][\amulti {Buyer}][{{\asort[str]} \cdot {\asort[int]} \cdot {\asort[loc]}}][{\aloc[m]}][.] \prefop Z}] \seqop
        \\
        \apref[@][\amulti {Buyer}][{{i : \asort[str]} \cdot {p : \asort[int]} \cdot {\nu l : \asort[loc]}}][{\aloc[m]}] \prefop
        \\
        \grec[\amulti{Seller}][W][\left({
        \begin{array}{l}
          \apref[\amulti {Buyer}][\amulti {Seller}][{i : \asort[{str}] \cdot {o : \asort[int]} }][{l}] \prefop
          \\
          \qquad 
          \apref[\amulti {Seller}][\amulti {Buyer}][{\asort[quit]}][{l}] \prefop
          \\
          \qquad 
          \apref[\amulti {Seller}][@][{{i : \asort[str]} \cdot {p : \asort[int]} \cdot {\nu l : \asort[loc]}}][{\aloc[m]}] \prefop
          \\
          \qquad
          Y	
          \\
          \qquad
          \chop
          \\
          \qquad 
          \apref[\amulti {Seller}][\amulti {Buyer}][{\asort[sold]}][{l}]\prefop
          %	\\
          %	\qquad \apref[\amulti {Seller}][\ptp {broker}][i :
          % {\asort[str]} \cdot o : {\asort[int]} ][{\aloc[m]}]\prefop
          % \\ \qquad
          Y	
          \\
          \qquad
          \chop
          \\
          \qquad
          \apref[\amulti {Seller}][\amulti {Buyer}][{\asort[more]}][{l}] \prefop W
          \\
          \chop
          \\
          \apref[\amulti {Buyer}][\amulti {Seller}][{\asort[{noway}]}][{l}] \prefop
          \\\qquad
          \apref[\amulti {Seller}][@][{{i : \asort[str]} \cdot {p : \asort[int]} \cdot {\nu l : \asort[loc]}}][{\aloc[m]}] \prefop
          \\
          \qquad
          Y
        \end{array}
        }\right)][1]
      \end{array}
      }\right)}][1]
    \end{array}
  \]
  % captures the behaviour sketched above.
  % 
  The broker is a unit role that triggers sellers to start
  advertising their items on the marketplace location $\aloc[m]$.
%
Sellers and buyers are modelled as multiple roles.
%
Each seller advertises one or more items at $\aloc[m]$ (see recursion at line~2).
%
Each buyer can inspect the advertised items (line~3) and eventually
start bargaining on a selected item of interest.
%
Note that the consumption at line~4 instantiates a private location
$l$ between the instance of $\amulti{Seller}$ advertising the item and
the instance of $\amulti{Buyer}$ interested in buying it.
%
Location $l$ is used to perform the bargaining phase.

The seller instance controls the recursion
$\grec[\amulti{Seller}][W][\cdots][1]$; the body of the recursive type
lets the buyer sharing location $l$ decide whether to stop the
bargaining (by exchanging a $\asort[noway]$ tuple, in which case the
seller re-advertises the unsold item at $\aloc[m]$) or to make an
offer to the seller (which can then decide either to stop the
bargaining, or to struck a deal, or to ask for an higher offer).
%
\finex
\end{example}



%%% Local Variables:
%%% mode: latex
%%% TeX-master: "main"
%%% End:

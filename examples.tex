We give a few simple global types (\cref{ex:cs,ex:CS,ex:CSx,ex:CSl})
to highlight some peculiarities of our choreographies as well as a
more complex example (\cref{ex:auction}) to illustrate the type of
protocol our choreographies can capture.

\begin{example}\label{ex:cs}
  Consider the following global type that describes the interaction of
  a client $\ptp c$ with a simple service $\ptp s$ that converts
  integers into strings.
  \[
    \aK_\eqref{ex:cs} =
    \apref[\ptp c][\ptp s][{\asort[int]}][{\aloc[l]}]  \prefop
    \apref[\ptp s][\ptp c][{\asort[str]}][{\aloc[l]}]
  \]
  The client $\ptp c$ produces an integer value on the locality
  $\aloc$.
  %
  This tuple must be consumed by the server $\ptp s$, which produces
  back the converted string for the client.
  %
  \finex
\end{example}

Elaborating on the previous example we now discuss a few features of
our setting.

\begin{example}\label{ex:CS}
  Assume that we consider client and server in \cref{ex:cs} as
  multiroles instead of single participants, and write
  \[
    \aK_\eqref{ex:CS} =
    \apref[\amulti {C}][\amulti {S}][{\asort[int]}][{\aloc[l]}] \prefop
    \apref[\amulti {S}][\amulti {C}][{\asort[str]}][{\aloc[l]}]
  \]
  In this case, $\aK_\eqref{ex:CS}$ states that each integer produced by a client
  will be consumed by a server, which will in turn produce a string
  for one of the clients.
\end{example}
The type in \cref{ex:CS} does not ensure that clients consume the
string conversion of the integer they produced, because all tuples are put at the same location $\aloc[l]$.
%
Name binders can be used to correlate tuples.
%
\begin{example}\label{ex:CSx}
  Consider
  \[
    \aK_\eqref{ex:CSx} =
    \apref[\amulti {C}][\amulti {S}][{\nu x:\asort[int]}][{\aloc[l]}] \prefop
    \apref[\amulti {S}][\amulti {C}][{x : {\asort[int]} \cdot {\asort[str]} }][{\aloc[l]}]
  \]
  $\aK_\eqref{ex:CSx}$ associates a fresh identifier $x$ to each value produced by
  a client and consumed by a server.
  %
  Despite the identifier is known only to the communicating instances,
  this does not forbid two clients to generate the same integer value.
  %
  Basically, the name $x$ allows to express constraints about the flow
  of values.
  %
  In particular, $x : \asort[int]$ in the second interaction states
  that a server must generate a tuple that contains the consumed
  integer.
  %
  Analogously, each client must consume a tuple matching the produced
  integer.
  %
  \finex
\end{example}
The choreography in \cref{ex:CSx} does not establishes a one-to-one
association between instances of $\amulti C$ and $\amulti S$.
%
In fact, an instance of $\amulti C$ not necessarily interacts with the
same instance of $\amulti S$ in the two communications when two
instances of $\amulti C$ generate the same integer in the first
interaction.
%
A one-to-one correspondence can be achieved by using fresh localities.
\begin{example}\label{ex:CSl}
  Consider
  \[
    \aK_\eqref{ex:CSl} = 
    \apref[\amulti {C}][\amulti {S}][{{\asort[int]} \cdot \nu x: {\asort[loc]}}][{\aloc[l]}]  \prefop
    \apref[\amulti {S}][\amulti {C}][{{\asort[str]}}][{x}]
  \]
  %
  A client generates a fresh locality identified by $x$, which is then
  used as the locality for the subsequent communication.
  %
  Since the name $x$ is known only to the two communicating instances,
  the second interaction can only take place between the two instances
  that know the locality $x$.
  %
  \finex
\end{example}

% \todo[inline]{{We think we can also deal with situations like the following, in which 
% A creates a private session for B and C. 
% %
% \[
%   \apref[\amulti {A}][\amulti {B}][{\nu y: {\asort[loc]}}][{\aloc[l]}] \prefop
% \]
% \[
%   \apref[\amulti {A}][\amulti {C}][{y : {\asort[loc]}}][{\aloc[l]}] \prefop
% \]
% \[
%   \apref[\amulti {B}][\amulti {C}][\asort][{y}]
% \]
% }}


The following example shows the kind of protocols that our global
types can model in a realistic scenario allowing us to combine
together most of the features of our framework.
%
For readability, give $h \geq 0$, we use the notation
$\grec[@][@][@][h]$ for a recursive protocol where the body $\aK$ is
repeated at least $h$ times.
%
Formally,
\[
\grec[@][@][@][h] =
\begin{cases}
  \grec[@][@][@] & \text{if } h = 0
  \\
  \aK \sust{X}{\grec[@][@][@][h-1]} & \text{otherwise}
\end{cases}
\]
%
\eMnote{forse non e' un'asta...ma una contrattazione}
\begin{example}[Auction]\label{ex:auction}
\todo{ma \`e lecito usare $\nu$ in $\apref[@][\amulti {Buyer}][{{i : \asort[str]} \cdot {p : \asort[int]} \cdot {\nu l : \asort[loc]}}][{\aloc[m]}]$ o dovrebbe essere $\apref[@][\amulti {Buyer}][{{i : \asort[str]} \cdot {p : \asort[int]} \cdot {l : \asort[loc]}}][{\aloc[m]}]$?}
%
Several sellers and buyers participate to an auction
triggered by a broker.
%
Sellers can offer many items and buyers can inspect the items and,
when an item of interest is found, start a negotiation with the
seller.
\[
\begin{array}{l}
  \apref[\ptp {broker}][\amulti {Seller}][{\atuple[start]}][{\aloc[m]}][.] \prefop
  \\  	
  \grec[][X][{\apref[\amulti {Seller}][@][{{\asort[str]} \cdot {\asort[int]} \cdot {\nu l : \asort[loc]}}][{\aloc[m]}] \prefop X}][1] \seqop
  \\  	
  \grec[\amulti{Buyer}][Y][{\left({
  \begin{array}{l}
    \grec[][Z][{\apref[@][\amulti {Buyer}][{{\asort[str]} \cdot {\asort[int]} \cdot {\asort[loc]}}][{\aloc[m]}][.] \prefop Z}] \seqop
    \\
    \apref[@][\amulti {Buyer}][{{i : \asort[str]} \cdot {p : \asort[int]} \cdot {\nu l : \asort[loc]}}][{\aloc[m]}] \prefop
    \\
    \grec[\amulti{Seller}][W][\left({
    \begin{array}{l}
      \apref[\amulti {Buyer}][\amulti {Seller}][{i : \asort[{str}] \cdot {o : \asort[int]} }][{l}] \prefop
      \\
      \qquad 
      \apref[\amulti {Seller}][\amulti {Buyer}][{\asort[quit]}][{l}] \prefop
      \\
      \qquad 
      \apref[\amulti {Seller}][@][{{i : \asort[str]} \cdot {p : \asort[int]} \cdot {\nu l : \asort[loc]}}][{\aloc[m]}] \prefop
      \\
      \qquad
          Y	
      \\
      \qquad
      \chop
          \\
      \qquad 
      \apref[\amulti {Seller}][\amulti {Buyer}][{\asort[sold]}][{l}]\prefop
      %	\\
      %	\qquad \apref[\amulti {Seller}][\ptp {broker}][i :
      % {\asort[str]} \cdot o : {\asort[int]} ][{\aloc[m]}]\prefop
      % \\ \qquad
      Y	
      \\
      \qquad
      \chop
      \\
      \qquad
      \apref[\amulti {Seller}][\amulti {Buyer}][{\asort[more]}][{l}] \prefop W
      \\
      \chop
      \\
      \apref[\amulti {Buyer}][\amulti {Seller}][{\asort[{noway}]}][{l}] \prefop
      \\\qquad
      \apref[\amulti {Seller}][@][{{i : \asort[str]} \cdot {p : \asort[int]} \cdot {\nu l : \asort[loc]}}][{\aloc[m]}] \prefop
      \\
      \qquad
      Y
    \end{array}
    }\right)][1]
  \end{array}
  }\right)}][1]
\end{array}
\]
%
The above global type captures the behaviour sketched above.
%
The broker is a participant an triggers sellers to start advertising
their items on the marketplace location $\aloc[m]$.
%
Sellers and buyers are modelled as multiroles.
%
Each seller advertises one or more items at $\aloc[m]$
%
Each buyer can inspect the advertised items (line 3) and eventually
start bargaining on a selected item of interest.
%
Note that the consumption at line 4 instantiates a private location
$l$ between the instance of $\amulti{Seller}$ advertising the item and
the instance of $\amulti{Buyer}$ interested in buying it.
%
Location $l$ is used to perform the bargaining phase.

The seller instance controls the recursion
$\grec[\amulti{Seller}][W][\cdots][1]$; the body of the recursive type
lets the buyer sharing location $l$ to decide whether to stop the
bargaining (by exchanging a $\asort[noway]$ tuple, in which case the
seller re-advertises the unsold item at $\aloc[m]$) or to make an
offer to the seller (which can then decide either to stop the
bargaining, or to struck a deal, or ask for an higher offer).
%
\finex
\end{example}



%%% Local Variables:
%%% mode: latex
%%% TeX-master: "main"
%%% End:

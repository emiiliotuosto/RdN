% !TEX root =  main.tex
%

We fix two disjoint sets $\participants = \{\ptp p, \ptp q, \ldots\}$
and $\multiroles = \{\amulti P, \amulti Q, \ldots \}$, respectively of
\emph{participants} and \emph{multiple} roles, and define the set of
\emph{roles} $\roleset = \participants \cup \multiroles$.
% \cup \multiroles^\unknownstarop \cup
% \multiroles^\unknownop$ (ranged over by $\arole, \arole_1,
% \ldots$) where $\multiroles^{\unknownstarop} = \{\unknownstar \sst
% \amulti P \in\multiroles\}$ and $\multiroles^{\unknownop} = \{
% \unknown \sst \amulti p \in \multiroles
% \}$ account for some flexibility when implementing multiroles: an
% action involving a
% $\unknownstar$ role can be optionally executed by an implementation
% of role $\amulti P$, while
% $\unknown$ establishes that exactly one implementer must execute
% that action.
%
We conventionally write multiroles with initial uppercase letter and
unique roles with initial lowercase letter.
%
%We write $\eqR$ for the least equivalence relation on $\roleset$
%satisfying
%\[\amulti P \eqR \unknownstar \hspace{2cm} \amulti P \eqR \unknown\]

Roles are to be thought of as types inhabited by instances of
processes enacting the behaviour specified in a choreography.
%
Participants are unit types while multiroles account for multiple
instances of processes all performing actions according to their role.
%
\todo{RB: forse \`e importante dire che tutti i multiruoli sono disgiunti, giusto?}
\todo{eM: direi di no; in the long run vorremmo che un processo possa coprire piu' ruoli}

%
Let us first define the \emph{prefixes} used in global types
with the following grammar:
% \todo{RB: ``consuming output'' mi sembra fuorviante... ma non ho proposte migliori}
%
\[\begin{array}{lcl@{\qquad\qquad}l}
  \apref & \bnfdef
  & \apref[\arole][@][@][\alocvar] & \text{(autonomous) output}
  \\ & \bnfmid
  & \apref[\arole][{}][@][\alocvar][.] & \text{(autonomous) read-only output}
  \\ & \bnfmid
  & \apref[{}][\arole][@][\alocvar] & \text{(autonomous) input}
  \\ & \bnfmid
  & \apref[{}][\arole][@][\alocvar][.] & \text{(autonomous) read}
  \\ & \bnfmid
  & \apref[\arole][\arole'][@][\alocvar] & \text{consuming interaction}
  \\ & \bnfmid
  & \apref[\arole][\arole'][@][\alocvar][.]  & \text{read-only interaction}
  \end{array}
\]
%
The set $\roles \apref \subseteq \roleset$ of roles in $\apref$ is
defined in the obvious way; note that $\roles \apref$ is a singleton
if, and only if, $\apref$ is an autonomous prefix.
%
We syntactically distinguish two kinds of prefixes.
%
The prefixes generated by the first four productions in the grammar of
$\apref$ above are the \emph{autonomous} prefixes, that is those
prefixes that processes can execute directly on a tuple space without
coordinating with other processes; the prefixes generated by the
remaining two productions of the grammar are the \emph{interaction}
prefixes, namely those involving a role generating tuples and a role
consuming/reading them.
%
Also, tuple types are generated and accessed according to two
modalities syntactically distinguished by the round brackets around
the tuple in prefixes.
%
More precisely, when a prefix surrounds a tuple $\atuple$ with round
brackets then $\atuple$ is meant to be consumed otherwise it is meant
to be read-only.
%
We call \emph{read-only outputs} and \emph{reads}) the autonomous
read-only prefixes, outputs and inputs prefixes those prescribing the
consumption of tuples, and consuming (resp. read-only) interactions
the interations prescribing inputs (resp. reads).

Global types $\aK$ have the following syntax
\eMnote{non e' chiaro se vogliamo $\aK \parop \aK $}
\begin{eqnarray*}
  \aK & \bnfdef & \asum[@][@][][]
                  \bnfmid
                  % \aK \parop \aK \bnfmid
                  \aK \seqop \aK \bnfmid
                  X \bnfmid
                  \grec
\end{eqnarray*}
% \todo{RB: forse il simbolo $\seqop$ \`e misleading: se ho capito bene parlando con Emilio di tratta di un una specie di parallelo asimmetrico, potremmo usare $>$ o qualche simbolo simile?}
where $I$ is a finite set of indexes; we write $\nil$ for
$\asum[@][@][][]$ when $I = \emptyset$ (we omit trailing occurrences
of $\nil$) and $\apref_j.\aK_j$ instead of $\asum[@][@][][]$ when
$I = \{j\}$.
%
The set $\roles \aK$ of roles of $\aK$ is defined in the obvious way.

The syntax of global types features prefix guarded choices, sequential
composition, and recursion.
%
To handle recursive behaviour, the construct $\grec$ singles out a
role $\arole \in \roles \aK$
%
% and specifies an injective function
% $\aphi : \roles \aK \setminus \{\arole\} \to \locset$ such that the
% locations in $\cod(\aphi)$ do not occur in $\aK$; intuitively,
% $\arole$
deciding when the recursion ends.
% and, for all $\arole' \neq \arole$ in $\aK$, the location
% $\aphi(\arole')$ is used to communicate the decision of $\arole$ to
% $\arole'$.
%
We omit the decoration $\arole$ when $\roles \aK = \{\arole\}$.

We extend the notions of defined and free names to global types as
follows:
\todo{RB: ma la $\nu$ pu\`o essere usata anche in una tupla di input/read (senza output)?}
\todo{eM: si, ma poi non sincronizza}
\[
 \fn{\apref[\arole][@][@][\alocvar]}
 = \fn \atuple \cup \{\alocvar \mapsto \asort[loc]\} 
\qquad
 \dn{\apref[\arole][@][@][\alocvar]} 
 = \dn \atuple 
\]
omitted prefixes are defined analogously.
\[
  \begin{array}{ll}
    \begin{array}{l@{\ =\ } ll}
      \fn{\asum[@][@][][]} & \displaystyle{\bigcup_{i\in I}} \fn{\apref_i} \cup (\fn{\aK_i}\setminus\dn{\apref_i})
      \\
      \fn{\aK_1 \seqop \aK_2} 
                &
                  \fn{\aK_1}\cup\fn{\aK_2}
      \\
      \fn \alvar & (\seq \alocvar \cap \varset)
      \\
      \fn \grec & \fn {\aK}
    \end{array}
    \begin{array}{l@{\ =\ } ll}
      \dn{\asum[@][@][][]} & \displaystyle{\bigcup_{i\in I}} \dn{\apref_i} \cup \dn{\aK_i}
      \\
      \dn{\aK_1 \seqop \aK_2} 
                           &
                             \dn{\aK_1}\cup\dn{\aK_2}
      \\
      \dn \alvar & \emptyset
      \\
      \dn \grec & \dn {\aK}
    \end{array}
  \end{array}
\]
%
We write $\names \_$ for the set of sorted names of a term, i.e.,
$\names \apref = \fn\apref \cup \dn\apref$ and similarly
$\names \aK = \fn\aK \cup \dn\aK$. A set $S$ of sorted names is
consistent, written $\consistent S$, if $x\mapsto \asort \in S$ and
$x\mapsto \asort' \in S$ implies $\asort = \asort'$.
 
The set of well-sorted terms are defined inductively as follows:

\begin{itemize}
\item $\apref$ is well-sorted if $\fn\apref\cap\dn\apref = \emptyset$ and  
$\consistent{\names\apref}$, i.e., there are no clashes/inconsistencies in the sorts of 
the names in $\atuple$ and the locality $\alocvar$
\item $\asum[@][@][][]$ is well-sorted if for all ${i\in I}$ both
  $\apref$ and ${\aK_i}$ are well-sorted and
  $\consistent{\names {\apref_i.\aK_i}}$
\item $\aK_1 \seqop \aK_2$ is well-sorted if $\aK_1$ and $\aK_2$ are
  well-sorted and $\consistent{\names{\aK_1 \seqop \aK_2}}$
\item $X$ is well-sorted and $\grec$ is well-sorted if $\aK$ is
  well-sorted.
\end{itemize}


We consider terms up-to $\alpha$-renaming of defined names and
recursion variables.
%
As usual we say that a global type $\aK$ is \emph{closed} when it does
not contain free occurrences of recursion variables $X$ or free
occurrences of names.%, i.e., $\supp{\fn\aK} \cap \varset = \emptyset$.


  
%%% Local Variables:
%%% mode: latex
%%% TeX-master: "main"
%%% End:

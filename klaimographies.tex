%
Let $\locset$ be a set of \emph{localities} and $\asort$ range over
basic sorts which include $\asort[int]$, $\asort[bool]$,
$\asort[str]$, etc.
%
Tuple types have the following syntax:
\begin{eqnarray*}
  \atuple & \bnfdef & \asort \bnfmid
                      \nu \aloc \bnfmid
                      x : \asort \bnfmid
                      \atuple \cdot \atuple \bnfmid
                      \wildcard \bnfmid
\end{eqnarray*}
%
Tuple types \emph{match} according to the relation $\_ \matches \_$ defined as
\[
  \atuple \matches \atuple' \iff
  \begin{cases}
    \atuple = \atuple'  & \text{if $\atuple$ and $\atuple'$ are sorts}
    \\
    \atuple_1 \matches \atuple'_1
    \land \atuple_2 \matches \atuple'_2  & \text{if } \atuple = \atuple_1 \cdot \atuple_2 \text{ and }  \atuple' = \atuple'_1 \cdot \atuple'_2
    \\
    \atuple = \wildcard \vee \atuple' = \wildcard & \text{otherwise}
  \end{cases}
\]

We define
\[
  \atuple \clashes \atuple' \iff \exists \atuple'' \qst \atuple'' \matches \atuple \land \atuple'' \matches \atuple'
\]
%
and say that $\atuple$ and $\atuple'$ \emph{are in conflict} when
$\atuple \clashes \atuple'$.

We fix two disjoint sets $\unitroles$ and $\multiroles$, respectively
of \emph{unique} and \emph{multiple} roles, and define the set of
\emph{roles}
$\roleset = \unitroles \cup \multiroles \cup \{\unknownstar,
\unknown\}$ (ranged over by $\arole, \arole', \ldots$) where
$\unknownstar$ and $\unknown$ are 'wildcards' representing any roles
and at least one role.
%
We conventionally write multiroles with initial uppercase letter and
unique roles with initial lowercase letter.

Global types $\aK$ have the following syntax
\eMnote{non e' chiaro se vogliamo $\aK \parop \aK $}
\begin{eqnarray*}
  \aK & \bnfdef & \asum \bnfmid
                  \aK \parop \aK \bnfmid
                  X \bnfmid
                  \arec
  \\
  \apref & \bnfdef & \apref[\arole] \bnfmid
                         \apref[{}][\arole] \bnfmid
                         \apref[{}][\arole][@][@][.] \bnfmid
                         \apref[\arole][\arole'] \bnfmid
                         \apref[\arole][\arole'][@][\aloc][.]
\end{eqnarray*}
where $I$ is a finite set of indexes and $\aloc \in \locset$ is a
locality.

Local types $\aL$ have the following syntax
\begin{eqnarray*}
  \aL & \bnfdef & \nil \bnfmid
                  \aL \chop \aL \bnfmid
                  \aout[].\aL \bnfmid
                  \ain[].\aL \bnfmid
                  \ard[].\aL \bnfmid
                  X \bnfmid
                  \arec[@][\aL]
\end{eqnarray*}

The projection $\proj$ of a global type $\aK$ on a role $\arole$ is
defined as
\[
  \proj =
  \begin{cases}
    & \text{if } \\
    & \text{if } \\
    & \text{if } \\
    X & \text{if } \aK = X \\
    \arec[@][{\proj[\aK']}] & \text{if } \aK = \arec[@][\aK']
  \end{cases}
\]


%%% Local Variables:
%%% mode: latex
%%% TeX-master: "main"
%%% End:

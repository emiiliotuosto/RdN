This section establishes the correspondence between the denotational
semantics of global types and the operational semantics of local
types.
%
The partial order on the events of a pomset yields an interpretation
of linear executions in terms of \emph{linearisations} similar to
interleaved semantics of concurrent systems.
%
Intuitively a linearisation of a pomset $\apom$ is a sequence of the
events $\esetof \apom$ that preserve the pomset's order
$\leqof \apom$.
%
We show that traces of projections of a global type correspond to
linearisations of its pomset semantics and that for each linearisation
in the pomset semantics there is a system executing a corresponding
trace.
%
We first formalise the notion of linearisation.

Given a pomset $\apom$ and a set of its $\aE \subseteq \esetof \apom$,
a permutation $\ae_1 \cdots \ae_n$ of the events in $\aE$ is a
\emph{linearisation of $\apom$} if
\begin{itemize}
\item $\aE \subseteq \esetof \apom$ preserves $\leqof \apom$ namely
  $\forall 1 \leq i < j \leq n \qst \neg (\ae_j \leqof \apom \ae_i)$
\item each event in $\esetof \apom$ corresponding to an access of an
  interaction is in $\aE$, namely
  if $\ae \in \esetof \apom$ and the tuple type in $\alfof \apom(\ae)$
  if of the form $\arole \cdot \atuple$ then $\ae \in \aE$
\item each output event in $\esetof \apom$ is in $\aE$ and, letting
  $I(\ae)$ be the set of events in $\esetof \apom$ which are labelled
  by inputs of a tuple type matching the one in $\alfof \apom(\ae)$,
  $I(\ae) \cap \aE = \emptyset \iff I(\ae) = \emptyset$
\item accesses in $\ae_1 \cdots \ae_n$ are preceded by a matching
  output, namely ($i$) for each $1 \leq i \leq n$ if $\ae_i$ accesses
  $\atuple$ at $\aloc$ then there is $1 \leq j < i$ such that $\ae_j$
  outputs $\atuple'$ at $\aloc$ with $\atuple' \matches \atuple$ and
  ($ii$) for all $j < h < i$ if $\ae_h$ inputs $\atuple''$ at $\aloc$
  then $\neg (\atuple' \matches \atuple'')$.
\end{itemize}

We say that a sequence $\gtrace$ of labels of events (decorations are
immaterial) is in \emph{normal form} if the defined names of any two
generating labels are disjoint; formally, for all $1 \leq i \neq j \leq n$
%
\[
  \alabel_i \text{ generates
  $\atuple_i$ at $\aloc$ } \land \alabel_j \text{ generates
  $\atuple_j$ at $\aloc$ } \implies \dn{\atuple_i} \cap \dn{\atuple_j}
= \emptyset
\]
%
Also, for $1 \leq i < j \leq n$, we say that $\alabel_j$ \emph{is in
  the scope  the scope of
$\gtrace$ 
is 

%%% Local Variables:
%%% mode: latex
%%% TeX-master: "main"
%%% End:
